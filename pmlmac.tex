%%
%% pmlmac.tex
%% 
%% Made by Alex Nelson <pqnelson@gmail.com>
%% Login   <alex@lisp>
%% 
%% Started on  2025-07-19T10:37:32-0700
%% Last update 2025-07-19T10:37:32-0700
%% 

%% Poor man's LaTeX in plain TeX

\def\makeatletter{\catcode`\@11\relax}

\def\makeatother{\catcode`\@12\relax}

\makeatletter

\newtoks\@temptokena

\def\@empty{}

\def\@ifundefined#1#2#3{%
  \expandafter\ifx\csname#1\endcsname\relax%
    #2%
  \else%
    #3%
  \fi}

\def\@namedef#1{\expandafter\def\csname#1\endcsname}

% HACK: this allows you to \usepackage for some packages
\let\protect\relax
\let\ProcessOptions\relax
\def\DeclareOption#1#2{}
\def\@makeother#1{\catcode`#112\relax}
\def\usepackage#1{\makeatletter
  \input #1.sty}

%% L%% print an L
%% \kern -.36em%% add a negative kern
%% {%% open a group
%%   \sbox \z@ T%% load box 0 with a T
%%   \vbox to\ht \z@ {%% start a vertical box as high as box 0
%%     \hbox {% start a horizontal box
%%       \check@mathfonts%% ensure the math fonts sizes are set up at the current font size
%%       \fontsize \sf@size \z@%% use the established font size for sub/superscripts
%%       \math@fontsfalse%% don't bother setting up all the math fonts for the new current size
%%       \selectfont%% select the font
%%       A%% print an A
%%     }%% finish the horizontal box
%%     \vss%% fill up the stated height
%%   }%% finish the \vbox
%% }%% end the group
%% \kern -.15em%% add a negative kern
%% \TeX%% print the TeX logo

\newdimen\z@ \z@=0pt % can be used both for 0pt and 0
\newcount\@tempcnta % needed for url.sty

\long\def\LaTeX{L\kern-.36em%
  {\sbox{\z@}T%
    \vbox to\ht \z@{\hbox{\sevenrm A}%
      \vss}%
  }%
  \kern-.15em%
  \TeX}


\def\LaTeX{L\kern-.26em \raise.6ex\hbox{\sevenrm A}%
   \kern-.15em\TeX}%

%% \def\LaTeX{L\kern-.26em \raise.6ex\hbox{\fiverm A}%
%%    \kern-.15em TeX}%
\def\AMSTeX{$\cal A\kern-.1667em \lower.5ex\hbox{$\cal M$}%
   \kern-.125em S$-\TeX}%
\def\BibTeX{{\rm B\kern-.05em {\sevenrm I\kern-.025em B}%
   \kern-.08em T\kern-.1667em \lower.7ex\hbox{E}%
   \kern-.125emX}}%
\font\mflogo = logo10
\def\MF{{\mflogo META}{\tenrm \-}{\mflogo FONT}}%

\def\color@begingroup{\begingroup}
\def\color@setgroup{\color@begingroup}
\def\color@endgroup{\endgraf\endgroup}
\long\def\sbox#1#2{\setbox #1\hbox {\color@begingroup #2\color@endgroup}}

%%%
%%% Environments
%%%

% Redefine \bye to use \@@end, so we can redefine \end
\let\@@end\end

% `\enddocument` needs to have an \endgroup to fix
% "semi simple group (level 1) entered at line N (\begingroup)"
\def\bye{\par\vfill\supereject\@@end}
\def\document{}
\def\enddocument{\endgroup\par\bye}


% define environment syntax
% \newif\if@ignore\@ignorefalse does not define things properly
\def\@ignorefalse{\global\let\if@ignore\iffalse}
\def\@ignoretrue {\global\let\if@ignore\iftrue}
\@ignorefalse

\long\def\begin#1{\begingroup\csname#1\endcsname}

\long\def\end#1{\csname end#1\endcsname\endgroup%
  \if@ignore\@ignorefalse\noindent\ignorespaces\fi}

\def\newenvironment#1#2#3{%
  \expandafter\gdef\csname #1\endcsname{#2}%
  \expandafter\gdef\csname end#1\endcsname{#3}%
}

%%%
%%% Counters
%%%
% TODO: support "\theH<counter>"?
\def\stepcounter#1{%
    \expandafter\global\expandafter\advance\csname c@#1\endcsname by1%
    \begingroup%
      \let\@elt\@stpelt%
      \csname cl@#1\endcsname%
    \endgroup%
}

\def\@currentcounter{}

\def\refstepcounter#1{%
  \stepcounter{#1}%
  \xdef\@currentcounter{\csname the#1\endcsname}%
}

\def\label#1{%
  \expandafter\ifx\csname r@#1\endcsname\relax\else%
    \message{Label already defined: #1}%
  \fi%
  % The "r@foo" macros should look like "\def\r@foo{{<\thefoo>}{\thepage}}".
  % The "\noexpand" are inserted to keep "{" and "}" from expanding
  \expandafter\xdef\csname r@#1\expandafter\endcsname\expandafter{%
    \expandafter\noexpand{\@currentcounter\noexpand}%
    \noexpand{\folio\noexpand}%
  }%
\ignorespaces}

\def\ref#1{%
  \ifx\csname r@#1\endcsname\relax%
    \message{Warning: reference #1 on page \folio undefined}{\bf??}%
  \else%
    \expandafter\expandafter\expandafter\@firstoftwo\csname r@#1\endcsname%
  \fi}

\def\eqref#1{(\ref{#1})}

\def\pageref#1{%
  \ifx\csname r@#1\endcsname\relax%
    \message{Warning: reference #1 on page \folio undefined}%
  \else%
    \expandafter\expandafter\expandafter\@secondoftwo\csname r@#1\endcsname%
  \fi}

\def\setcounter#1#2{
    \expandafter\global\csname c@#1\endcsname=#2
}

% @stpelt{<counter>} sets <counter> equal to -1, then invokes
% \stepcounter{<counter>} to propagate resetting
\def\@stpelt#1{%
  \setcounter{#1}{-1}%
  \stepcounter{#1}%
}

% HACK: TeX defines "\newcount" to be outer, which breaks \@definecount
% so we just remove the "\outer" prefix
\def\newcount{\alloc@0\count\countdef\insc@unt}

% Constructs `\cl@<counter>` which is of the form `\@elt <counter1>
% \@elt <counter-2> ... \@elt <counter-N>`
\def\@definecounter#1{\expandafter\newcount\csname c@#1\endcsname
  \setcounter{#1}{0}
  \global\expandafter\let\csname cl@#1\endcsname\@empty
  \expandafter
  \gdef\csname the#1\expandafter\endcsname\expandafter
     {\expandafter\number\csname c@#1\endcsname}
}

\let\newcounter\@definecounter

% linked list operations
\def\@cons#1#2{\begingroup\let\@elt\relax\xdef#1{#1\@elt #2}\endgroup}
\def\@car#1#2\@nil{#1}
\def\@cdr#1#2\@nil{#2}

% \@addtoreset{<foo>}{<bar>} will reset <foo> when <bar> is stepped
\def\@addtoreset#1#2{\expandafter\@cons\csname cl@#2\endcsname {{#1}}}

% ASSUME: #1 and #2 are both counters
\def\@removefromreset#1#2{
  \begingroup
    \expandafter\let\csname c@#1\endcsname\@removefromreset
    \def\@elt##1{%
      \expandafter\ifx\csname c@##1\endcsname\@removefromreset
      \else
        \noexpand\@elt{##1}%
      \fi}%
    \expandafter\xdef\csname cl@#2\endcsname
      {\csname cl@#2\endcsname}%
  \endgroup%
}

%% Pretty printing counters
\def\arabic#1{\expandafter\number\csname c@#1\endcsname}

\def\roman#1{\expandafter\romannumeral\csname c@#1\endcsname}

\def\@slowromancap#1{\ifx @#1\else \if i#1I\else \if v#1V\else \if x#1X\else \if l#1L\else \if c#1C\else \if d#1D\else \if m#1M\else #1\fi \fi \fi \fi \fi \fi \fi \expandafter \@slowromancap \fi}
\def\@Roman#1{\expandafter\@slowromancap\romannumeral#1@}
\def\Roman#1{\expandafter\@Roman\csname c@#1\endcsname}

\def\@alph#1{\ifcase #1\or a\or b\or c\or d\or e\or f\or g\or h\or i\or j\or k\or l\or m\or n\or o\or p\or q\or r\or s\or t\or u\or v\or w\or x\or y\or z\else \@ctrerr \fi}
\def\alph#1{\expandafter\@alph\csname c@#1\endcsname}

\def\@Alph#1{\ifcase #1\or A\or B\or C\or D\or E\or F\or G\or H\or I\or J\or K\or L\or M\or N\or O\or P\or Q\or R\or S\or T\or U\or V\or W\or X\or Y\or Z\else \@ctrerr \fi}
\def\Alph#1{\expandafter\@Alph\csname c@#1\endcsname}

% TODO: fnsymbol

\def\value#1{\csname c@#1\endcsname}

%%%%
%%%% Math related stuff
%%%%

%%
%% Equation environments
%%
\let\normalfont\relax
\let\normalcolor\relax
\@definecounter{equation}
\def\equation{$$\refstepcounter{equation}}
\def\endequation{\eqno \hbox{\@eqnnum}$$\@ignoretrue}
\def\@eqnnum{{\normalfont \normalcolor (\theequation)}}

\expandafter\def\csname equation*\endcsname{%
  \relax\ifmmode
      \@badmath
  \else
      \ifvmode
         \nointerlineskip
         \makebox[.6\linewidth]{}%
      \fi
      $$%                   %  amsthm tries to patch this and expects a $
                            %  will be adjusted when amsthm changes
  \fi
}
\expandafter\def\csname endequation*\endcsname{%
   \relax\ifmmode
      \ifinner
         \@badmath
      \else
         $$
      \fi
   \else
      \@badmath
   \fi
   \ignorespaces\@ignoretrue
}%

%% fractions
\def\frac#1#2{{\begingroup#1\endgroup\over#2}}

\def\stackrel#1#2{\mathrel{\mathop{#2}\limits^{#1}}}

%%%
%%% @ifnextchar
\long\def\@firstoftwo#1#2{#1}
\long\def\@secondoftwo#1#2{#2}

\let\og@colon\:
\def\:{\let\@sptoken= } \:  % this makes \@sptoken a space token
\def\:{\@xifnch} \expandafter\def\: {\futurelet\@let@token\@ifnch}
\let\:\og@colon

\def\@ifnch{%
  \ifx\@let@token\@sptoken
    \let\reserved@c\@xifnch
  \else
    \ifx\@let@token\reserved@d
      \let\reserved@c\reserved@a
    \else
      \let\reserved@c\reserved@b
    \fi
  \fi
  \reserved@c}

\long\def\@ifnextchar#1#2#3{%
  \let\reserved@d=#1%
  \def\reserved@a{#2}%
  \def\reserved@b{#3}%
  \futurelet\@let@token\@ifnch%
}

\def\@ifstar#1{\@ifnextchar*{\@firstoftwo{#1}}}


%% Fonts
\font\tensc=cmcsc10 % caps and small caps
\font\twelverm=cmr12
\font\eightrm=cmr8
\font\sixrm=cmr6 \font\fiverm=cmr5
\font\eighti=cmmi8
\font\ninei=cmmi9  \skewchar\ninei='177
\font\eighti=cmmi8  \skewchar\eighti='177
\font\sixi=cmmi6  \skewchar\sixi='177

\font\tenbi=cmmib10  \skewchar\tenbi='177
\font\ninebi=cmmib9  \skewchar\ninebi='177

\font\ninesy=cmsy9  \skewchar\ninesy='60
\font\eightsy=cmsy8  \skewchar\eightsy='60
\font\sixsy=cmsy6  \skewchar\sixsy='60

\font\tenbsy=cmbsy10  \skewchar\tenbsy='60
\font\sevenbsy=cmbsy7  \skewchar\sevenbsy='60
\font\fivebsy=cmbsy5  \skewchar\fivebsy='60

\font\elevenex=cmex10 scaled\magstephalf
\font\nineex=cmex9
\font\eightex=cmex8
\font\sevenex=cmex7

\font\ninebf=cmbx9
\font\eightbf=cmbx8
\font\sixbf=cmbx6

\font\tenthinbf=cmb10
\font\ninethinbf=cmb10 at 9.25pt
\font\eightthinbf=cmb10 at 8.5pt

\font\twelvett=cmtt12  \hyphenchar\twelvett=-1  % inhibit hyphenation in tt
\font\tensltt=cmsltt10  \hyphenchar\tensltt=-1
\font\ninett=cmtt9  \hyphenchar\ninett=-1
\font\ninesltt=cmsltt10 at 9pt  \hyphenchar\ninesltt=-1
\font\eighttt=cmtt8  \hyphenchar\eighttt=-1
\font\seventt=cmtt8 scaled 875  \hyphenchar\seventt=-1

\font\ninesl=cmsl9
\font\eightsl=cmsl8

\font\nineit=cmti9
\font\eightit=cmti8

\font\eightss=cmssq8
\font\eightssi=cmssqi8
\font\sixss=cmssq8 scaled 800
\font\tenssbx=cmssbx10

\def\footnotesize{\def\rm{\fam0\eightrm}%
  %\clearance=3.9125 pt
  \textfont0=\eightrm \scriptfont0=\sixrm \scriptscriptfont0=\fiverm
  \textfont1=\eighti \scriptfont1=\sixi \scriptscriptfont1=\fivei
  \textfont2=\eightsy \scriptfont2=\sixsy \scriptscriptfont2=\fivesy
  \textfont3=\eightex \scriptfont3=\sevenex \scriptscriptfont3=\sevenex
  \def\it{\fam\itfam\eightit}%
  \textfont\itfam=\eightit
  \def\sl{\fam\slfam\eightsl}%
  \textfont\slfam=\eightsl
  \def\bf{\fam\bffam\eightbf}%
  \textfont\bffam=\eightbf \scriptfont\bffam=\sixbf
   \scriptscriptfont\bffam=\fivebf
  \def\tt{\fam\ttfam\eighttt}%
  \let\sltt=\error
  \textfont\ttfam=\eighttt
  \def\oldstyle{\fam\@ne\eighti}%
  \normalbaselineskip=9pt
  \def\bigfences{\textfont3=\nineex}%
  \let\big=\eightbig
  \let\Big=\eightBig
  \let\bigg=\eightbigg
  \let\Bigg=\eightBigg
  \setbox\strutbox=\hbox{\vrule height7pt depth2pt width\z@}%
  \setbox0=\hbox{$\partial$}%\setbox\ush=\hbox{\rotu0}%
  %\bitmapsize=8pt
  \let\adbcfont=\sixrm
  \let\mc=\sevenrm % for slightly smaller caps
  \let\boldit=\error
  \let\ii=\eightii
  \def\MF{{\manfnt opqr}\-{\manfnt stuq}}%
  \normalbaselines\rm}


%%%
%%% Title
%%%
\def\title#1{\gdef\@title{#1}}
\def\author#1{\gdef\@author{#1}}
\long\def\date#1{\gdef\@date{#1}}
\font\cmman=cmssbx12 scaled\magstep5 
\font\inchhigh=cminch
%\font\foofont=cminch scaled\magstephalf1
\def\@maketitleaddenda{}
\def\@maketitle{\vskip2em%
  %% \edef\@@title{\uppercase{\@title}}
  %% \centerline{\foofont \@@title}%
  \centerline{\cmman \@title}%
  \vskip1.5em\centerline{\twelverm\@author}% author
  \ifx\@date\relax\else\vskip 1em\centerline{\twelverm \@date }\par\fi% date
  \vskip 1.5em%
}
\def\maketitle{\@maketitle
  \mbox{ }\par
  %\vfil\@maketitleaddenda
  \gdef\@maketitle{}
  \mbox{ }\vfill
  \@maketitleaddenda
  \eject}

\def\mbox#1{\leavevmode\hbox{#1}}

%%%
%%% aligned, taken from amstex.tex
%%%
\def\strut@{\copy\strutbox@}
\newbox\strutbox@

\newif\ifinany@
\def\Let@{\relax\iffalse{\fi\let\\=\cr\iffalse}\fi}
\def\aligned{\null\,\vcenter\aligned@}
\def\vspace@{\def\vspace##1{\crcr\noalign{\vskip##1\relax}}}
\def\aligned@{\bgroup\vspace@\Let@
 \ifinany@\else\openup\jot\fi\ialign
 \bgroup\hfil\strut@$\m@th\displaystyle{##}$&
 $\m@th\displaystyle{{}##}$\hfil\crcr}
\def\endaligned{\crcr\egroup\egroup}


\def\center{\centering}
\def\endcenter{}

% graphics
\input epsf
\def\includegraphics{\epsfbox}

\makeatother