%%
%% pmbook.tex
%% 
%% Made by Alex Nelson <pqnelson@gmail.com>
%% Login   <alex@lisp>
%% 
%% Started on  2025-07-28T10:15:54-0700
%% Last update 2025-07-28T10:15:54-0700
%% 

%% Poor man's "book class" macros

% ASSUMES you have you loaded pmlmac.tex
\makeatletter



%% \font\sstext=ecss1000
%% \font\sssub=ecss1000 at 7pt
%% \font\sssubsub=ecss1000 at 5pt

%% \newfam\ssfam
%% \textfont\ssfam=\sstext
%% \scriptfont\ssfam=\sssub
%% \scriptscriptfont\ssfam=\sssubsub

%% \def\sf{\fam\ssfam\sstext} % usually \sf as \ss is ß
%% \def\textsf#1{{\sf #1}}


% blackboard bold https://tex.stackexchange.com/a/156303/14751
\newfam\bbbfam
\font\bbbten=msbm10
\font\bbbseven=msbm7
\font\bbbfive=msbm5
\textfont\bbbfam=\bbbten
\scriptfont\bbbfam=\bbbseven
\scriptscriptfont\bbbfam=\bbbfive
\def\bbb{\fam=\bbbfam}
\def\mathbb#1{{\bbb#1}}

% fraktur
\newfam\frakfam
\font\frakten=eufm10
\font\frakseven=eufm7
\font\frakfive=eufm5
\textfont\frakfam=\frakten
\scriptfont\frakfam=\frakseven
\scriptscriptfont\frakfam=\frakfive
\def\frak{\fam=\frakfam}
\def\mathfrak#1{{\frak#1}}

%%
%% Table of contents

\def\tableofcontents{\begingroup\openin15=toc.tex
  \ifeof15\else\input{toc.tex}\fi\endgroup}

\newwrite\tocfile
\immediate\openout\tocfile={toc2.tex}

\immediate\write\tocfile{\noexpand\chapter*{Contents}\noexpand\begingroup\baselineskip 15pt plus 5pt}

\def\contentsline#1#2#3#4{\csname l@#1\endcsname {#2}{#3}}

\def\addcontentsline#1#2#3#4{\toks0={{#1}{#2}{#3}{#4}}%
\immediate\write\tocfile{\noexpand\contentsline\noexpand{#1\noexpand}\noexpand{#2\noexpand}\noexpand{#3\noexpand}\noexpand{#4\noexpand}}}

\def\l@chapter#1#2{\line{\rm#2\diamondleaders\hfil\hbox to 2em{\hss#1}}}
\def\l@section#1#2{\line{\qquad\rm#2\diamondleaders\hfil\hbox to 2em{\hss#1}}}


\countdef\counter=255
\gdef\diamondleaders{\global\advance\counter by 1
  \ifodd\counter \kern-10pt \fi
  \leaders\hbox to 20pt{\ifodd\counter \kern13pt \else\kern3pt \fi
    .\hss}}

\def\thepage{\folio}

\def\bye{\immediate\write\tocfile{\noexpand\endgroup}%
  \immediate\closeout\tocfile%
  \par\vfill\supereject\@@end}

%%
%% Sections
%%
\newif\iffront\frontfalse
\let\og@advancepageno\advancepageno
\def\frontmatter{\def\folio{\romannumeral\pageno}\fronttrue\gdef\thesection{\arabic{section}}}

\def\mainmatter{\global\frontfalse\gdef\thesection{\thechapter.\arabic{section}}\gdef\advancepageno{\og@advancepageno\gdef\folio{\number\pageno}
\global\let\advancepageno\og@advancepageno}}

%\def\mainmatter{\gdef\folio{\number\pageno}\frontfalse}

\newcounter{chapter}
\newcounter{section}
\newcounter{subsection}
\@addtoreset{section}{chapter}
\@addtoreset{subsection}{section}
\def\thesection{\thechapter.\fi\arabic{section}}
\def\thesubsection{\thesection\Alph{subsection}}
% number equations within each chapter
\@addtoreset{equation}{chapter}
\def\theequation{\thechapter.\arabic{equation}}


\font\titlefont=cmssbx10 scaled\magstep2

\def\s@chapter#1{\vfill\eject%\refstepcounter{chapter}
    \leftline{\twelvess \spaceskip=10pt \def\\{\kern1pt}\phantom{Chapter}}
    \vskip 4pc
    \rightline{\titlefont #1}
    \def\\{}
    \ifx\rhead\omitrhead\else{\ninepoint\xdef\rhead{\uppercase{#1}}}\fi
    \addcontentsline{chapter}{\thepage}{\hbox to 1.5em{\hfil}\enspace #1}{}%
    \vskip 2pc plus 1 pc minus 1 pc
}

\def\@chapter#1{\vfill\eject\iffront\else\refstepcounter{chapter}\fi
    \leftline{\twelvess \spaceskip=10pt \def\\{\kern1pt}\iffront\phantom{Chapter}\else Chapter \thechapter\fi}
    \vskip 4pc
    \rightline{\titlefont #1}
    \def\\{}
    \ifx\rhead\omitrhead\else{\ninepoint\xdef\rhead{\uppercase{#1}}}\fi
    \addcontentsline{chapter}{\thepage}{\hbox to 1.5em{\hfil\iffront\else\thechapter\fi}\enspace #1}{}%
    \vskip 2pc plus 1 pc minus 1 pc
}

\def\chapter{\@ifstar\s@chapter\@chapter}
%{\iffront\s@chapter\@chapter\fi}}

\def\starred{}
\def\starit{\def\starred{\llap{*}}}

\newif\ifrunon\runonfalse 
\newdimen\spaceleft

% Theorem "slogans" should not appear on one page, and then the rest
% of the theorem on the next.
\def\theorembreak{%
  \spaceleft=\vsize%
  \advance\spaceleft by-\pagetotal%
  \ifdim\spaceleft<1in%
    \vfill\eject%
  \else%
    \smallbreak%
  \fi%
}


% if there's less than 2 inches left on the page, just skip to the
% next page to start the section
\def\sectionbreak{%
  \spaceleft=\vsize%
  \advance\spaceleft by-\pagetotal%
  \ifdim\spaceleft<2in%
    \vfill\eject%
  \else%
    \bigbreak%\vskip 2pc plus 1pc minus 5pt%\vskip 1 cm plus 1 pc minus 5 pt%
  \fi%
}

\def\common@section#1#2{%\mark{\currentsection \noexpand\else #1}
  \sectionbreak%
    \refstepcounter{#2}%
    %% \ifrunon \runonfalse\vskip 1 cm plus 1 pc minus 5 pt
    %% \else \vfill\eject
    %%   {\output{\setbox0=\box255}\null\vfill\eject} % set \topmark for sure
    %% \fi
    %\tenpoint
    \leftline{\tenssbx\starred\csname the#2\endcsname. \uppercase{#1}}
    \addcontentsline{section}{\thepage}{\starred\csname the#2\endcsname\enspace #1}{}%
    \def\starred{}%
    \mark{#1\noexpand\else #1}%
    \def\currentsection{#1}%
    %{\ninepoint\xdef\rhead{\uppercase{#2}}}
    \nobreak\smallskip\noindent}

\def\section#1{\common@section{#1}{section}}
\def\subsection#1{\common@section{#1}{subsection}}

%% Poor man's amsthm proof environment
\def\@addpunct#1{\ifnum \spacefactor >\@m \else #1\fi}
\def\openbox{\leavevmode
  \hbox to.77778em{%
  \hfil\vrule
  \vbox to.675em{\hrule width.6em\vfil\hrule}%
  \vrule\hfil}}
\def\qedsymbol{\openbox}

% Knuth's taocpmac.tex uses the following for his qedsymbol equivalent:
\def\slug{\hbox{\kern1.5pt\vrule width2.5pt height6pt depth1.5pt\kern1.5pt}}
% This one looks a little better...
\def\slugg{\hbox{\kern1.25pt\vrule width3pt height6pt depth1.5pt\kern1.25pt}}


% \def\qedsymbol{\sluggg}

\def\qed{%
  \leavevmode\unskip\penalty9999 \hbox{}\nobreak\hfill
  \quad\hbox{\qedsymbol}%
}

\let\QED@stack\@empty
\let\qed@elt\relax

\def\pushQED#1{%
  \toks@{\qed@elt{#1}}\@temptokena\expandafter{\QED@stack}%
  \xdef\QED@stack{\the\toks@\the\@temptokena}%
}

\def\popQED@elt#1#2\relax{#1\gdef\QED@stack{#2}}
\def\popQED{%
  \begingroup\let\qed@elt\popQED@elt \QED@stack\relax\relax\endgroup
}
\def\proofname{Proof}
% knuth uses \it for his proofheadfont
\def\proofheadfont{\tensc}

\def\x@proof[#1]{\pushQED{\qed}\smallbreak%\par
  %\ifdim\lastskip<\medskipamount \removelastskip\penalty55\medskip\fi%\medskip%
  \noindent{\proofheadfont #1\@addpunct{.}} \ignorespaces%
}
\def\@proof{\x@proof[\proofname]}

\def\proof{\@ifnextchar[\x@proof\@proof}
\def\endproof{\popQED}

% \iff is defined to be "\;\Longleftrightarrow\;"
%\def\;{\mskip\thickmuskip}
%\thickmuskip=5mu plus 5mu, 5mu = 5/18 of an em
\ifx\implies\@undefined
  \def\implies{\;\Longrightarrow\;}
\fi
\ifx\impliedby\@undefined
  \def\impliedby{\;\Longlefhtarrow\;}
\fi

\def\property#1{\item\textsc{#1\@addpunct{:}}\ \ignorespaces}

\usepackage{url}

\makeatother
\endinput