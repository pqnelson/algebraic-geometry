%%
%% macros.tex
%% 
%% Made by Alex Nelson <pqnelson@gmail.com>
%% Login   <alex@lisp>
%% 
%% Started on  2025-07-19T10:37:32-0700
%% Last update 2025-07-19T10:37:32-0700
%% 

\makeatletter

%% Strikethrough
%% https://tex.stackexchange.com/a/738717/14751
\def\strikethrough#1{\leavevmode\strikethr@ugh#1 \relax}
\def\strikethr@ugh#1 {\strike@word{#1}\futurelet\next\strike@next}
\def\strike@next{\ifx\next\relax\else\strike@space\expandafter\strikethr@ugh\fi}
\def\strike@word#1{\setbox0=\hbox{#1}\rlap{\strike@rule width\wd0}\box0}
\def\strike@space{\leaders\strike@rule\sp@ce}
\def\strike@rule{\vrule height2.4pt depth-2pt}
\def\sp@ce{\hskip\fontdimen2\font plus\fontdimen3\font minus\fontdimen4\font}

%% Date manipulation
% ISO-8601 date looks like: yyyy-mm-dd
\def\pad@one@zero#1{\ifnum#1<10{0\number#1}\else{\number#1}\fi}
\def\isodate{\number\year-\pad@one@zero{\month}-\pad@one@zero{\day}}
\newif\ifdst\dstfalse                % guess it's not DST by default

\def\@dst@in@oct@or@mar{%
  \begingroup %
% let \r := (\divide\year by100) mod 4
% \ifcase\r \global\anchor=2 \or \global\anchor=0 \or \global\anchor=5 \or%
%           \global\anchor=3 \fi
% let \doomsday := \year mod 100
% \ifodd\doomsday\advance\doomsday by11\fi
% \doomsday=\divide\doomsday by2
% \ifodd\doomsday\advance\doomsday by11\fi
% \doomsday = 7 - (\doomsday mod 7)
% \advance\doomsday\by\r
% \doomsday = \doomsday mod 7
  % count0 = century
  \count0 = \year         %
  \divide\count0 by 100   %
  % count1 = century mod 4
  \count2 = \count0       %
  \divide\count2 by 4     %
  \count1 = \count0       %
  \multiply\count2 by 4   %
  \advance\count1 by-\count2 %
  % count2 = anchor
  \ifcase\count1 %
    \count2=2 \or \count2 = 0 \or \count2=5 \or \count2=3 %
  \fi %
  % count 3 = year - 100*century
  \count4 = \count0           % 
  \multiply\count4 by 100     % count4 = 100*century
  \count3 = \year             %
  \advance\count3 by -\count4 % count3 = year - 100*century
  \ifodd\count3 \advance\count3 by11 \fi  % ifodd count3, count3 += 11
  \divide\count3 by2                      % count3 := count3 / 2
  \ifodd\count3 \advance\count3 by 11 \fi % ifodd count3, count3 += 11
  % count3 = (count3 mod 7)
  \count4=\count3             %
  \divide\count4 by 7         % 
  \multiply\count4 by 7       % count4 = 7 * floor(count3 / 7)
  \advance\count3 by -\count4 %
  % count3 = count3 - 7
  \advance\count3 by -7       %
  \count4=0                   %
  \advance\count4 by -\count3 %
  \count3=\count4 % so now, count3 = 7 - (doomsday mod 7)
  % count3 := count3 + anchor
  \advance\count3 by \count2  %
  % doomsday = doomsday mod 7
  \count4 = \count3           %
  \divide\count4 by 7         %
  \multiply\count4 by 7       % count4 = 7 * floor(count3 / 7)
  \advance\count3 by -\count4 %
  \message{Day of week for Doomsday is \number\count3 ^^J}
  % so now, \count3 is doomsday
  \ifnum\month=10
    % if Oct 10 is Wed or later, then the second Sunday of October
    % is *after* October 10; otherwise, it is *before* October 10.
    % if doomsday < 3, then Second Sunday = 11 - doomsday
    % else Second Sunday = 18 - doomsday
    \ifnum\count3 < 3  %
      \count0 = 10     %
    \else              %
      \count0 = 17     %
    \fi                %
    \advance\count0 by -\count3 % count0 is now second Sunday of October
    \ifnum\day<\count0 %
      \global\dsttrue  %
    \fi                %
  \else%
    \ifnum\month=3     %
      % Second Sunday of March is the Sunday before Mar 14;
      % i.e., 14 - doomsday
      \count0 = 14     %
      \advance\count0 by -\count3% count0 is now second Sunday of March
      \advance\count0 by -1 % count0 is now day BEFORE second Sunday of March
      % so any day AFTER that in March *would* be in Daylight savings time...
      \ifnum\day>\count0 \global\dsttrue \fi %
    \fi %
  \fi %
\endgroup%
}

\ifnum\month=3
  \@dst@in@oct@or@mar 
\else
  \ifnum\month>3
    \ifnum\month<10
      \dsttrue
    \else
      \ifnum\month=10
        \@dst@in@oct@or@mar
      \fi
    \fi
  \fi
\fi

\def\isotz{%$\mathord{-}$
-\ifdst700\else800\fi} % Pacific time zone

% XXX: TeX does not give us the time in seconds, so we just truncate
% the ISO 8601 time down.
\def\isotime{\begingroup%
  \count0 = \time\divide\count0 by 60 %
  \count2 = \count0 % the hour
  \count4 = \time\multiply\count0 by 60 %
  \advance\count4 by -\count0 % the minute
  \pad@one@zero{\count2}:\pad@one@zero{\count4}:00\isotz %
\endgroup}

\def\isodatetime{\isodate T\isotime}

\def\@maketitleaddenda{\mbox{Compiled: {\tt\isodatetime}}}


% \journal[abbrevated name]{full name} will print the abbreviated
% name, if present. If absent, it prints the full name.
% ALSO: there will be no "extra spacing" after periods, to facilitate
% abbreviations properly and ease writing them.
\def\@journal[#1]#2{\emph{\frenchspacing#1}}
\def\@@journal#1{\@journal[#1]{}}
\def\journal{\@ifnextchar[\@journal\@@journal}
% \volume{x}
\def\volume{\textbf}

%%
%% Quotes
%%
\font\eightss=cmssq8
\font\eightssi=cmssqi8
\font\sixss=cmssq8 scaled 800
\font\tenssbx=cmssbx10
\font\twelvess=cmss12


\def\chapterquotes{
  \baselineskip 10pt
  \parfillskip \z@
  \interlinepenalty 10000
  \leftskip \z@ plus 40pc minus \parindent
  \let\rm=\eightss \let\sl=\eightssi \let\adbcfont=\sixss \let\em=\sl
  \everypar{\sl}
  \def\from{\par\nobreak\smallskip\noindent\rm--- }
  \def\author##1(##2){\from ##1\unskip\enspace(##2)}
  \def\\{\hskip.05em} % can say 3\\:\\16
  \obeylines}
\def\endchapterquotes{}


%%
%% Theorems
%%

\newcounter{theorem}
\@addtoreset{theorem}{section}

\def\thetheorem{\thesection.\arabic{theorem}}

% \earmuffed\begin{theorem}...\end{theorem} will place Bourbaki
% earmuffs before the theorem number, and after the theorem
% statement. This works for **any** theorem type.
\newif\ifearmuffed\earmuffedfalse
\def\earmuffed{\earmuffedtrue}

\def\theorembreak{\smallbreak}
% \newtheorem{environment-name}{Printed Name}{body font}
% produces an environment with a mandatory argument (in brackets) for
% the "slogan" to be printed.
% Ex: \begin{definition}[Monoids are unital loops] ... \end{definition}
% Without the optional slogan, the body of the theorem will start on
% the same line as the theorem heading.
\def\newtheorem#1#2#3{
  \expandafter\gdef\csname @#1\endcsname[##1]{% \begin{theorem}[...]
    \theorembreak% \medbreak
    \refstepcounter{theorem}%
    \noindent\ifearmuffed\llap{\future@start}\fi{\bf \thetheorem} {\textsc{#2:}} {\sl##1\@addpunct{.}}\hfill\break#3\ignorespaces%
  }
  \expandafter\gdef\csname @@#1\endcsname{% \begin{theorem} % no bracket
    \theorembreak% \medbreak
    \refstepcounter{theorem}%
    \ifearmuffed\llap{\future@start}\fi%
    \noindent{\bf \thetheorem} {\textsc{#2:}} #3\ignorespaces%
  }
  \expandafter\gdef\csname #1\endcsname{
    \def\unbracketed@thm{\csname @@#1\endcsname}
    \def\bracketed@thm{\csname @#1\endcsname}
    \@ifnextchar[\bracketed@thm\unbracketed@thm
  }
  %% \expandafter\gdef\csname #1\endcsname[##1]{% \begin{theorem}
  %%   \theorembreak% \medbreak
  %%   \refstepcounter{theorem}\noindent{\bf \thetheorem} {\textsc{#2:}} {\sl##1\@addpunct{.}}\hfill\break#3\ignorespaces%
  %% }
  %%% End theorem
  \ifLaTeX
  \expandafter\gdef\csname end#1\endcsname{% \end{theorem}
    \ifearmuffed\llap{\future@end}\fi%
    \global\earmuffedfalse%
    \ifdim\lastskip<\bigskipamount \removelastskip\penalty55\smallskip\fi%
  }
  \else
  \expandafter\gdef\csname end#1\endcsname{% \end{theorem}
    \ifearmuffed\llap{\future@end}\fi%
    \global\earmuffedfalse%
    \@ignoretrue\smallbreak%\medbreak%
  }
  \fi%
}

\newtheorem{theorem}{Theorem}{\sl}
\newtheorem{proposition}{Proposition}{\sl}
\newtheorem{lemma}{Lemma}{\sl}
\newtheorem{corollary}{Corollary}{\sl}
\newtheorem{scheme}{Scheme}{\sl}
\newtheorem{definition}{Definition}{}
\newtheorem{axiom}{Axiom}{\sl}
\newtheorem{axiom-scheme}{Axiom Scheme}{\sl}
\newtheorem{example}{Example}{}
\newtheorem{remark}{Remark}{}

%% \newenvironment{theorem}{% \begin{theorem}
%%   \medbreak
%%   \stepcounter{theorem}\noindent{\bf \thetheorem} {\tensc Theorem \thetheorem.\enspace}\sl\ignorespaces%
%% }{% \end{theorem}
%%     \ifdim\lastskip<\medskipamount \removelastskip\penalty55\medskip\fi%
%% }


%% Math subject classification (stuf)
% \@subjclass[year]{primary, secondary}
\expandafter\ifx\csname subjclass\endcsname\relax
  \def\@subjclass[#1]#2{}
  \def\@@subjclass#1{\@subjclass[2000]{#1}}
  \def\subjclass{\@ifnextchar[\@subjclass\@@subjclass}
\fi


%%%
%%% Commutative diagrams
%%%
\input xy
\xyoption{all}
\SelectTips{cm}{}


%%%
%%% Plain TeX href, since LaTeX will have hyperref available
%%%
\ifLaTeX\else
\def\href#1#2{\leavevmode\special{html:<a href="#1">}#2\special{html:</a>}}


% Allen Kennington's macros for internal referencing
% This \PreHatch macro to prefix a text string with a hatch character.
{\catcode`\^=6 \catcode`\#=12 \gdef\PreHatch^1{#^1}}

% Anchor points for cross-reference hyperlinks.
\def\LinkNameText#1#2{%
 \special{html:<a name="#1">}#2\special{html:</a>}}
\def\LinkNamePRE#1{\special{html:<a name="#1">}}
\def\LinkNamePOST{\special{html:</a>}}
\def\LinkName#1{\LinkNameText{#1}{}}

% Cross-reference hyperlinks to defined anchor points.
\def\LinkHrefText#1#2{%
 \special{html:<a href="\PreHatch{#1}">}#2\special{html:</a>}}

% Pre-text and post-text macros.
\def\LinkHrefPRE#1{\special{html:<a href="\PreHatch{#1}">}}
\def\LinkHrefPOST{\special{html:</a>}}

%\enablehyperlinks
%\enablehyperlinks[dvipdfm]
\iftrue%\expandafter\ifx\csname enablehyperlinks\endcsname\relax
  \message{DEFINING HLDEST}
  % This is the "plain PDF" solution
%%   \def\hldest#1#2#3{\leavevmode\special{pdf: dest (#3)
%% [@thispage /XYZ @xpos @ypos 0]}}
  \def\hldest#1#2#3{\leavevmode\LinkNamePRE{#3}\LinkNamePOST{}}
  \message{DEFINING HLSTART}
  \def\hlstart#1#2#3{\LinkHrefPRE{#3}}
  \message{DEFINING HLEND}
  \def\hlend{\LinkHrefPOST}
\else
  \message{TURNING ON HYPERLINKS}
  \enablehyperlinks[dvipdfm]
  %\hldriver@dvipdfm
  %% \hltype{xyz}
  %% \hldesttype{xyz}
\fi

% create link anchors for labels
\def\labelhook#1{\hldest{name}{}{#1}}
%\LinkNamePRE{#1}\LinkNamePOST{}

% create links when making references
\let\ogref\ref

%% \expandafter\ifx\csname hlstart\endcsname\relax
%%   \message{DEFINING HLSTART}
%%   \def\hlstart#1#2#3{\LinkHrefPRE{#3}}
%%   \message{DEFINING HLEND}
%%   \def\hlend{\LinkHrefPOST}
%% \fi

\def\ref#1{\leavevmode%
  \hlstart{name}{}{#1}\ogref{#1}\hlend}


\fi

%% PDF metadata
\special{pdf: docinfo <<
/Author (Alex Nelson)
/Title (Algebraic Geometry)
/Subject (Algebraic Geometry)
>>}

\makeatother

\endinput


% for Bbb, see https://tex.stackexchange.com/a/236331/14751
\input amstex

\catcode`@=11
\font@\tenbbold=bbold10
\font@\sevenbbold=bbold7
\font@\fivebbold=bbold5
\newfam\bboldfam
\textfont\bboldfam=\tenbbold
\scriptfont\bboldfam=\sevenbbold
\scriptscriptfont\bboldfam=\fivebbold
\def\xbb{\RIfM@\expandafter\xbb@\else
 \expandafter\nonmatherr@\expandafter\xbb\fi}
\def\xbb@#1{{\xbb@@{#1}}}
\def\xbb@@#1{\noaccents@\fam\bboldfam\relax#1}

\let\mathbb\Bbb
