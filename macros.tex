%%
%% macros.tex
%% 
%% Made by Alex Nelson <pqnelson@gmail.com>
%% Login   <alex@lisp>
%% 
%% Started on  2025-07-19T10:37:32-0700
%% Last update 2025-07-19T10:37:32-0700
%% 

\ifLaTeX\else
  %%
%% pmbook.tex
%% 
%% Made by Alex Nelson <pqnelson@gmail.com>
%% Login   <alex@lisp>
%% 
%% Started on  2025-07-28T10:15:54-0700
%% Last update 2025-07-28T10:15:54-0700
%% 

%% Poor man's "book class" macros

%%
%% pmlmac.tex
%% 
%% Made by Alex Nelson <pqnelson@gmail.com>
%% Login   <alex@lisp>
%% 
%% Started on  2025-07-19T10:37:32-0700
%% Last update 2025-07-19T10:37:32-0700
%% 

%% Poor man's LaTeX in plain TeX

\def\makeatletter{\catcode`\@11\relax}

\def\makeatother{\catcode`\@12\relax}

\makeatletter

\newtoks\@temptokena

\def\@empty{}

\def\@ifundefined#1#2#3{%
  \expandafter\ifx\csname#1\endcsname\relax%
    #2%
  \else%
    #3%
  \fi}

\def\@namedef#1{\expandafter\def\csname#1\endcsname}

% HACK: this allows you to \usepackage for some packages
\let\protect\relax
\let\ProcessOptions\relax
\def\DeclareOption#1#2{}
\def\@makeother#1{\catcode`#112\relax}
\def\usepackage#1{\makeatletter
  \input #1.sty}

%% L%% print an L
%% \kern -.36em%% add a negative kern
%% {%% open a group
%%   \sbox \z@ T%% load box 0 with a T
%%   \vbox to\ht \z@ {%% start a vertical box as high as box 0
%%     \hbox {% start a horizontal box
%%       \check@mathfonts%% ensure the math fonts sizes are set up at the current font size
%%       \fontsize \sf@size \z@%% use the established font size for sub/superscripts
%%       \math@fontsfalse%% don't bother setting up all the math fonts for the new current size
%%       \selectfont%% select the font
%%       A%% print an A
%%     }%% finish the horizontal box
%%     \vss%% fill up the stated height
%%   }%% finish the \vbox
%% }%% end the group
%% \kern -.15em%% add a negative kern
%% \TeX%% print the TeX logo

\newdimen\z@ \z@=0pt % can be used both for 0pt and 0
\newcount\@tempcnta % needed for url.sty

\long\def\LaTeX{L\kern-.36em%
  {\sbox{\z@}T%
    \vbox to\ht \z@{\hbox{\sevenrm A}%
      \vss}%
  }%
  \kern-.15em%
  \TeX}


\def\LaTeX{L\kern-.26em \raise.6ex\hbox{\sevenrm A}%
   \kern-.15em\TeX}%

%% \def\LaTeX{L\kern-.26em \raise.6ex\hbox{\fiverm A}%
%%    \kern-.15em TeX}%
\def\AMSTeX{$\cal A\kern-.1667em \lower.5ex\hbox{$\cal M$}%
   \kern-.125em S$-\TeX}%
\def\BibTeX{{\rm B\kern-.05em {\sevenrm I\kern-.025em B}%
   \kern-.08em T\kern-.1667em \lower.7ex\hbox{E}%
   \kern-.125emX}}%
\font\mflogo = logo10
\def\MF{{\mflogo META}{\tenrm \-}{\mflogo FONT}}%

\def\color@begingroup{\begingroup}
\def\color@setgroup{\color@begingroup}
\def\color@endgroup{\endgraf\endgroup}
\long\def\sbox#1#2{\setbox #1\hbox {\color@begingroup #2\color@endgroup}}

%%%
%%% Environments
%%%

% Redefine \bye to use \@@end, so we can redefine \end
\let\@@end\end

% `\enddocument` needs to have an \endgroup to fix
% "semi simple group (level 1) entered at line N (\begingroup)"
\def\bye{\par\vfill\supereject\@@end}
\def\document{}
\def\enddocument{\endgroup\par\bye}


% define environment syntax
% \newif\if@ignore\@ignorefalse does not define things properly
\def\@ignorefalse{\global\let\if@ignore\iffalse}
\def\@ignoretrue {\global\let\if@ignore\iftrue}
\@ignorefalse

\long\def\begin#1{\begingroup\csname#1\endcsname}

\long\def\end#1{\csname end#1\endcsname\endgroup%
  \if@ignore\@ignorefalse\noindent\ignorespaces\fi}

\def\newenvironment#1#2#3{%
  \expandafter\gdef\csname #1\endcsname{#2}%
  \expandafter\gdef\csname end#1\endcsname{#3}%
}

%%%
%%% Counters
%%%
% TODO: support "\theH<counter>"?
\def\stepcounter#1{%
    \expandafter\global\expandafter\advance\csname c@#1\endcsname by1%
    \begingroup%
      \let\@elt\@stpelt%
      \csname cl@#1\endcsname%
    \endgroup%
}

\def\@currentcounter{}

\def\refstepcounter#1{%
  \stepcounter{#1}%
  \xdef\@currentcounter{\csname the#1\endcsname}%
}

\def\label#1{%
  \expandafter\ifx\csname r@#1\endcsname\relax\else%
    \message{Label already defined: #1}%
  \fi%
  % The "r@foo" macros should look like "\def\r@foo{{<\thefoo>}{\thepage}}".
  % The "\noexpand" are inserted to keep "{" and "}" from expanding
  \expandafter\xdef\csname r@#1\expandafter\endcsname\expandafter{%
    \expandafter\noexpand{\@currentcounter\noexpand}%
    \noexpand{\folio\noexpand}%
  }%
\ignorespaces}

\def\ref#1{%
  \ifx\csname r@#1\endcsname\relax%
    \message{Warning: reference #1 on page \folio undefined}{\bf??}%
  \else%
    \expandafter\expandafter\expandafter\@firstoftwo\csname r@#1\endcsname%
  \fi}

\def\eqref#1{(\ref{#1})}

\def\pageref#1{%
  \ifx\csname r@#1\endcsname\relax%
    \message{Warning: reference #1 on page \folio undefined}%
  \else%
    \expandafter\expandafter\expandafter\@secondoftwo\csname r@#1\endcsname%
  \fi}

\def\setcounter#1#2{
    \expandafter\global\csname c@#1\endcsname=#2
}

% @stpelt{<counter>} sets <counter> equal to -1, then invokes
% \stepcounter{<counter>} to propagate resetting
\def\@stpelt#1{%
  \setcounter{#1}{-1}%
  \stepcounter{#1}%
}

% HACK: TeX defines "\newcount" to be outer, which breaks \@definecount
% so we just remove the "\outer" prefix
\def\newcount{\alloc@0\count\countdef\insc@unt}

% Constructs `\cl@<counter>` which is of the form `\@elt <counter1>
% \@elt <counter-2> ... \@elt <counter-N>`
\def\@definecounter#1{\expandafter\newcount\csname c@#1\endcsname
  \setcounter{#1}{0}
  \global\expandafter\let\csname cl@#1\endcsname\@empty
  \expandafter
  \gdef\csname the#1\expandafter\endcsname\expandafter
     {\expandafter\number\csname c@#1\endcsname}
}

\let\newcounter\@definecounter

% linked list operations
\def\@cons#1#2{\begingroup\let\@elt\relax\xdef#1{#1\@elt #2}\endgroup}
\def\@car#1#2\@nil{#1}
\def\@cdr#1#2\@nil{#2}

% \@addtoreset{<foo>}{<bar>} will reset <foo> when <bar> is stepped
\def\@addtoreset#1#2{\expandafter\@cons\csname cl@#2\endcsname {{#1}}}

% ASSUME: #1 and #2 are both counters
\def\@removefromreset#1#2{
  \begingroup
    \expandafter\let\csname c@#1\endcsname\@removefromreset
    \def\@elt##1{%
      \expandafter\ifx\csname c@##1\endcsname\@removefromreset
      \else
        \noexpand\@elt{##1}%
      \fi}%
    \expandafter\xdef\csname cl@#2\endcsname
      {\csname cl@#2\endcsname}%
  \endgroup%
}

%% Pretty printing counters
\def\arabic#1{\expandafter\number\csname c@#1\endcsname}

\def\roman#1{\expandafter\romannumeral\csname c@#1\endcsname}

\def\@slowromancap#1{\ifx @#1\else \if i#1I\else \if v#1V\else \if x#1X\else \if l#1L\else \if c#1C\else \if d#1D\else \if m#1M\else #1\fi \fi \fi \fi \fi \fi \fi \expandafter \@slowromancap \fi}
\def\@Roman#1{\expandafter\@slowromancap\romannumeral#1@}
\def\Roman#1{\expandafter\@Roman\csname c@#1\endcsname}

\def\@alph#1{\ifcase #1\or a\or b\or c\or d\or e\or f\or g\or h\or i\or j\or k\or l\or m\or n\or o\or p\or q\or r\or s\or t\or u\or v\or w\or x\or y\or z\else \@ctrerr \fi}
\def\alph#1{\expandafter\@alph\csname c@#1\endcsname}

\def\@Alph#1{\ifcase #1\or A\or B\or C\or D\or E\or F\or G\or H\or I\or J\or K\or L\or M\or N\or O\or P\or Q\or R\or S\or T\or U\or V\or W\or X\or Y\or Z\else \@ctrerr \fi}
\def\Alph#1{\expandafter\@Alph\csname c@#1\endcsname}

% TODO: fnsymbol

\def\value#1{\csname c@#1\endcsname}

%%%%
%%%% Math related stuff
%%%%

%%
%% Equation environments
%%
\let\normalfont\relax
\let\normalcolor\relax
\@definecounter{equation}
\def\equation{$$\refstepcounter{equation}}
\def\endequation{\eqno \hbox{\@eqnnum}$$\@ignoretrue}
\def\@eqnnum{{\normalfont \normalcolor (\theequation)}}

\expandafter\def\csname equation*\endcsname{%
  \relax\ifmmode
      \@badmath
  \else
      \ifvmode
         \nointerlineskip
         \makebox[.6\linewidth]{}%
      \fi
      $$%                   %  amsthm tries to patch this and expects a $
                            %  will be adjusted when amsthm changes
  \fi
}
\expandafter\def\csname endequation*\endcsname{%
   \relax\ifmmode
      \ifinner
         \@badmath
      \else
         $$
      \fi
   \else
      \@badmath
   \fi
   \ignorespaces\@ignoretrue
}%

%% fractions
\def\frac#1#2{{\begingroup#1\endgroup\over#2}}

\def\stackrel#1#2{\mathrel{\mathop{#2}\limits^{#1}}}

%%%
%%% @ifnextchar
\long\def\@firstoftwo#1#2{#1}
\long\def\@secondoftwo#1#2{#2}

\let\og@colon\:
\def\:{\let\@sptoken= } \:  % this makes \@sptoken a space token
\def\:{\@xifnch} \expandafter\def\: {\futurelet\@let@token\@ifnch}
\let\:\og@colon

\def\@ifnch{%
  \ifx\@let@token\@sptoken
    \let\reserved@c\@xifnch
  \else
    \ifx\@let@token\reserved@d
      \let\reserved@c\reserved@a
    \else
      \let\reserved@c\reserved@b
    \fi
  \fi
  \reserved@c}

\long\def\@ifnextchar#1#2#3{%
  \let\reserved@d=#1%
  \def\reserved@a{#2}%
  \def\reserved@b{#3}%
  \futurelet\@let@token\@ifnch%
}

\def\@ifstar#1{\@ifnextchar*{\@firstoftwo{#1}}}


%% Fonts
\font\tensc=cmcsc10 % caps and small caps
\font\twelverm=cmr12
\font\eightrm=cmr8
\font\sixrm=cmr6 \font\fiverm=cmr5
\font\eighti=cmmi8
\font\ninei=cmmi9  \skewchar\ninei='177
\font\eighti=cmmi8  \skewchar\eighti='177
\font\sixi=cmmi6  \skewchar\sixi='177

\font\tenbi=cmmib10  \skewchar\tenbi='177
\font\ninebi=cmmib9  \skewchar\ninebi='177

\font\ninesy=cmsy9  \skewchar\ninesy='60
\font\eightsy=cmsy8  \skewchar\eightsy='60
\font\sixsy=cmsy6  \skewchar\sixsy='60

\font\tenbsy=cmbsy10  \skewchar\tenbsy='60
\font\sevenbsy=cmbsy7  \skewchar\sevenbsy='60
\font\fivebsy=cmbsy5  \skewchar\fivebsy='60

\font\elevenex=cmex10 scaled\magstephalf
\font\nineex=cmex9
\font\eightex=cmex8
\font\sevenex=cmex7

\font\ninebf=cmbx9
\font\eightbf=cmbx8
\font\sixbf=cmbx6

\font\tenthinbf=cmb10
\font\ninethinbf=cmb10 at 9.25pt
\font\eightthinbf=cmb10 at 8.5pt

\font\twelvett=cmtt12  \hyphenchar\twelvett=-1  % inhibit hyphenation in tt
\font\tensltt=cmsltt10  \hyphenchar\tensltt=-1
\font\ninett=cmtt9  \hyphenchar\ninett=-1
\font\ninesltt=cmsltt10 at 9pt  \hyphenchar\ninesltt=-1
\font\eighttt=cmtt8  \hyphenchar\eighttt=-1
\font\seventt=cmtt8 scaled 875  \hyphenchar\seventt=-1

\font\ninesl=cmsl9
\font\eightsl=cmsl8

\font\nineit=cmti9
\font\eightit=cmti8

\font\eightss=cmssq8
\font\eightssi=cmssqi8
\font\sixss=cmssq8 scaled 800
\font\tenssbx=cmssbx10

\def\footnotesize{\def\rm{\fam0\eightrm}%
  %\clearance=3.9125 pt
  \textfont0=\eightrm \scriptfont0=\sixrm \scriptscriptfont0=\fiverm
  \textfont1=\eighti \scriptfont1=\sixi \scriptscriptfont1=\fivei
  \textfont2=\eightsy \scriptfont2=\sixsy \scriptscriptfont2=\fivesy
  \textfont3=\eightex \scriptfont3=\sevenex \scriptscriptfont3=\sevenex
  \def\it{\fam\itfam\eightit}%
  \textfont\itfam=\eightit
  \def\sl{\fam\slfam\eightsl}%
  \textfont\slfam=\eightsl
  \def\bf{\fam\bffam\eightbf}%
  \textfont\bffam=\eightbf \scriptfont\bffam=\sixbf
   \scriptscriptfont\bffam=\fivebf
  \def\tt{\fam\ttfam\eighttt}%
  \let\sltt=\error
  \textfont\ttfam=\eighttt
  \def\oldstyle{\fam\@ne\eighti}%
  \normalbaselineskip=9pt
  \def\bigfences{\textfont3=\nineex}%
  \let\big=\eightbig
  \let\Big=\eightBig
  \let\bigg=\eightbigg
  \let\Bigg=\eightBigg
  \setbox\strutbox=\hbox{\vrule height7pt depth2pt width\z@}%
  \setbox0=\hbox{$\partial$}%\setbox\ush=\hbox{\rotu0}%
  %\bitmapsize=8pt
  \let\adbcfont=\sixrm
  \let\mc=\sevenrm % for slightly smaller caps
  \let\boldit=\error
  \let\ii=\eightii
  \def\MF{{\manfnt opqr}\-{\manfnt stuq}}%
  \normalbaselines\rm}


%%%
%%% Title
%%%
\def\title#1{\gdef\@title{#1}}
\def\author#1{\gdef\@author{#1}}
\long\def\date#1{\gdef\@date{#1}}
\font\cmman=cmssbx12 scaled\magstep5 
\font\inchhigh=cminch
%\font\foofont=cminch scaled\magstephalf1
\def\@maketitleaddenda{}
\def\@maketitle{\vskip2em%
  %% \edef\@@title{\uppercase{\@title}}
  %% \centerline{\foofont \@@title}%
  \centerline{\cmman \@title}%
  \vskip1.5em\centerline{\twelverm\@author}% author
  \ifx\@date\relax\else\vskip 1em\centerline{\twelverm \@date }\par\fi% date
  \vskip 1.5em%
}
\def\maketitle{\@maketitle
  \mbox{ }\par
  %\vfil\@maketitleaddenda
  \gdef\@maketitle{}
  \mbox{ }\vfill
  \@maketitleaddenda
  \eject}

\def\mbox#1{\leavevmode\hbox{#1}}

%%%
%%% aligned, taken from amstex.tex
%%%
\def\strut@{\copy\strutbox@}
\newbox\strutbox@

\newif\ifinany@
\def\Let@{\relax\iffalse{\fi\let\\=\cr\iffalse}\fi}
\def\aligned{\null\,\vcenter\aligned@}
\def\vspace@{\def\vspace##1{\crcr\noalign{\vskip##1\relax}}}
\def\aligned@{\bgroup\vspace@\Let@
 \ifinany@\else\openup\jot\fi\ialign
 \bgroup\hfil\strut@$\m@th\displaystyle{##}$&
 $\m@th\displaystyle{{}##}$\hfil\crcr}
\def\endaligned{\crcr\egroup\egroup}


\def\center{\centering}
\def\endcenter{}

% graphics
\input epsf
\def\includegraphics{\epsfbox}

\makeatother
\makeatletter


%% Poor man's \emph
\newif\if@emph \@emphfalse
\def\em{\it} % because sometimes it may be necessary to switch to \sl
\def\toggle@emph{\if@emph\@emphfalse\else\@emphtrue\fi}
\def\emph#1{{\if@emph\rm\else\em\fi\toggle@emph #1\/}}

% usual font manipulations
\def\textit#1{{\it #1\/}}
\def\textsl#1{{\sl #1\/}}
\def\textbf#1{{\bf #1}}
\def\texttt#1{{\tt #1}}
\def\textrm#1{{\rm #1}}
\font\tensf=cmss10
\def\textsf#1{{\tensf #1}}
\def\mathcal#1{{\cal #1}}


\font\sften=cmss10
\font\sfseven=cmss7
\font\sffive=cmss5
\newfam\sffam
\textfont\sffam=\sften
\scriptfont\sffam=\sfseven
\scriptscriptfont\sffam=\sffive
\def\sf{\fam\sffam\sften}
\def\textsf#1{{\sf#1}}
\def\text#1{{\rm #1}}

%% \font\sstext=ecss1000
%% \font\sssub=ecss1000 at 7pt
%% \font\sssubsub=ecss1000 at 5pt

%% \newfam\ssfam
%% \textfont\ssfam=\sstext
%% \scriptfont\ssfam=\sssub
%% \scriptscriptfont\ssfam=\sssubsub

%% \def\sf{\fam\ssfam\sstext} % usually \sf as \ss is ß
%% \def\textsf#1{{\sf #1}}


% blackboard bold https://tex.stackexchange.com/a/156303/14751
\newfam\bbbfam
\font\bbbten=msbm10
\font\bbbseven=msbm7
\font\bbbfive=msbm5
\textfont\bbbfam=\bbbten
\scriptfont\bbbfam=\bbbseven
\scriptscriptfont\bbbfam=\bbbfive
\def\bbb{\fam=\bbbfam}
\def\mathbb#1{{\bbb#1}}


%%
%% Table of contents

\def\tableofcontents{\begingroup\openin15=toc.tex
  \ifeof15\else\input{toc.tex}\fi\endgroup}

\newwrite\tocfile
\immediate\openout\tocfile={toc2.tex}

\immediate\write\tocfile{\noexpand\chapter*{Contents}\noexpand\begingroup\baselineskip 15pt plus 5pt}

\def\contentsline#1#2#3#4{\csname l@#1\endcsname {#2}{#3}}

\def\addcontentsline#1#2#3#4{\toks0={{#1}{#2}{#3}{#4}}%
\immediate\write\tocfile{\noexpand\contentsline\noexpand{#1\noexpand}\noexpand{#2\noexpand}\noexpand{#3\noexpand}\noexpand{#4\noexpand}}}

\def\l@chapter#1#2{\line{\rm#2\diamondleaders\hfil\hbox to 2em{\hss#1}}}
\def\l@section#1#2{\line{\qquad\rm#2\diamondleaders\hfil\hbox to 2em{\hss#1}}}


\countdef\counter=255
\gdef\diamondleaders{\global\advance\counter by 1
  \ifodd\counter \kern-10pt \fi
  \leaders\hbox to 20pt{\ifodd\counter \kern13pt \else\kern3pt \fi
    .\hss}}

\def\thepage{\folio}

\def\bye{\immediate\write\tocfile{\noexpand\endgroup}%
  \immediate\closeout\tocfile%
  \par\vfill\supereject\@@end}

%%
%% Sections
%%
\newif\iffront\frontfalse
\let\og@advancepageno\advancepageno
\def\frontmatter{\def\folio{\romannumeral\pageno}\fronttrue\gdef\thesection{\arabic{section}}}

\def\mainmatter{\global\frontfalse\gdef\thesection{\thechapter.\arabic{section}}\gdef\advancepageno{\og@advancepageno\gdef\folio{\number\pageno}
\global\let\advancepageno\og@advancepageno}}

%\def\mainmatter{\gdef\folio{\number\pageno}\frontfalse}

\newcounter{chapter}
\newcounter{section}
\newcounter{subsection}
\@addtoreset{section}{chapter}
\@addtoreset{subsection}{section}
\def\thesection{\thechapter.\fi\arabic{section}}
\def\thesubsection{\thesection\Alph{subsection}}
% number equations within each chapter
\@addtoreset{equation}{chapter}
\def\theequation{\thechapter.\arabic{equation}}


\def\s@chapter#1{\vfill\eject%\refstepcounter{chapter}
    \leftline{\twelvess \spaceskip=10pt \def\\{\kern1pt}\phantom{Chapter}}
    \vskip 4pc
    \rightline{\titlefont #1}
    \def\\{}
    \ifx\rhead\omitrhead\else{\ninepoint\xdef\rhead{\uppercase{#1}}}\fi
    \addcontentsline{chapter}{\thepage}{\hbox to 1.5em{\hfil}\enspace #1}{}%
    \vskip 2pc plus 1 pc minus 1 pc
}

\def\@chapter#1{\vfill\eject\iffront\else\refstepcounter{chapter}\fi
    \leftline{\twelvess \spaceskip=10pt \def\\{\kern1pt}\iffront\phantom{Chapter}\else Chapter \thechapter\fi}
    \vskip 4pc
    \rightline{\titlefont #1}
    \def\\{}
    \ifx\rhead\omitrhead\else{\ninepoint\xdef\rhead{\uppercase{#1}}}\fi
    \addcontentsline{chapter}{\thepage}{\hbox to 1.5em{\hfil\iffront\else\thechapter\fi}\enspace #1}{}%
    \vskip 2pc plus 1 pc minus 1 pc
}

\def\chapter{\@ifstar\s@chapter\@chapter}
%{\iffront\s@chapter\@chapter\fi}}

\def\starred{}
\def\starit{\def\starred{\llap{*}}}

\newif\ifrunon\runonfalse 
\newdimen\spaceleft

% Theorem "slogans" should not appear on one page, and then the rest
% of the theorem on the next.
\def\theorembreak{%
  \spaceleft=\vsize%
  \advance\spaceleft by-\pagetotal%
  \ifdim\spaceleft<1in%
    \vfill\eject%
  \else%
    \smallbreak%
  \fi%
}


% if there's less than 2 inches left on the page, just skip to the
% next page to start the section
\def\sectionbreak{%
  \spaceleft=\vsize%
  \advance\spaceleft by-\pagetotal%
  \ifdim\spaceleft<2in%
    \vfill\eject%
  \else%
    \bigbreak%\vskip 2pc plus 1pc minus 5pt%\vskip 1 cm plus 1 pc minus 5 pt%
  \fi%
}

\def\common@section#1#2{%\mark{\currentsection \noexpand\else #1}
  \sectionbreak%
    \refstepcounter{#2}%
    %% \ifrunon \runonfalse\vskip 1 cm plus 1 pc minus 5 pt
    %% \else \vfill\eject
    %%   {\output{\setbox0=\box255}\null\vfill\eject} % set \topmark for sure
    %% \fi
    %\tenpoint
    \leftline{\tenssbx\starred\csname the#2\endcsname. \uppercase{#1}}
    \addcontentsline{section}{\thepage}{\starred\csname the#2\endcsname\enspace #1}{}%
    \def\starred{}%
    \mark{#1\noexpand\else #1}%
    \def\currentsection{#1}%
    %{\ninepoint\xdef\rhead{\uppercase{#2}}}
    \nobreak\smallskip\noindent}

\def\section#1{\common@section{#1}{section}}
\def\subsection#1{\common@section{#1}{subsection}}

%% Poor man's amsthm proof environment
\def\@addpunct#1{\ifnum \spacefactor >\@m \else #1\fi}
\def\openbox{\leavevmode
  \hbox to.77778em{%
  \hfil\vrule
  \vbox to.675em{\hrule width.6em\vfil\hrule}%
  \vrule\hfil}}
\def\qedsymbol{\openbox}


\def\qed{%
  \leavevmode\unskip\penalty9999 \hbox{}\nobreak\hfill
  \quad\hbox{\qedsymbol}%
}

\let\QED@stack\@empty
\let\qed@elt\relax

\def\pushQED#1{%
  \toks@{\qed@elt{#1}}\@temptokena\expandafter{\QED@stack}%
  \xdef\QED@stack{\the\toks@\the\@temptokena}%
}

\def\popQED@elt#1#2\relax{#1\gdef\QED@stack{#2}}
\def\popQED{%
  \begingroup\let\qed@elt\popQED@elt \QED@stack\relax\relax\endgroup
}
\def\proofname{Proof}
% knuth uses \it for his proofheadfont
\def\proofheadfont{\tensc}

\def\x@proof[#1]{\pushQED{\qed}\smallbreak%\par
  %\ifdim\lastskip<\medskipamount \removelastskip\penalty55\medskip\fi%\medskip%
  \noindent{\proofheadfont #1\@addpunct{.}} \ignorespaces%
}
\def\@proof{\x@proof[\proofname]}

\def\proof{\@ifnextchar[\x@proof\@proof}
\def\endproof{\popQED}

% \iff is defined to be "\;\Longleftrightarrow\;"
%\def\;{\mskip\thickmuskip}
%\thickmuskip=5mu plus 5mu, 5mu = 5/18 of an em
\ifx\implies\@undefined
  \def\implies{\;\Longrightarrow\;}
\fi
\ifx\impliedby\@undefined
  \def\impliedby{\;\Longlefhtarrow\;}
\fi

%%
%% Lists
%%
%\newenvironment{itemize}{\def\item{\par}}{\par}
\def\itemize{\smallbreak%
  %\advance\leftskip\parindent%
  %\def\item{\par\noindent$\bullet$\enspace\@ignoretrue\ignorespaces}%
  \def\@item[##1]{\par\noindent\hang\textindent{##1}}%
  \def\@@item{\@item[$\bullet$]}%
  \def\item{\@ifnextchar[\@item\@@item}%\par\noindent\hang\textindent{$\bullet$}}%
}
\def\enditemize{\@ignoretrue\smallbreak}%\noindent\ignorespaces}

\newcounter{enumi}
\def\enumerate{\smallbreak\setcounter{enumi}{0}%
  \def\item{\par\refstepcounter{enumi}\noindent\hang\textindent{(\theenumi)}}%
}
\def\endenumerate{\@ignoretrue\smallbreak}%\noindent}

\def\textsc#1{{\tensc #1}}
\def\property#1{\item\textsc{#1\@addpunct{:}}}



  
\usepackage{url}

\makeatother
\endinput
\fi

\makeatletter


%% Date manipulation
% ISO-8601 date looks like: yyyy-mm-dd
\def\pad@one@zero#1{\ifnum#1<10{0\number#1}\else{\number#1}\fi}
\def\isodate{\number\year-\pad@one@zero{\month}-\pad@one@zero{\day}}
\newif\ifdst\dstfalse                % guess it's not DST by default

\def\@dst@in@oct@or@mar{%
  \begingroup %
% let \r := (\divide\year by100) mod 4
% \ifcase\r \global\anchor=2 \or \global\anchor=0 \or \global\anchor=5 \or%
%           \global\anchor=3 \fi
% let \doomsday := \year mod 100
% \ifodd\doomsday\advance\doomsday by11\fi
% \doomsday=\divide\doomsday by2
% \ifodd\doomsday\advance\doomsday by11\fi
% \doomsday = 7 - (\doomsday mod 7)
% \advance\doomsday\by\r
% \doomsday = \doomsday mod 7
  % count0 = century
  \count0 = \year         %
  \divide\count0 by 100   %
  % count1 = century mod 4
  \count2 = \count0       %
  \divide\count2 by 4     %
  \count1 = \count0       %
  \multiply\count2 by 4   %
  \advance\count1 by-\count2 %
  % count2 = anchor
  \ifcase\count1 %
    \count2=2 \or \count2 = 0 \or \count2=5 \or \count2=3 %
  \fi %
  % count 3 = year - 100*century
  \count4 = \count0           % 
  \multiply\count4 by 100     % count4 = 100*century
  \count3 = \year             %
  \advance\count3 by -\count4 % count3 = year - 100*century
  \ifodd\count3 \advance\count3 by11 \fi  % ifodd count3, count3 += 11
  \divide\count3 by2                      % count3 := count3 / 2
  \ifodd\count3 \advance\count3 by 11 \fi % ifodd count3, count3 += 11
  % count3 = (count3 mod 7)
  \count4=\count3             %
  \divide\count4 by 7         % 
  \multiply\count4 by 7       % count4 = 7 * floor(count3 / 7)
  \advance\count3 by -\count4 %
  % count3 = count3 - 7
  \advance\count3 by -7       %
  \count4=0                   %
  \advance\count4 by -\count3 %
  \count3=\count4 % so now, count3 = 7 - (doomsday mod 7)
  % count3 := count3 + anchor
  \advance\count3 by \count2  %
  % doomsday = doomsday mod 7
  \count4 = \count3           %
  \divide\count4 by 7         %
  \multiply\count4 by 7       % count4 = 7 * floor(count3 / 7)
  \advance\count3 by -\count4 %
  \message{Day of week for Doomsday is \number\count3 ^^J}
  % so now, \count3 is doomsday
  \ifnum\month=10
    % if Oct 10 is Wed or later, then the second Sunday of October
    % is *after* October 10; otherwise, it is *before* October 10.
    % if doomsday < 3, then Second Sunday = 11 - doomsday
    % else Second Sunday = 18 - doomsday
    \ifnum\count3 < 3  %
      \count0 = 10     %
    \else              %
      \count0 = 17     %
    \fi                %
    \advance\count0 by -\count3 % count0 is now second Sunday of October
    \ifnum\day<\count0 %
      \global\dsttrue  %
    \fi                %
  \else%
    \ifnum\month=3     %
      % Second Sunday of March is the Sunday before Mar 14;
      % i.e., 14 - doomsday
      \count0 = 14     %
      \advance\count0 by -\count3% count0 is now second Sunday of March
      \advance\count0 by -1 % count0 is now day BEFORE second Sunday of March
      % so any day AFTER that in March *would* be in Daylight savings time...
      \ifnum\day>\count0 \global\dsttrue \fi %
    \fi %
  \fi %
\endgroup%
}

\ifnum\month=3
  \@dst@in@oct@or@mar 
\else
  \ifnum\month>3
    \ifnum\month<10
      \dsttrue
    \else
      \ifnum\month=10
        \@dst@in@oct@or@mar
      \fi
    \fi
  \fi
\fi

\def\isotz{%$\mathord{-}$
-\ifdst700\else800\fi} % Pacific time zone

% XXX: TeX does not give us the time in seconds, so we just truncate
% the ISO 8601 time down.
\def\isotime{\begingroup%
  \count0 = \time\divide\count0 by 60 %
  \count2 = \count0 % the hour
  \count4 = \time\multiply\count0 by 60 %
  \advance\count4 by -\count0 % the minute
  \pad@one@zero{\count2}:\pad@one@zero{\count4}:00\isotz %
\endgroup}

\def\isodatetime{\isodate T\isotime}

\def\@maketitleaddenda{\mbox{Compiled: {\tt\isodatetime}}}


% \journal[abbrevated name]{full name} will print the abbreviated
% name, if present. If absent, it prints the full name.
% ALSO: there will be no "extra spacing" after periods, to facilitate
% abbreviations properly and ease writing them.
\def\@journal[#1]#2{\emph{\frenchspacing#1}}
\def\@@journal#1{\@journal[#1]{}}
\def\journal{\@ifnextchar[\@journal\@@journal}
% \volume{x}
\def\volume{\textbf}

%%
%% Quotes
%%
\font\eightss=cmssq8
\font\eightssi=cmssqi8
\font\sixss=cmssq8 scaled 800
\font\tenssbx=cmssbx10
\font\twelvess=cmss12
\font\titlefont=cmssbx10 scaled\magstep2


\def\chapterquotes{
  \baselineskip 10pt
  \parfillskip \z@
  \interlinepenalty 10000
  \leftskip \z@ plus 40pc minus \parindent
  \let\rm=\eightss \let\sl=\eightssi \let\adbcfont=\sixss \let\em=\sl
  \everypar{\sl}
  \def\from{\par\nobreak\smallskip\noindent\rm--- }
  \def\author##1(##2){\from ##1\unskip\enspace(##2)}
  \def\\{\hskip.05em} % can say 3\\:\\16
  \obeylines}
\def\endchapterquotes{}


%%
%% Theorems
%%

\newcounter{theorem}
\@addtoreset{theorem}{section}

\def\thetheorem{\thesection.\arabic{theorem}}

\def\theorembreak{\smallbreak}
% \newtheorem{environment-name}{Printed Name}{body font}
% produces an environment with a mandatory argument (in brackets) for
% the "slogan" to be printed.
% Ex: \begin{definition}[Monoids are unital loops] ... \end{definition}
% Without the optional slogan, the body of the theorem will start on
% the same line as the theorem heading.
\def\newtheorem#1#2#3{
  \expandafter\gdef\csname @#1\endcsname[##1]{% \begin{theorem}
    \theorembreak% \medbreak
    \refstepcounter{theorem}\noindent{\bf \thetheorem} {\textsc{#2:}} {\sl##1\@addpunct{.}}\hfill\break#3\ignorespaces%
  }
  \expandafter\gdef\csname @@#1\endcsname{% \begin{theorem}
    \theorembreak% \medbreak
    \refstepcounter{theorem}\noindent{\bf \thetheorem} {\textsc{#2:}} #3\ignorespaces%
  }
  \expandafter\gdef\csname #1\endcsname{
    \def\unbracketed@thm{\csname @@#1\endcsname}
    \def\bracketed@thm{\csname @#1\endcsname}
    \@ifnextchar[\bracketed@thm\unbracketed@thm
  }
  %% \expandafter\gdef\csname #1\endcsname[##1]{% \begin{theorem}
  %%   \theorembreak% \medbreak
  %%   \refstepcounter{theorem}\noindent{\bf \thetheorem} {\textsc{#2:}} {\sl##1\@addpunct{.}}\hfill\break#3\ignorespaces%
  %% }
  %%% End theorem
  \ifLaTeX
  \expandafter\gdef\csname end#1\endcsname{% \end{theorem}
    \ifdim\lastskip<\bigskipamount \removelastskip\penalty55\smallskip\fi%
  }
  \else
  \expandafter\gdef\csname end#1\endcsname{% \end{theorem}
    \@ignoretrue\smallbreak%\medbreak%
  }
  \fi%
}

\newtheorem{theorem}{Theorem}{\sl}
\newtheorem{proposition}{Proposition}{\sl}
\newtheorem{lemma}{Lemma}{\sl}
\newtheorem{corollary}{Corollary}{\sl}
\newtheorem{scheme}{Scheme}{\sl}
\newtheorem{definition}{Definition}{}
\newtheorem{axiom}{Axiom}{\sl}
\newtheorem{axiom-scheme}{Axiom Scheme}{\sl}
\newtheorem{example}{Example}{}
\newtheorem{remark}{Remark}{}

%% \newenvironment{theorem}{% \begin{theorem}
%%   \medbreak
%%   \stepcounter{theorem}\noindent{\bf \thetheorem} {\tensc Theorem \thetheorem.\enspace}\sl\ignorespaces%
%% }{% \end{theorem}
%%     \ifdim\lastskip<\medskipamount \removelastskip\penalty55\medskip\fi%
%% }


%% Math subject classification (stuf)
% \@subjclass[year]{primary, secondary}
\expandafter\ifx\csname subjclass\endcsname\relax
  \def\@subjclass[#1]#2{}
  \def\@@subjclass#1{\@subjclass[2000]{#1}}
  \def\subjclass{\@ifnextchar[\@subjclass\@@subjclass}
\fi


%%%
%%% Commutative diagrams
%%%
\input xy
\xyoption{all}
\SelectTips{cm}{}
\makeatother


%%%
%%% Plain TeX href, since LaTeX will have hyperref available
%%%
\ifLaTeX\else
\def\href#1#2{\special{html:<a href="#1">}#2\special{html:</a>}}
\fi

\endinput


% for Bbb, see https://tex.stackexchange.com/a/236331/14751
\input amstex
\loadmsbm

\catcode`@=11
\font@\tenbbold=bbold10
\font@\sevenbbold=bbold7
\font@\fivebbold=bbold5
\newfam\bboldfam
\textfont\bboldfam=\tenbbold
\scriptfont\bboldfam=\sevenbbold
\scriptscriptfont\bboldfam=\fivebbold
\def\xbb{\RIfM@\expandafter\xbb@\else
 \expandafter\nonmatherr@\expandafter\xbb\fi}
\def\xbb@#1{{\xbb@@{#1}}}
\def\xbb@@#1{\noaccents@\fam\bboldfam\relax#1}
\let\mathbb\Bbb
