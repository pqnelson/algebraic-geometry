%%
%% macros.tex
%% 
%% Made by Alex Nelson <pqnelson@gmail.com>
%% Login   <alex@lisp>
%% 
%% Started on  2025-07-19T10:37:32-0700
%% Last update 2025-07-19T10:37:32-0700
%% 

\ifLaTeX\else
  %%
%% pmbook.tex
%% 
%% Made by Alex Nelson <pqnelson@gmail.com>
%% Login   <alex@lisp>
%% 
%% Started on  2025-07-28T10:15:54-0700
%% Last update 2025-07-28T10:15:54-0700
%% 

%% Poor man's "book class" macros

% ASSUMES you have you loaded pmlmac.tex
\makeatletter



%% \font\sstext=ecss1000
%% \font\sssub=ecss1000 at 7pt
%% \font\sssubsub=ecss1000 at 5pt

%% \newfam\ssfam
%% \textfont\ssfam=\sstext
%% \scriptfont\ssfam=\sssub
%% \scriptscriptfont\ssfam=\sssubsub

%% \def\sf{\fam\ssfam\sstext} % usually \sf as \ss is ß
%% \def\textsf#1{{\sf #1}}


% blackboard bold https://tex.stackexchange.com/a/156303/14751
\newfam\bbbfam
\font\bbbten=msbm10
\font\bbbseven=msbm7
\font\bbbfive=msbm5
\textfont\bbbfam=\bbbten
\scriptfont\bbbfam=\bbbseven
\scriptscriptfont\bbbfam=\bbbfive
\def\bbb{\fam=\bbbfam}
\def\mathbb#1{{\bbb#1}}

% fraktur
\newfam\frakfam
\font\frakten=eufm10
\font\frakseven=eufm7
\font\frakfive=eufm5
\textfont\frakfam=\frakten
\scriptfont\frakfam=\frakseven
\scriptscriptfont\frakfam=\frakfive
\def\frak{\fam=\frakfam}
\def\mathfrak#1{{\frak#1}}

%%
%% Table of contents

\def\tableofcontents{\begingroup\openin15=toc.tex
  \ifeof15\else\input{toc.tex}\fi\endgroup}

\newwrite\tocfile
\immediate\openout\tocfile={toc2.tex}

\immediate\write\tocfile{\noexpand\chapter*{Contents}\noexpand\begingroup\baselineskip 15pt plus 5pt}

\def\contentsline#1#2#3#4{\csname l@#1\endcsname {#2}{#3}}

\def\addcontentsline#1#2#3#4{\toks0={{#1}{#2}{#3}{#4}}%
\immediate\write\tocfile{\noexpand\contentsline\noexpand{#1\noexpand}\noexpand{#2\noexpand}\noexpand{#3\noexpand}\noexpand{#4\noexpand}}}

\def\l@chapter#1#2{\line{\rm#2\diamondleaders\hfil\hbox to 2em{\hss#1}}}
\def\l@section#1#2{\line{\qquad\rm#2\diamondleaders\hfil\hbox to 2em{\hss#1}}}


\countdef\counter=255
\gdef\diamondleaders{\global\advance\counter by 1
  \ifodd\counter \kern-10pt \fi
  \leaders\hbox to 20pt{\ifodd\counter \kern13pt \else\kern3pt \fi
    .\hss}}

\def\thepage{\folio}

\def\bye{\immediate\write\tocfile{\noexpand\endgroup}%
  \immediate\closeout\tocfile%
  \par\vfill\supereject\@@end}

%%
%% Sections
%%
\newif\iffront\frontfalse
\let\og@advancepageno\advancepageno
\def\frontmatter{\def\folio{\romannumeral\pageno}\fronttrue\gdef\thesection{\arabic{section}}}

\def\mainmatter{\global\frontfalse\gdef\thesection{\thechapter.\arabic{section}}\gdef\advancepageno{\og@advancepageno\gdef\folio{\number\pageno}
\global\let\advancepageno\og@advancepageno}}

%\def\mainmatter{\gdef\folio{\number\pageno}\frontfalse}

\newcounter{chapter}
\newcounter{section}
\newcounter{subsection}
\@addtoreset{section}{chapter}
\@addtoreset{subsection}{section}
\def\thesection{\thechapter.\fi\arabic{section}}
\def\thesubsection{\thesection\Alph{subsection}}
% number equations within each chapter
\@addtoreset{equation}{chapter}
\def\theequation{\thechapter.\arabic{equation}}


\font\titlefont=cmssbx10 scaled\magstep2

\def\s@chapter#1{\vfill\eject%\refstepcounter{chapter}
    \leftline{\twelvess \spaceskip=10pt \def\\{\kern1pt}\phantom{Chapter}}
    \vskip 4pc
    \rightline{\titlefont #1}
    \def\\{}
    \ifx\rhead\omitrhead\else{\ninepoint\xdef\rhead{\uppercase{#1}}}\fi
    \addcontentsline{chapter}{\thepage}{\hbox to 1.5em{\hfil}\enspace #1}{}%
    \vskip 2pc plus 1 pc minus 1 pc
}

\def\@chapter#1{\vfill\eject\iffront\else\refstepcounter{chapter}\fi
    \leftline{\twelvess \spaceskip=10pt \def\\{\kern1pt}\iffront\phantom{Chapter}\else Chapter \thechapter\fi}
    \vskip 4pc
    \rightline{\titlefont #1}
    \def\\{}
    \ifx\rhead\omitrhead\else{\ninepoint\xdef\rhead{\uppercase{#1}}}\fi
    \addcontentsline{chapter}{\thepage}{\hbox to 1.5em{\hfil\iffront\else\thechapter\fi}\enspace #1}{}%
    \vskip 2pc plus 1 pc minus 1 pc
}

\def\chapter{\@ifstar\s@chapter\@chapter}
%{\iffront\s@chapter\@chapter\fi}}

\def\starred{}
\def\starit{\def\starred{\llap{*}}}

\newif\ifrunon\runonfalse 
\newdimen\spaceleft

% Theorem "slogans" should not appear on one page, and then the rest
% of the theorem on the next.
\def\theorembreak{%
  \spaceleft=\vsize%
  \advance\spaceleft by-\pagetotal%
  \ifdim\spaceleft<1in%
    \vfill\eject%
  \else%
    \smallbreak%
  \fi%
}


% if there's less than 2 inches left on the page, just skip to the
% next page to start the section
\def\sectionbreak{%
  \spaceleft=\vsize%
  \advance\spaceleft by-\pagetotal%
  \ifdim\spaceleft<2in%
    \vfill\eject%
  \else%
    \bigbreak%\vskip 2pc plus 1pc minus 5pt%\vskip 1 cm plus 1 pc minus 5 pt%
  \fi%
}

\def\common@section#1#2{%\mark{\currentsection \noexpand\else #1}
  \sectionbreak%
    \refstepcounter{#2}%
    %% \ifrunon \runonfalse\vskip 1 cm plus 1 pc minus 5 pt
    %% \else \vfill\eject
    %%   {\output{\setbox0=\box255}\null\vfill\eject} % set \topmark for sure
    %% \fi
    %\tenpoint
    \leftline{\tenssbx\starred\csname the#2\endcsname. \uppercase{#1}}
    \addcontentsline{section}{\thepage}{\starred\csname the#2\endcsname\enspace #1}{}%
    \def\starred{}%
    \mark{#1\noexpand\else #1}%
    \def\currentsection{#1}%
    %{\ninepoint\xdef\rhead{\uppercase{#2}}}
    \nobreak\smallskip\noindent}

\def\section#1{\common@section{#1}{section}}
\def\subsection#1{\common@section{#1}{subsection}}

%% Poor man's amsthm proof environment
\def\@addpunct#1{\ifnum \spacefactor >\@m \else #1\fi}
\def\openbox{\leavevmode
  \hbox to.77778em{%
  \hfil\vrule
  \vbox to.675em{\hrule width.6em\vfil\hrule}%
  \vrule\hfil}}
\def\qedsymbol{\openbox}

% Knuth's taocpmac.tex uses the following for his qedsymbol equivalent:
\def\slug{\hbox{\kern1.5pt\vrule width2.5pt height6pt depth1.5pt\kern1.5pt}}
% This one looks a little better...
\def\slugg{\hbox{\kern1.25pt\vrule width3pt height6pt depth1.5pt\kern1.25pt}}


% \def\qedsymbol{\sluggg}

\def\qed{%
  \leavevmode\unskip\penalty9999 \hbox{}\nobreak\hfill
  \quad\hbox{\qedsymbol}%
}

\let\QED@stack\@empty
\let\qed@elt\relax

\def\pushQED#1{%
  \toks@{\qed@elt{#1}}\@temptokena\expandafter{\QED@stack}%
  \xdef\QED@stack{\the\toks@\the\@temptokena}%
}

\def\popQED@elt#1#2\relax{#1\gdef\QED@stack{#2}}
\def\popQED{%
  \begingroup\let\qed@elt\popQED@elt \QED@stack\relax\relax\endgroup
}
\def\proofname{Proof}
% knuth uses \it for his proofheadfont
\def\proofheadfont{\tensc}

\def\x@proof[#1]{\pushQED{\qed}\smallbreak%\par
  %\ifdim\lastskip<\medskipamount \removelastskip\penalty55\medskip\fi%\medskip%
  \noindent{\proofheadfont #1\@addpunct{.}} \ignorespaces%
}
\def\@proof{\x@proof[\proofname]}

\def\proof{\@ifnextchar[\x@proof\@proof}
\def\endproof{\popQED}

% \iff is defined to be "\;\Longleftrightarrow\;"
%\def\;{\mskip\thickmuskip}
%\thickmuskip=5mu plus 5mu, 5mu = 5/18 of an em
\ifx\implies\@undefined
  \def\implies{\;\Longrightarrow\;}
\fi
\ifx\impliedby\@undefined
  \def\impliedby{\;\Longlefhtarrow\;}
\fi

\def\property#1{\item\textsc{#1\@addpunct{:}}\ \ignorespaces}

\usepackage{url}

\def\arXiv#1{\texttt{arXiv:#1}}


%% Cases

%% \def\og@cases#1{\left \{\,\vcenter {\normalbaselines \m@th \ialign {$##\hfil $&$\quad ##\hfil$ \crcr #1\crcr }}\right}
%% \def\cases{\og@cases\bgroup}
%% \def\endcases{\egroup}

\makeatother
\endinput
\fi

\makeatletter


%% Date manipulation
% ISO-8601 date looks like: yyyy-mm-dd
\def\pad@one@zero#1{\ifnum#1<10{0\number#1}\else{\number#1}\fi}
\def\isodate{\number\year-\pad@one@zero{\month}-\pad@one@zero{\day}}
\newif\ifdst\dstfalse                % guess it's not DST by default

\def\@dst@in@oct@or@mar{%
  \begingroup %
% let \r := (\divide\year by100) mod 4
% \ifcase\r \global\anchor=2 \or \global\anchor=0 \or \global\anchor=5 \or%
%           \global\anchor=3 \fi
% let \doomsday := \year mod 100
% \ifodd\doomsday\advance\doomsday by11\fi
% \doomsday=\divide\doomsday by2
% \ifodd\doomsday\advance\doomsday by11\fi
% \doomsday = 7 - (\doomsday mod 7)
% \advance\doomsday\by\r
% \doomsday = \doomsday mod 7
  % count0 = century
  \count0 = \year         %
  \divide\count0 by 100   %
  % count1 = century mod 4
  \count2 = \count0       %
  \divide\count2 by 4     %
  \count1 = \count0       %
  \multiply\count2 by 4   %
  \advance\count1 by-\count2 %
  % count2 = anchor
  \ifcase\count1 %
    \count2=2 \or \count2 = 0 \or \count2=5 \or \count2=3 %
  \fi %
  % count 3 = year - 100*century
  \count4 = \count0           % 
  \multiply\count4 by 100     % count4 = 100*century
  \count3 = \year             %
  \advance\count3 by -\count4 % count3 = year - 100*century
  \ifodd\count3 \advance\count3 by11 \fi  % ifodd count3, count3 += 11
  \divide\count3 by2                      % count3 := count3 / 2
  \ifodd\count3 \advance\count3 by 11 \fi % ifodd count3, count3 += 11
  % count3 = (count3 mod 7)
  \count4=\count3             %
  \divide\count4 by 7         % 
  \multiply\count4 by 7       % count4 = 7 * floor(count3 / 7)
  \advance\count3 by -\count4 %
  % count3 = count3 - 7
  \advance\count3 by -7       %
  \count4=0                   %
  \advance\count4 by -\count3 %
  \count3=\count4 % so now, count3 = 7 - (doomsday mod 7)
  % count3 := count3 + anchor
  \advance\count3 by \count2  %
  % doomsday = doomsday mod 7
  \count4 = \count3           %
  \divide\count4 by 7         %
  \multiply\count4 by 7       % count4 = 7 * floor(count3 / 7)
  \advance\count3 by -\count4 %
  \message{Day of week for Doomsday is \number\count3 ^^J}
  % so now, \count3 is doomsday
  \ifnum\month=10
    % if Oct 10 is Wed or later, then the second Sunday of October
    % is *after* October 10; otherwise, it is *before* October 10.
    % if doomsday < 3, then Second Sunday = 11 - doomsday
    % else Second Sunday = 18 - doomsday
    \ifnum\count3 < 3  %
      \count0 = 10     %
    \else              %
      \count0 = 17     %
    \fi                %
    \advance\count0 by -\count3 % count0 is now second Sunday of October
    \ifnum\day<\count0 %
      \global\dsttrue  %
    \fi                %
  \else%
    \ifnum\month=3     %
      % Second Sunday of March is the Sunday before Mar 14;
      % i.e., 14 - doomsday
      \count0 = 14     %
      \advance\count0 by -\count3% count0 is now second Sunday of March
      \advance\count0 by -1 % count0 is now day BEFORE second Sunday of March
      % so any day AFTER that in March *would* be in Daylight savings time...
      \ifnum\day>\count0 \global\dsttrue \fi %
    \fi %
  \fi %
\endgroup%
}

\ifnum\month=3
  \@dst@in@oct@or@mar 
\else
  \ifnum\month>3
    \ifnum\month<10
      \dsttrue
    \else
      \ifnum\month=10
        \@dst@in@oct@or@mar
      \fi
    \fi
  \fi
\fi

\def\isotz{%$\mathord{-}$
-\ifdst700\else800\fi} % Pacific time zone

% XXX: TeX does not give us the time in seconds, so we just truncate
% the ISO 8601 time down.
\def\isotime{\begingroup%
  \count0 = \time\divide\count0 by 60 %
  \count2 = \count0 % the hour
  \count4 = \time\multiply\count0 by 60 %
  \advance\count4 by -\count0 % the minute
  \pad@one@zero{\count2}:\pad@one@zero{\count4}:00\isotz %
\endgroup}

\def\isodatetime{\isodate T\isotime}

\def\@maketitleaddenda{\mbox{Compiled: {\tt\isodatetime}}}


% \journal[abbrevated name]{full name} will print the abbreviated
% name, if present. If absent, it prints the full name.
% ALSO: there will be no "extra spacing" after periods, to facilitate
% abbreviations properly and ease writing them.
\def\@journal[#1]#2{\emph{\frenchspacing#1}}
\def\@@journal#1{\@journal[#1]{}}
\def\journal{\@ifnextchar[\@journal\@@journal}
% \volume{x}
\def\volume{\textbf}

%%
%% Quotes
%%
\font\eightss=cmssq8
\font\eightssi=cmssqi8
\font\sixss=cmssq8 scaled 800
\font\tenssbx=cmssbx10
\font\twelvess=cmss12
\font\titlefont=cmssbx10 scaled\magstep2


\def\chapterquotes{
  \baselineskip 10pt
  \parfillskip \z@
  \interlinepenalty 10000
  \leftskip \z@ plus 40pc minus \parindent
  \let\rm=\eightss \let\sl=\eightssi \let\adbcfont=\sixss \let\em=\sl
  \everypar{\sl}
  \def\from{\par\nobreak\smallskip\noindent\rm--- }
  \def\author##1(##2){\from ##1\unskip\enspace(##2)}
  \def\\{\hskip.05em} % can say 3\\:\\16
  \obeylines}
\def\endchapterquotes{}


%%
%% Theorems
%%

\newcounter{theorem}
\@addtoreset{theorem}{section}

\def\thetheorem{\thesection.\arabic{theorem}}

\def\theorembreak{\smallbreak}
% \newtheorem{environment-name}{Printed Name}{body font}
% produces an environment with a mandatory argument (in brackets) for
% the "slogan" to be printed.
% Ex: \begin{definition}[Monoids are unital loops] ... \end{definition}
% Without the optional slogan, the body of the theorem will start on
% the same line as the theorem heading.
\def\newtheorem#1#2#3{
  \expandafter\gdef\csname @#1\endcsname[##1]{% \begin{theorem}
    \theorembreak% \medbreak
    \refstepcounter{theorem}\noindent{\bf \thetheorem} {\textsc{#2:}} {\sl##1\@addpunct{.}}\hfill\break#3\ignorespaces%
  }
  \expandafter\gdef\csname @@#1\endcsname{% \begin{theorem}
    \theorembreak% \medbreak
    \refstepcounter{theorem}\noindent{\bf \thetheorem} {\textsc{#2:}} #3\ignorespaces%
  }
  \expandafter\gdef\csname #1\endcsname{
    \def\unbracketed@thm{\csname @@#1\endcsname}
    \def\bracketed@thm{\csname @#1\endcsname}
    \@ifnextchar[\bracketed@thm\unbracketed@thm
  }
  %% \expandafter\gdef\csname #1\endcsname[##1]{% \begin{theorem}
  %%   \theorembreak% \medbreak
  %%   \refstepcounter{theorem}\noindent{\bf \thetheorem} {\textsc{#2:}} {\sl##1\@addpunct{.}}\hfill\break#3\ignorespaces%
  %% }
  %%% End theorem
  \ifLaTeX
  \expandafter\gdef\csname end#1\endcsname{% \end{theorem}
    \ifdim\lastskip<\bigskipamount \removelastskip\penalty55\smallskip\fi%
  }
  \else
  \expandafter\gdef\csname end#1\endcsname{% \end{theorem}
    \@ignoretrue\smallbreak%\medbreak%
  }
  \fi%
}

\newtheorem{theorem}{Theorem}{\sl}
\newtheorem{proposition}{Proposition}{\sl}
\newtheorem{lemma}{Lemma}{\sl}
\newtheorem{corollary}{Corollary}{\sl}
\newtheorem{scheme}{Scheme}{\sl}
\newtheorem{definition}{Definition}{}
\newtheorem{axiom}{Axiom}{\sl}
\newtheorem{axiom-scheme}{Axiom Scheme}{\sl}
\newtheorem{example}{Example}{}
\newtheorem{remark}{Remark}{}

%% \newenvironment{theorem}{% \begin{theorem}
%%   \medbreak
%%   \stepcounter{theorem}\noindent{\bf \thetheorem} {\tensc Theorem \thetheorem.\enspace}\sl\ignorespaces%
%% }{% \end{theorem}
%%     \ifdim\lastskip<\medskipamount \removelastskip\penalty55\medskip\fi%
%% }


%% Math subject classification (stuf)
% \@subjclass[year]{primary, secondary}
\expandafter\ifx\csname subjclass\endcsname\relax
  \def\@subjclass[#1]#2{}
  \def\@@subjclass#1{\@subjclass[2000]{#1}}
  \def\subjclass{\@ifnextchar[\@subjclass\@@subjclass}
\fi


%%%
%%% Commutative diagrams
%%%
\input xy
\xyoption{all}
\SelectTips{cm}{}
\makeatother


%%%
%%% Plain TeX href, since LaTeX will have hyperref available
%%%
\ifLaTeX\else
\def\href#1#2{\special{html:<a href="#1">}#2\special{html:</a>}}
\fi

\endinput


% for Bbb, see https://tex.stackexchange.com/a/236331/14751
\input amstex
\loadmsbm

\catcode`@=11
\font@\tenbbold=bbold10
\font@\sevenbbold=bbold7
\font@\fivebbold=bbold5
\newfam\bboldfam
\textfont\bboldfam=\tenbbold
\scriptfont\bboldfam=\sevenbbold
\scriptscriptfont\bboldfam=\fivebbold
\def\xbb{\RIfM@\expandafter\xbb@\else
 \expandafter\nonmatherr@\expandafter\xbb\fi}
\def\xbb@#1{{\xbb@@{#1}}}
\def\xbb@@#1{\noaccents@\fam\bboldfam\relax#1}
\let\mathbb\Bbb
