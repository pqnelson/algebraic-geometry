%%
%% notation.tex
%% 
%% Made by Alex Nelson <pqnelson@gmail.com>
%% Login   <alex@lisp>
%% 
%% Started on  2025-07-19T19:48:26-0700
%% Last update 2025-07-19T19:48:26-0700
%%
\makeatletter

% \Def\macro will raise error if \macro is already defined
\newtoks\tokscheck
\def\isalreadydefined#1{\tokscheck={#1}%
  \expandafter\if\expandafter\relax\tokscheck\relax\else\error\fi}

\def\Def#1{\isalreadydefined{#1}%
  \def#1}

% iff proof
\Def\forwardproof{$(\Longrightarrow)$}
\Def\backwardproof{$(\Longleftarrow)$}

% Bourbaki-style "future topic" earmuffs
\Def\future@start{*}%\smash{\raise0.25ex\hbox{*}}}
\Def\future@end{\smash{\raise-1.0ex\hbox{*}}} % -1.3ex
\Def\future#1{\leavevmode\future@start #1\future@end}

% definiens are defined using the \define{...} macro
\Def\define#1{{\rm\bf``#1''}}

% \needs is just a "note to self"
\Def\needs#1{}

% Latin phrases should be italicized
\Def\Latin#1{\textit{#1\/}}
\Def\French#1{\textit{#1\/}}
\Def\German#1{\textit{#1\/}}

%% Constants like e, etc.
\Def\I{{\rm i}}
\Def\E{{\rm e}}

%% Sets
\Def\CC{\mathbb{C}}
\Def\FF{\mathbb{F}}
\Def\HH{\mathbb{H}}
\Def\NN{\mathbb{N}}
\Def\QQ{\mathbb{Q}}
\Def\Reals{\RR}
\Def\RR{\mathbb{R}}
\Def\ZZ{\mathbb{Z}}
\Def\Integers{\ZZ}
\Def\FinSETS{\textbf{U}_{0}}
\Def\SETS{\textbf{U}_{1}}
\Def\TarskiClass#1{{\bf T}(#1)}
\Def\GrothendieckUniverse#1{{\bf G}(#1)}
\Def\VonNeumannUniverse{{\bf V}}
\Def\UNIVERSE#1{\mathop{\bf U}\nolimits_{#1}}
\Def\succ{\mathop{\rm succ}\nolimits}

%% Set operations
\let\Union\bigcup
\let\Intersects\bigcap
\let\propersubset\subset
\let\og@subset\subset
\let\subset\subseteq
\ifLaTeX
\usepackage{amssymb}
\else
\input amssym.tex
\fi
\Def\powerset#1{\mathcal{P}\left(#1\right)}
\Def\symdiff{\mathbin{\Delta}}

\Def\Complement#1{#1^{\complement}}

%% Relations
\Def\composeRelation{\mathbin{;}}
\Def\rng{\mathop{\rm rng}\nolimits}
\Def\dom{\mathop{\rm dom}\nolimits}

%% partial functions
\Def\pto{\rightharpoonup}

%% functions
\Def\id{\mathop{\rm id}\nolimits}
\Def\incl{\iota}
\Def\into{\hookrightarrow}
\Def\onto{\twoheadrightarrow}
% \preimage{f}{A} = f^{-1}(A) for the preimage of A under f
\Def\preimage#1#2{#1^{-1}(#2)}
\Def\im{\mathop{\rm Im}\nolimits} % image of a function, morphism, etc.
\Def\coim{\mathop{\rm coim}\nolimits} % coimage of a morphism

\let\ImaginaryPart\Im
\let\Im\im

%% Group Theory
\Def\ord{\mathop{\rm ord}\nolimits}
\let\subgroup\leq

%% Topology
\Def\closure#1{\mathop{\rm Cl}\nolimits(#1)}
\Def\interior#1{\mathop{\rm Int}\nolimits(#1)}

%% Category theory
\Def\cod{\mathop{\rm cod}\nolimits}
\Def\op#1{#1^{\text{op}}}
\Def\coker{\mathop{\rm coker}\nolimits}
\Def\cat#1{{\bf#1}} %\mathop{\bf #1} vertically lowers #1
\Def\Ob{\mathop{\rm Ob}\nolimits}
\Def\hom{\mathop{\rm Hom}\nolimits}
\Def\Hom{\hom}
\Def\Mor{\mathop{\rm Mor}\nolimits}
\Def\End{\mathop{\rm End}\nolimits}
\Def\delooping#1{\cat{B}#1}

% natural transformations
\Def\To{\Rightarrow}

% particular categories
\Def\Cat{\cat{Cat}} % category of all small categories
\Def\CAT{\cat{CAT}}
\Def\Ens{\cat{Ens}}
\Def\Grp{\cat{Grp}} % category of groups
\Def\Ab{\cat{Ab}}   % category of Abelian groups
\Def\Set{\cat{Set}} % category of sets
\Def\Top{\cat{Top}} % category of topological spaces

%% Ring Theory
\Def\UnitGroup#1{#1^{\times}}
\Def\Bags{\mathop{\rm Bags}\nolimits}
\Def\EmptyBag{e}
% See (48) in https://tug.org/TUGboat/tb35-3/tb111wermuth.pdf
\Def\monus{\mathbin{\mathop{\kern0pt \smash{-}}%
\limits^{\raise.55ex\hbox{$\textstyle.$}}}}

% ideals
\let\Radical\sqrt
\newcommand\ideal{\mathrel{\trianglelefteq}}
\def\properideal{\mathrel{\triangleleft}}
\Def\Radical#1{\sqrt{#1}}
\Def\NilRadical{\mathop{\rm nil}\nolimits}

\Def\Spec{\mathop{\rm Spec}\nolimits}
\Def\MSpec{\mathop{\rm MSpec}\nolimits}

% Matrices
\Def\Mat#1#2{\mathop{\rm Mat}\nolimits(#1, #2)}

%% Logos
%\def\logo#1{\hbox{\textsf{#1}}}
\Def\logo#1{\hbox{\textsf{#1}}}

\Def\HOL/{\logo{HOL}}
\Def\Mizar/{\logo{Mizar}}

% Axiomatic set theories
\Def\MK/{\logo{MK}}
\Def\NBG/{\logo{NBG}}
\Def\PRA/{\logo{PRA}}
\Def\TG/{\logo{TG}}
\Def\Z/{\logo{Z}}
\Def\ZF/{\logo{ZF}}
\Def\ZFC/{\logo{ZFC}}
\Def\ZFpmC/{\logo{ZF{\rm(}C{\rm)}}}

%% Notation for structures
\Def\structure#1{\langle\!\langle#1\rangle\!\rangle}
% poor man's U+2AAA \smt symbol https://tex.stackexchange.com/a/712877/14751
\def\angin@#1#2{
    \vphantom{#2}%
    \ooalign{$\m@th#1#2$\cr$\m@th#1-$\cr}%
}
% unicode-math package defines \smt; it's an acronym/abbreviation
%``SMaller Than''
% It shows up in the writings of Boole and Pierce
% https://tex.stackexchange.com/q/482843/14751
\def\smt{\mathrel{\mathpalette\angin@{<}}} % U+02AAA
\def\lat{\mathrel{\mathpalette\angin@{>}}} % U+02AAB

\font\sidenotefont=cmti7

% https://tex.stackexchange.com/a/86088/14751
% See also exercise 14.28 of the TeXbook
\long\def\sidenote#1{%
  \vadjust{\llap{\smash{\vtop{%
    \parindent=0pt
    %\hsize=0.85in %1.7in
    \hsize=6.5pc%1in
    \advance\hsize by-1em
    \parfillskip=0pt
    \leftskip=0pt plus1fil
    \baselineskip=10pt\sidenotefont\vglue-\ht\strutbox #1}}\kern0.6667em}}}

% \mml is used to refer to locations in the Mizar Mathematical Library
\def\mml{\catcode`\_11\relax\@mml}
\ifLaTeX
  \usepackage[fulladjust]{marginnote}
  %  \marginparwidth=1in
  \renewcommand*{\marginfont}{\small\upshape\ttfamily}
  \def\@mml#1{\marginnote{#1}\catcode`\_8\ignorespaces}
\else
  %\def\@mml#1{{\font\sidenotefont=cmtt10 scaled 800\sidenote{#1}}\catcode`\_8\ignorespaces}
  \def\@mml#1{{\font\sidenotefont=cmtt8\sidenote{#1}}\catcode`\_8\ignorespaces}
\fi

\makeatother
\endinput