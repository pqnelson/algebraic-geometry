%%
%% linear-algebra.tex
%% 
%% Made by Alex Nelson <pqnelson@gmail.com>
%% Login   <alex@lisp>
%% 
%% Started on  2025-10-26T13:07:18-0700
%% Last update 2025-10-26T13:07:18-0700
%% 

\chapter{Linear Algebra}

\begin{remark}[Modules over other stuff]
We can consider ``modules over monads'', or even ``modules over more
general stuff''. All we really need to define a module $M$ over $X$ is an
action of $X$ on $M$ which is ``compatible'' with the structure of
$M$. In fact, we could internalize the notion of a ``left module
object'' internal to \emph{any} monoidal category. Grothendieck first
did this in FGA III.

This would allow us to define, for example, modules over algebras,
modules over Lie algebras, modules over\dots well, you get the
idea. See also Daniel Quillen's ``Module theory over nonunital rings''
(unpublished manuscript, dated August 1996, \url{https://ncatlab.org/nlab/files/QuillenModulesOverRngs.pdf}),
and Positselski's ``Tensor-Hom formalism for modules over nonunital
rings'' (\arXiv{2308.16090}).
\end{remark}

\begin{remark}[Modules are an Abelian group acted upon by something else]
The fundamental idea underlying a module is that it consists of an
Abelian group (written additively) equipped with an action upon it. 
We could have a module $M$ over a set $S$ defined by an action
$\mu\colon S\times M\to M$ such that for each $s\in S$, $\mu(s,-)$ is
an endomorphism of $M$. When $S$ is an associative magma,
we have $\mu(xy,m)=\mu(x,\mu(y,m))$ for all $x,y\in S$ and $m\in M$.
When $S$ is further unital, then $\mu(1,-)=\id_{M}$.

Readers familiar with ``groups with operators'' should see a strong
resemblence here. It's no mistake, either. But it's only with the
benefit of hindsight that we realize these ``operators'' are precisely
elements of a set which act on the group. Since this action is usually
associative, it gives a natural representation of a monoid on a group.
\end{remark}

\section{Modules over rings}

\begin{definition}[(Left) Module structure]\mml{vectsp_1}
Let $L$ be a 1-sorted structure.
We define the \define{Module Structure} over $L$ to extend the
additive loop structure, $M=\structure{U,+,0,\star_{L}}$ where
$\star_{L}\colon L\times U\to U$ is the \define{Left Scalar Multiplication}.
\end{definition}

\begin{definition}[Left module over a ring]\mml{vectsp_2}
Let $R$ be a ring.
We define [the soft type of] a \define{Left Module} of $R$ to be a
nonempty module structure $M$ over $R$ such that
\begin{itemize}
\property{Right Complementable}\mml{algstr_0:def 16} for all $x\in M$, there exists a $y\in M$ such that $x+y=0$
\property{Vector distributive}\mml{vectsp_1:def 13} for any scalar
$x\in R$, for any vectors $v\in M$ and $w\in M$, we have $x(v+w)=(xv)+(xw)$
\property{Scalar distributive}\mml{vectsp_1:def 14} for any scalars
$x\in R$ and $y\in R$, for any vector $v\in M$, we have $(x+y)v=(xv)+(yv)$
\property{Scalar associative}\mml{vectsp_1:def 15} for any scalars
$x\in R$ and $y\in R$, and for any vector $v\in M$, we have $(xy)v=x(yv)$
\property{Scalar unital}\mml{vectsp_1:def 16} For any vector $v\in M$, we have $1_{R}v=v$
\property{Vector addition is commutative}\mml{rlvect_1:def 2} For any vectors
$u\in M$, $v\in M$, we have $u+v=v+u$
\property{Vector addition is associative}\mml{rlvect_1:def 3} For any vectors
$u\in M$, $v\in M$, $w\in M$, we have $(u+v)+w=u+(v+w)$
\property{Right zeroed}\mml{rlvect_1:def 4} For any vector $v\in M$,
we have $v+0=v$.
\end{itemize}
\end{definition}

\begin{example}
Let $R$ be any associative unital ring. Then $R$ is a left module over itself.
\end{example}

\begin{definition}[Right scalar multiplication for left modules over commutative rings]
Let $R$ be a commutative ring. Let $M$ be a left module over $R$.
Let $m$ be an element of $M$, let $r$ be an element of $R$.
We can define the \define{Right Scalar Multiplication} of $R$ on $M$
by $m\star_{R}r := r\star_{L}m$. This makes sense since multiplication
is commutative in the ring.
\end{definition}

\begin{definition}[Vector space over a field]\mml{vectsp_1}
Let $\FF$ be a field. We define [the soft type of] a \define{Vector Space}
over $\FF$ to be a left module over $\FF$.
\end{definition}

\begin{definition}[Right module structure]\mml{vectsp_2}
Let $R$ be a 1-sorted structure. We define the \define{Right Module Structure}
over $R$ to extend the additive loop structure, being $M=\structure{U,+,0,\star_{R}}$
where $\star_{R}\colon U\times R\to U$ is the \define{Right Scalar Multiplication}
operation (usually we just write $vr$ instead of $v\star_{R}r$).
\end{definition}

\begin{definition}[Right module over a ring]\mml{vectsp_2}
Let $R$ be a ring.
We define [the soft type of] a \define{Right Module} over $R$
to be a nonempty right module structure $M$ over $R$ such that
\begin{itemize}
\property{Right Complementable}\mml{algstr_0:def 16} for all $x\in M$, there exists a $y\in M$ such that $x+y=0$
\property{Vector addition is commutative}\mml{rlvect_1:def 2} For any vectors
$u\in M$, $v\in M$, we have $u+v=v+u$
\property{Vector addition is associative}\mml{rlvect_1:def 3} For any vectors
$u\in M$, $v\in M$, $w\in M$, we have $(u+v)+w=u+(v+w)$
\property{Right zeroed}\mml{rlvect_1:def 4} For any vector $v\in M$,
we have $v+0=v$
\property{Right-module-like}\mml{vectsp_2:def 8} For any scalars $x\in R$
and $y\in R$, for any vectors $v\in M$ and $w\in M$, we have
$(v+w)x=(vx)+(wx)$ and $v(x+y)=(vx)+(vy)$ and $v(yx)=(vy)x$ and $v1_{R}=v$.
\end{itemize}
\end{definition}

\begin{example}
Let $R$ be any associative unital ring. 
Then $R$ is a right module over itself.
\end{example}

\begin{definition}[Bi-Module structure]\mml{vectsp_2}
Let $L$, $R$ be 1-sorted structures.
We define the structure \define{Bi-Module Structure} over $L$, $R$ to
be $B=\structure{U,+,0,\star_{L},\star_{R}}$
extending (1) the left module structure over $L$, and (2) the right
module structure over $R$;
where
we have left scalar multiplication $\star_{L}\colon L\times U\to U$
and right scalar multiplication $\star_{R}\colon U\times R\to U$.
\end{definition}

\begin{definition}[Bimodule over rings]\mml{vectsp_2}
Let $R_{1}$ and $R_{2}$ be rings.
We define [the soft type of] a \define{Bimodule} over $R_{1}$, $R_{2}$
to be a nonempty Bimodule structure $M$ over $R_{1}$, $R_{2}$ such
that
\begin{itemize}
\property{Vector distributive}\mml{vectsp_1:def 13} for any scalar
$x\in R_{1}$, for any vectors $v\in M$ and $w\in M$, we have $x(v+w)=(xv)+(xw)$
\property{Scalar distributive}\mml{vectsp_1:def 14} for any scalars
$x\in R_{1}$ and $y\in R_{1}$, for any vector $v\in M$, we have $(x+y)v=(xv)+(yv)$
\property{Scalar associative}\mml{vectsp_1:def 15} for any scalars
$x\in R_{1}$ and $y\in R_{1}$, and for any vector $v\in M$, we have $(xy)v=x(yv)$
\property{Scalar unital}\mml{vectsp_1:def 16} For any vector $v\in M$, we have $1_{R_{1}}v=v$
\property{Right Complementable}\mml{algstr_0:def 16} for all $x\in M$, there exists a $y\in M$ such that $x+y=0$
\property{Vector addition is commutative}\mml{rlvect_1:def 2} For any vectors
$u\in M$, $v\in M$, we have $u+v=v+u$
\property{Vector addition is associative}\mml{rlvect_1:def 3} For any vectors
$u\in M$, $v\in M$, $w\in M$, we have $(u+v)+w=u+(v+w)$
\property{Right zeroed}\mml{rlvect_1:def 4} For any vector $v\in M$,
we have $v+0=v$
\property{Right-module-like}\mml{vectsp_2:def 8} For any scalars $x\in R_{2}$
and $y\in R_{2}$, for any vectors $v\in M$ and $w\in M$, we have
$(v+w)x=(vx)+(wx)$ and $v(x+y)=(vx)+(vy)$ and $v(yx)=(vy)x$ and $v1_{R_{2}}=v$
\property{Bi-module-like}\mml{vectsp_2:def 9} For any scalars $x\in R_{1}$
and $y\in R_{2}$, for any vector $v\in M$, we have $x(vy)=(xv)y$.
\end{itemize}
\end{definition}

\begin{remark}[Bimodules are left and right modules]
Observe that a bimodule $M$ over $L$, $R$ is automatically a left
module over $L$ \emph{and simultaneously} a right module over $R$.
\end{remark}

\begin{remark}[Notation and nomenclature for modules]
It is not uncommon to see $L$-modules for left modules of $L$, and
module-$R$ for right modules of $R$, and $L$-modules-$R$ for a
bimodule over $L$, $R$. There are variations on this notation (e.g.,
using subscripts to say ${}_{L}$Mod${}_{R}$ for a bimodule over $L$,
$R$, etc.).
\end{remark}

\begin{example}
Let $R$ be any associative unital ring. 
Then $R$ is a bimodule over itself.
\end{example}