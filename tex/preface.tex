%%
%% preface.tex
%% 
%% Made by Alex Nelson <pqnelson@gmail.com>
%% Login   <alex@lisp>
%% 
%% Started on  2025-07-19T19:23:26-0700
%% Last update 2025-07-19T19:23:26-0700
%% 

\chapter{Preface}

\begin{remark}[Structure of this manuscript]
This is my collection of notes on stuff related to algebraic geometry,
and possibly other subjects in pure mathematics. Algebraic geometry is
a horribly presented subject, since it has evolved in ``waves''. Each
``wave'' reused the same terminology, but assigned different meanings
to the terms (or so it seems to me, as an outsider).

I am hoping to make sense of it. I am organizing my notes into
``chunks'', following Alan Kennington's writing style in his book on
differential geometry.

Of course, I am doing this with an eye towards formalizing the
mathematics in \Mizar/. So the proofs are probably going to be written
in a ``pidgin \Mizar/'', if you will.\mml{MML ref} Consequently, I will also be
writing references to articles appearing in the Mizar Mathematical
Library in the left margin. This is where the relevant result may be found.
\end{remark}

\begin{remark}[Types of definitions]
Again, following \Mizar/, we will use their taxonomy of definitions:
\begin{itemize}
\item[$-$] Predicates produce formulas from terms. They are always
  well-defined. 
\item[$-$] Term constructors (or \emph{functors}) produce terms
  parametrized by finitely many terms. We need to prove this is
  well-defined (abbreviations need to be proven to have the correct
  type, ``standard'' term definitions need to be proven to exist and
  be unique).
\item[$-$] Soft Types are predicates in the ambient logic; these are
  dependent types. We need to prove there exists at least one term
  satisfying this predicate (i.e., the type is nonempty) since we are
  using classical logic and not \emph{free} logic.
\item[$-$] Attributes are modifiers to soft types. We need to prove
  there always exists a term satisfying this attribute before we can
  use it as an adjective on types.
\item[$-$] Structures describe tuples in ``Mathematical Reality'' and
  describes a new soft type. This is always well-defined.
\end{itemize}%
\end{remark}

\begin{remark}[Stipulated mathematics]
We will stipulate that the real numbers (and possibly the complex
numbers) have been introduced and shown to exist. We denote the set of
real numbers by $\RR$.
\end{remark}

\begin{remark}[Referring to material out of order]
We will isolate discussions which require material yet to be
presented, we follow Bourbaki's example and wrap it in ``earmuffs''
(as the Lisp programmers call it): \future{$\cdots$}.
\end{remark}