%%
%% topology.tex
%% 
%% Made by Alex Nelson <pqnelson@gmail.com>
%% Login   <alex@lisp>
%% 
%% Started on  2025-07-23T09:41:40-0700
%% Last update 2025-07-23T09:41:40-0700
%% 

\chapter{Point-Set Topology}

\begin{remark}[Motivation for topology]
Different mathematicians have different motivations for topology. For
our purposes, we need some notion of a ``space'', and topological
spaces are good enough. The definitions may seem \Latin{ad hoc}, but
they are ``reverse engineered'' from considering the situation in
$\RR^{n}$. 
\end{remark}

\begin{remark}[Topology requires classical mathematics]
In many proofs and constructions throughout topology, we will require
the axiom of choice (which forces our hand to use classical
logic). There are ways to formalize some of these notions in more
constructive settings. We are not interested in them.
\end{remark}

\section{Topological Spaces}

\begin{definition}[Topological space]\mml{pre_topc}
A \define{Topological Space} $T$ consists of a structure
$T=\structure{U,\tau}$ where
\begin{itemize}
\item the set $U$ is the \emph{carrier} of $T$, and
\item the \emph{topology} $\tau$ of $T$ is a family of subsets of $U$
\end{itemize}
where we call the elements of $\tau$ \define{Open} subsets of $T$ and
the elements of $T$ are called \define{Points} of $T$, such that
\begin{itemize}
\property{$U$ is open} $U\in\tau$
\property{$\emptyset$ is open} $\emptyset\in\tau$
\property{Topology closed under arbitrary unions}
  for any $A$ being an arbitrary family of subsets of $T$ such that
  $A\subset\tau$ (or, equivalently, every $a\in A$ implies $a\in\tau$),
  we have $\Union A\in\tau$
\property{Topology closed under finite intersections}
  for any subsets $A\subset T$ and $B\subset T$ such that $A\in\tau$
  and $B\in\tau$, we have $A\cap B\in\tau$.
\end{itemize}\vskip-\medskipamount
\end{definition}

\begin{example}[Nonempty topological space]
Consider the set $X=\{\emptyset\}$. Take $\tau=\{\emptyset,X\}$.
Then $T=\structure{U=X,\tau}$ forms a topological space.

\begin{proof}
We see that $\tau$ really is a family of subsets of $X$.
We see $X\in\tau$ and $\emptyset\in\tau$.

For $A$ being a family of subsets of $T$ such that $A\subset\tau$ we
have $\Union A\in\tau$. There are only four possibilities here since
the only family of subsets would be $A=\emptyset$ or $A=\{\emptyset\}$
or $A=\{X\}$ or $A=\{\emptyset,X\}$. In any of these four cases, we
see $\Union A\in\tau$.

Let $A\subset T$ and $B\subset T$. Assume $A\in\tau$ and $B\in\tau$. 
Then $A=\emptyset$ or $A=X$, and separately either $B=\emptyset$ or $B=X$.
Then we see $A\cap B=\emptyset$ or $A\cap B=X$ are the only possibilities. 
Hence $A\cap B\in\tau$.
\end{proof}
\end{example}

\begin{definition}[Closed subsets]\mml{pre_topc:def 3}
Let $T$ be a topological space, let $F\subset T$ be a subset of $T$.
We call $F$ \define{Closed} if $T\setminus F$ is an open subset of $T$.
\end{definition}

\begin{remark}[Variable $F$ used for closed subsets]
Bourbaki used the variable $F$ (and $F_{1}$, $F_{2}$, etc.) for closed
subsets of a topological space. This is because the French word for
``closed'' is \French{ferm\'{e}} which starts with the letter ``F''.
(Bourbaki uses $\mathop{\rm Fr}(A)$ for the boundary of a subset
$A\subset T$ since the french word for ``boundary'' is
\French{fronti\`{e}re}.) 
\end{remark}

\begin{example}[Trivial closed subsets]
Let $T = \structure{U,\tau}$ be a topological space.
Then both $\emptyset$ and $U$ is a closed subset of $T$.
In particular, this also means we can use ``closed'' as an adjective
for subsets of topological spaces.

(This is a registered fact in \texttt{pre\_topc}.)

\begin{proof}[Proof sketch]
We see $U$ is an open subset of $T$, therefore $T\setminus U=\emptyset$
is closed in $T$

Similarly, $\emptyset$ is an open subset of $T$, therefore
$T\setminus\emptyset=U$ is closed in $T$.
\end{proof}
\end{example}

\begin{theorem}[Union of closed subsets is closed]\mml{tops_1:9, tops_2:21}
Let $T$ be a topological space. Let $F_{1}$, $F_{2}$ be closed subsets
of $T$. Then $F_{1}\cup F_{2}$ is a closed subset of $T$.

More generally, if we have a finite family of closed subsets
$\mathcal{F}=\{F_{i}\mid i=1,\dots,n\}$ of $T$, then their union $\bigcup\mathcal{F}$
is a closed subset of $T$.
\end{theorem}
\begin{proof}
We can prove this by induction on $n$.

The base case is true for $n=0$ or $n=1$.

The inductive hypothesis: Assume $F=F_{1}\cup\cdots\cup F_{n}$ is closed.

The inductive case:
We want to prove $F_{1}\cup\cdots\cup F_{n}\cup F_{n+1}$ is closed.
But $F_{1}\cup\cdots\cup F_{n}\cup F_{n+1}=F\cup F_{n+1}$. We just
need to prove this is true. We see $T\setminus(F\cup F_{n+1})=(T\setminus F)\cap(T\setminus F_{n+1})$
by de Morgan, and the right-hand side is the intersection of two open
sets (which is open). This implies that $T\setminus(F\cup F_{n+1})$ is open,
hence $F\cup F_{n+1}$ is closed, as desired.
\end{proof}

\begin{theorem}[Intersection of closed subsets is closed]\mml{pre_topc:14, tops_2:22}\label{thm:topology:intersection-of-arbitrary-closed-sets-is-closed}%
Let $T$ be a topological space. Let
$\mathcal{F}=\{F_{\alpha}\}_{\alpha\in A}$ be an arbitrary family of
subsets of $T$ such that $F_{\alpha}$ is closed for every $\alpha\in A$.
Then $\bigcap\mathcal{F}$ is a closed subset of $T$.
\end{theorem}

\begin{proof}
Let $\mathcal{F}=\{F_{\alpha}\}_{\alpha\in A}$ be an arbitrary family of
subsets of $T$ such that $F_{\alpha}$ is closed for every $\alpha\in A$.
Then by Theorem~\ref{setfam_1:27},
\begin{equation}
T\setminus\Intersects\mathcal{F}=\Union\{T\setminus F_{\alpha}\mid\alpha\in A\},
\end{equation}
but $T\setminus F_{\alpha}$ is open for each $\alpha$. So that means
the right-hand side of the previous equation is the union of an
arbitrary family of open sets, which we know is open. This proves
$T\setminus\Intersects\mathcal{F}$ is open, which implies
$\Intersects\mathcal{F}$ is closed.
\end{proof}

\begin{definition}[Closure of a subset]\mml{pre_topc:16}
Let $T$ be a topological space. Let $E\subset T$ be any subset.
We can define the \define{Closure} of $E$ to be the subset
$\closure{E}$ of $T$
equal to $\closure{E}:=\bigcap\{F\subset T\mid E\subset F, F\mbox{ is closed}\}$.

This is an ``idempotent'' notion: $\closure{\closure{E}}=\closure{E}$.
\end{definition}

\begin{theorem}[Closure of a set contains the set]\mml{pre_topc:18}\label{pre-topc:18}
Let $T$ be a topological space. Let $E\subset T$ be any subset.
Then $E\subset\closure{E}$.
\end{theorem}

\begin{proof}
We will prove $\forall x\ldotp x\in E\implies x\in\closure{E}$.
Let $x$ be any arbitrary object. Assume $x\in E$.

For every closed subset $F\subset T$ such that $E\subset F$, we see
that $x\in F$ by definition of the subset predicate. Then we have
$x\in\Intersects\{F\subset T\mid F\mbox{ is closed}, E\subset F\}$ by
Definition~\ref{setfam-1:def-1} of the intersection of a family of sets. Hence
$x\in\closure{E}$ by Definition of the closure.
\end{proof}

\begin{theorem}[Closure of empty set is empty]\mml{pcomps_1:2}
Let $T$ be a topological space. Let $E$ be a subset of $T$.
Then $\closure{E}\neq\emptyset$ if and only if $E\neq\emptyset$.
\end{theorem}

\begin{proof}
\backwardproof\ Assume $E\neq\emptyset$. Then choose $x$ to be some
element of $E$ by the axiom of choice. Since $E\subset\closure{E}$ by Theorem~\ref{pre-topc:18},
we must have $x\in\closure{E}$. Hence $\closure{E}\neq\emptyset$.

\forwardproof\ We will prove the contrapositive (``if $E=\emptyset$,
then $\closure{E}=\emptyset$''). Assume $E=\emptyset$. Then
$E\subset\emptyset$. In particular, since $\emptyset$ is always a
closed set, this means that the closure is defined as the intersection
of a family of subsets containing the empty set. This intersection is
always empty by Theorem~\ref{setfam_1:4}. Hence $\closure{E}=\emptyset$.
\end{proof}

\begin{theorem}[Closure of a set is a closed set]\mml{pre_topc:22}\label{pre_topc:22}
Let $T$ be a topological space. Let $E\subset T$ be any subset.
Then $\closure{E}$ is a closed subset of $T$.
\end{theorem}

(This is also registered in \texttt{TOPS\_1}.)

\begin{proof}
Since $\closure{E}$ is the intersection of an arbitrary family of
closed sets containing $E$, the result follows from Theorem~\ref{thm:topology:intersection-of-arbitrary-closed-sets-is-closed}.
\end{proof}

\begin{theorem}[Closure is smallest closed set]\mml{pre_topc:15}\label{pre-topc:15}%
Let $T$ be a topological space. Let $E$ be a subset of $T$.
Let $x$ be an element of $T$.

Then $x\in\closure{E}$ if and only if for any closed subset $F$ of $T$
such that $E\subset F$, we have $x\in F$.
In particular, this means $\closure{E}\subset F$.
\end{theorem}

\begin{proof}
\backwardproof\ Assume for any subset $F$ of $T$ which is closed and
contains $E\subset F$, we have $x\in F$.
But then $x\in\Intersects\{F\subset T\mid F\mbox{ is closed}, E\subset F\}$
by Definition~\ref{setfam-1:def-1} of the intersection of family of sets.
Hence the result by definition of the closure of $E$.

\forwardproof\ Assume $x\in\closure{E}$.
Then $x\in\Intersects\{F\subset T\mid F\mbox{ is closed}, E\subset F\}$.
Then for each $F\subset T$ such that $F$ is closed and $E\subset F$,
we have $x\in F$ by Definition of the intersection of family of sets.
Hence the result.
\end{proof}

\begin{theorem}[Closure is subset order-preserving]\mml{pre_topc:19}
Let $T$ be a topological space. Let $A$ and $B$ be subsets of $T$.
If $A\subset B$, then $\closure{A}\subset\closure{B}$.
\end{theorem}

\begin{proof}
Assume $A\subset B$.
We will prove $\forall x\ldotp x\in\closure{A}\implies x\in\closure{B}$.
Let $x$ be an arbitrary object.
Assume $x\in\closure{A}$.
% Thesis: $x\in\closure{B}$.

We claim for any closed subset $D$ of $T$ such that $B\subset D$, we
have $x\in D$. We assumed $A\subset B$, which means that $A\subset D$.
Hence this means that $x\in D$ by Theorem~\ref{pre-topc:15}.

Then $x\in\closure{B}$, again by Theorem~\ref{pre-topc:15}.
\end{proof}

\begin{definition}[Interior of a set]\mml{tops_1:def 1}\label{defn:topology:interior-of-set}%
Let $T$ be a topological space. Let $E\subset T$ be a subset.
We define the \define{Interior} of $E$ to be the subset $\interior{E}$ of $T$ equal
to $\interior{E}:=T\setminus\closure{T\setminus E}$.

This is an ``idempotent'' notion: $\interior{\interior{E}}=\interior{E}$.
\end{definition}

\begin{theorem}[Interior of topological space is the space itself]\mml{tops_1:15}
Let $T$ be a topological space. Let $U_{T}$ be the underlying set of $T$.
Then $\interior{U_{T}}=U_{T}$.
\end{theorem}

\begin{proof}
We see that $\interior{U_{T}}=U_{T}\setminus\closure{U_{T}\setminus U_{T}}=U_{T}\setminus\closure{\emptyset}=U_{T}\setminus\emptyset=U_{T}$.
Hence the result.
\end{proof}

\begin{theorem}[Interior of set is always a subset]\mml{tops_1:16}
Let $T$ be a topological space. Let $E$ be a subset of $T$.
Then $\interior{E}\subset E$.
\end{theorem}

\begin{proof}
We will prove $\forall x\ldotp x\in\interior{E}\implies x\in E$.
Let $x$ be any object. Assume $x\in\interior{E}$. We know $\Complement{E}\subset\closure{\Complement{E}}$
by Theorem~\ref{pre-topc:18}. Since $x\in\interior{E}$, this means
$x\notin\Complement{\closure{E}}$ and in particular $x\notin\Complement{E}$
by Definition~\ref{xboole-0:def-5}.
Hence $x\in E$, again by Definition~\ref{xboole-0:def-5}.
\end{proof}

\begin{theorem}[Interior of a set is an open set]\mml{tops_1}
Let $T$ be a topological space. Let $E$ be a subset of $T$.
Then $\interior{E}$ is an open subset of $T$.
\end{theorem}

This is a registered fact in \texttt{TOPS\_1}.

\begin{proof}
We see that $\closure{T\setminus E}$ is always a closed set by Theorem~\ref{pre_topc:22}.
Then its complement is open.
Hence the result by Definition~\ref{defn:topology:interior-of-set}.
\end{proof}

\begin{definition}[Neighborhood of a point]\mml{connsp_2:def 1}
Let $T$ be a nonempty topological space. Let $x\in T$ be a point.
We define a [soft type] \define{Neighborhood} of $x$ to be a subset
$N_{x}$ of $T$ such that $x\in\interior{N_{x}}$.

This defines a mode. We can check it's well-defined because the
underlying set $U_{T}$ of $T$ is a neighborhood for every point of $T$.
\end{definition}

\begin{theorem}[Carrier of topological space is a neighborhood]\mml{topgrp_1:21}
Let $T$ be a nonempty topological space. Let $x$ be a point of $T$.
Then the carrier $U_{T}$ of $T$ is a neighborhood of $x$.
\end{theorem}

\begin{corollary}[Existence of an open non-empty neighborhood]\mml{topgrp_1}
For any non-empty topological space $T$, for any point $x\in T$ there
exists a neighborhood $N$ of $x$ such that $N$ is an open non-empty subset of $T$.
\end{corollary}

This is a registered fact in the Mizar article \texttt{TOPGRP\_1}.

%%% SUBSPACES

\begin{definition}[Subspaces]\mml{pre_topc:def 4}
Let $T = \structure{U_{T},\tau_{T}}$ be a topological space.
We define a \define{Subspace} of $T$ to be a topological space
$S=\structure{U_{S},\tau_{S}}$ such that $U_{S}\subset U_{T}$ and also
for $P\subset S$ we have $P\in\tau_{S}$ if and only if there is a
subset $Q\subset T$ such that $Q\in\tau_{T}$ and $P = Q\cap U_{S}$.

If $P\subset T$ is a nonempty subset, then we may introduce the term
$T|_{P}$ for the [strict] Subspace of $T$ whose underlying set is $P$.
\end{definition}


%%% CONTINUOUS
\section{Continuous maps}

\begin{definition}[Function is continuous at a point]\mml{tmap_1:43}
Let $S$, $T$ be nonempty topological spaces, let $f\colon S\to T$ be a function,
let $x\in S$ be a point. We say $f$ \define{is Continuous at} $x$ if
for each $V\subset T$ open such that $f(x)\in V$, there exists a
$U\subset S$ open such that $x\in U$ and $f(U)\subset V$.

This is a predicate, so it's already well-defined.
\end{definition}

\begin{definition}[Continuous functions]\mml{tops_2:43}
Let $S$, $T$ be topological spaces. Let $f\colon S\to T$ be a function.
We call $f$ \define{Continuous} if for every closed subset $P\subset T$
we have $\preimage{f}{P}$ is closed in $S$.

(Note: if $T$ is the empty space, then $S$ must be the empty space.)
\end{definition}

\begin{remark}[Intuition behind continuous functions]
We should think of continuous functions using metric spaces $\structure{U,d}$.
If $X$ and $Y$ are metric spaces, then a function $f\colon X\to Y$ is
continuous at $x_{0}\in X$ if for each $\varepsilon>0$ there is a $\delta>0$ such that 
$d_{X}(x,x_{0})<\delta$ implies $d_{Y}(y,f(x_{0}))<\varepsilon$.

This can be rephrased as: for every open ball $B_{Y}$ in $Y$
containing $f(x_{0})$, there exists an open ball $B_{X}$ in $X$ such
that $\preimage{f}{B_{Y}}\subset B_{X}$. We can replace ``open ball''
with ``open sets'' to generalize the notion to functions between
topological spaces.

The definition we have formalized is logically equivalent, though
perhaps pedagogically deficient.
\end{remark}

\begin{example}[Identity function is continuous]
Let $T$ be a topological space. The identity function $\id_{T}\colon T\to T$
is a continuous function. (This is a registered fact in
\texttt{tmap\_1} and \texttt{borsuk\_2}.)
\end{example}

\begin{example}[Inclusion map is continuous]\mml{tmap_1:87}\label{tmap_1:87}
Let $T$ be a topological space.
Let $S$ be a subspace of $T$.
Then $\incl\colon S\into T$ is a continuous function.
\end{example}

\begin{example}[Constant function is continuous]
Let $T$ and $T'$ be topological spaces, let $c\in T'$.
The constant function $f\colon T\to T'$ sending $x\in T$ to $f(x)=c$
is continuous. (This is a fact registered in \texttt{topalg\_6}.)
\end{example}

\begin{theorem}[Continuous functions defined using open subsets]\mml{tops_2:43}
Let $S$, $T$ be topological spaces. Let $f\colon S\to T$ be a function.
Then $f$ is continuous if and only if for each open subset $V\subset T$,
$\preimage{f}{V}$ is an open subset of $S$.
\end{theorem}

\begin{proof}
\forwardproof\ Assume $f$ is continuous. Let $V\subset T$ be an open subset.
Then $\Complement{V}$ is a closed subset. Then $\preimage{f}{\Complement{V}}$
is closed in $S$. We see that
$\preimage{f}{\Complement{V}}=(\preimage{f}{T})\setminus(\preimage{f}{V})$
but since $f(S)\subset T$, we see $\preimage{f}{T}=S$, so we have
$\preimage{f}{\Complement{V}}=S\setminus(\preimage{f}{V})=\Complement{\preimage{f}{V}}$.
Hence $\preimage{f}{V}$ is open.

\backwardproof\ Assume for every open subset $V\subset T$, we have
$\preimage{f}{V}$ is an open subset of $S$. We will prove every closed
subset $F\subset T$, $\preimage{f}{F}$ is open in $S$. Let $F\subset T$
be closed. Then $\Complement{F}$ is open. Then $\preimage{f}{\Complement{V}}=(\preimage{f}{Y})\setminus(\preimage{f}{V})$
and by similar reasoning as the forward direction proof, $\preimage{f}{\Complement{V}}=\Complement{\preimage{f}{V}}$,
and so $\preimage{f}{V}$ is closed. Hence the result.
\end{proof}

\begin{theorem}[Continuous function is continuous at every point]\mml{tmap_1:44}
Let $T$ and $T'$ be topological spaces, let $f\colon T\to T'$ be a function.
Then $f$ is continuous if and only if for each $x\in T$, $f$ is
continuous at $x$.
\end{theorem}

\begin{theorem}[Composing continuous functions yields a continuous function]\mml{tops_2:46}\label{tops_2:46}
Let $T_{1}$, $T_{2}$, $T_{3}$ be topological spaces.
Let $f\colon T_{1}\to T_{2}$ and $g\colon T_{2}\to T_{3}$ be
continuous functions.
Then $g\circ f\colon T_{1}\to T_{3}$ is a continuous function.
\end{theorem}

\begin{proof}
Let $U\subset T_{3}$ be an open set.
Then $\preimage{g}{U}\subset T_{2}$ is open by continuity of $g$,
and $\preimage{f}{\preimage{g}{U}}\subset T_{1}$ is open by continuity
of $f$.
Then $\preimage{(g\circ f)}{U}\subset T_{1}$ is open.
Hence $g\circ f$ is continuous.
\end{proof}

\begin{theorem}[Continuous function restricted to subspace is continuous]\mml{pre_topc:27}
Let $T$ and $T'$ be topological spaces, let $S\subset T$ be a subspace
of $T$. If $f\colon T\to T'$, then its restriction to $S$,
$$f|_{S}\colon S\to T',$$
is a continuous function.
\end{theorem}

\begin{proof}
We see that $f|_{S}=f\circ\incl\colon S\to T'$, and since both $\incl$
is continuous (by Example~\ref{tmap_1:87}) and $f$ is continuous, then
$f\circ\incl$ is continuous by Theorem~\ref{tops_2:46}. Hence the result.
\end{proof}

\begin{remark}[Category of topological spaces]
We can form the large category $\Top$ of topological spaces, specifically:
\begin{itemize}
\property{Objects} The objects of $\Top$ consists of topological spaces
\property{Morphisms} The morphisms of $\Top$ are the continuous
  functions between topological spaces in $\Top$.
\end{itemize}%
We can form a small category if we are given a set $U$ of topological
spaces, which we denote by $\Top_{U}$.
\end{remark}

\begin{definition}[Homeomorphism and Homeomorphic spaces]\mml{t_0topsp:def~1, tops_2:def~5, topgrp_1:def~3}
Let $X$ and $Y$ be topological spaces. A \define{Homeomorphism} from
$X$ to $Y$ is a continuous function $f\colon X\to Y$ such that there
exists an inverse function $f^{-1}\colon Y\to X$ which is also a
continuous function. That is to say, it is a bijection $f$ of the
underlying sets such that $f$ is continuous and its inverse $f^{-1}$
is continuous.

Caution: a continuous bijection \emph{is not necessarily a homeomorphism!}
For example, consider the function $f\colon[0,1)\to S^{1}$ defined by
$f(x)=\exp(2\pi\I x)$. This is a bijection and it is continuous, but
it is not a homeomorphism.
\end{definition}

\begin{remark}[Topological invariants preserved by homeomorphisms]
We care about ``topological invariants'' in topology. It is hard to
actually determine if two spaces are homeomorphic or not, but it is
easier to find some invariant which one space has but the other lacks.

We could define a ``topological invariant'' to be an equivalence class
of topological spaces (which share the same invariant or property).
\end{remark}

\begin{theorem}[Identity map is a homeomorphism]\mml{topgrp_1:20}\label{topgrp_1:20}
Let $T$ be a topological space. Then the identity mapping $\id_{T}$ is
a homeomorphism.
\end{theorem}

\begin{theorem}[Inverse of a homeomorphism is again a homeomorphism]\mml{tops_2:56}\label{tops_2:56}
Let $S$, $T$ be topological spaces. Let $T$ be nonempty.
If $f\colon S\to T$ is a homeomorphism, then $f^{-1}\colon T\to S$
is a homeomorphism.
\end{theorem}

\begin{theorem}[Composing homeomorphisms yields a homeomorphism]\mml{tops_2:57}\label{tops_2:57}
Let $T_{1}$, $T_{2}$, $T_{3}$ be nonempty topological spaces.
If $f\colon T_{1}\to T_{2}$ and $g\colon T_{2}\to T_{3}$ are homeomorphisms,
then $g\circ f\colon T_{1}\to T_{3}$ is a homeomorphism.
\end{theorem}

\begin{proof}
We want to prove $g\circ f$ is continuous, $(g\circ f)^{-1}$ is continuous,
and $g\circ f$ is a bijection.

We know $f$ and $g$ are continuous. This establishes $g\circ f$ is continuous.

We know $f$ and $g$ are bijections. Hence $g\circ f$ is a bijection.

We know $(g\circ f)^{-1}=f^{-1}\circ g^{-1}$, and we know that
$f^{-1}$ is continuous and $g^{-1}$ is continuous.
Hence $(g\circ f)^{-1}$ is continuous.
\end{proof}

\begin{definition}[Predicate ``are homeomorphic'' among topological spaces]
When there exists a homeomorphism between $X$ and $Y$, we say that $X$
and $Y$ are \define{Homeomorphic}. Importantly, we will later prove
this relation (``$X$ and $Y$ are homeomorphic'') forms an equivalence relation.
\end{definition}

\begin{theorem}[Homeomorphic is a reflexive relation]\mml{borsuk_3:lm1}
Let $T$ be a topological space. Then $T$ is homeomorphic to $T$.
\end{theorem}

\begin{proof}
This follows immediately from Theorem~\ref{topgrp_1:20}.
\end{proof}

\begin{theorem}[Homeomorphic is a symmetric relation among spaces]\mml{borsuk_3:lm2}
Let $S$, $T$ be topological spaces. If $S$ is homeomorphic to $T$,
then $T$ is homeomorphic to $S$.
\end{theorem}

\begin{proof}
Let $f\colon S\to T$ be a homeomorphism. Then $f^{-1}\colon T\to S$ is
a homeomorphism by Theorem~\ref{tops_2:56}. Hence the result.
\end{proof}

\begin{theorem}[Homeomorphic is a transitive relation among spaces]\mml{borsuk_3:3}
Let $T_{1}$, $T_{2}$, $T_{3}$ be topological spaces.
If $T_{1}$ is homeomorphic to $T_{2}$, and $T_{2}$ is homeomorphic to $T_{3}$,
then $T_{1}$ is homeomorphic to $T_{3}$.
\end{theorem}

\begin{proof}
Let $f\colon T_{1}\to T_{2}$ and $g\colon T_{2}\to T_{3}$ be homeomorphisms.
Then $g\circ f\colon T_{1}\to T_{3}$ is a homeomorphism by
Theorem~\ref{tops_2:57}. Hence the result.
\end{proof}

\begin{remark}[``Are homeomorphic'' is an equivalence relation of topological spaces]
We have established in these previous three theorems that ``are
homeomorphic'' is a symmetric, reflexive, transitive predicate of two
topological spaces.
\end{remark}

\begin{definition}[Embedding of a space]\mml{compact1:def 7}
Let $X$ and $Y$ be topological spaces.
Let $f\colon X\to Y$ be a function.
We say that $f$ is an \define{Embedding} if $f$ is a homeomorphism
between $X$ and $f(X)$.
\end{definition}

\begin{theorem}[Inclusion mapping is an embedding]\mml{compact1:3}
Let $X$ be a subspace of $Y$.
Then the inclusion mapping (\S\ref{yellow_9:def 1}) $\incl\colon X\into Y$ is an embedding.
\end{theorem}

\section{Connected Spaces}

\begin{definition}[Separated subsets of a topological space]\mml{connsp_1:def 1}
Let $T$ be a topological space. Let $A$, $B$ be subsets of $T$.
We say $A$ and $B$ \define{are separated} if $\closure{A}\cap B=\emptyset$
and $A\cap\closure{B}=\emptyset$.
\end{definition}

\begin{theorem}[Separated open subsets]\mml{connsp_1:3}
Let $T$ be a topological space.
Let $A$, $B$ be open subsets of $T$.
If they cover $T=A\cup B$ and are disjoint $A\cap B=\emptyset$,
then $A$ and $B$ are separated.
\end{theorem}

\begin{definition}[Connected space]\mml{connsp_1:def 2}
Let $T$ be a topological space.
We say $T$ is \define{Connected} if for any subsets $A$ and $B$ of $T$
which cover $T=A\cup B$ and also with $A$ and $B$ are separated, then
either $A=\emptyset$ or $B=\emptyset$.

If $T$ is not connected, then we say $T$ is \define{Disconnected} and
we call $A$ and $B$ a \define{Separation Pair} of $T$. We can always
describe $T$ as the union of connected disjoint subspaces called the
\define{Connected Components} of $T$. (Connected spaces have one
connected component.)
\end{definition}

\begin{remark}[Separation]
We will often want to consider different ``degrees'' of
``separation-ness'' for a topological space. For example, can any two
disjoint points be contained by two disjoint open subsets?

Or if we have a point $x$ and an open subset $U$ such that $x\notin U$,
can we find open subsets $O_{1}$ and $O_{2}$ such that $x\in O_{1}$
and $U\propersubset O_{2}$ and $O_{1}\cap O_{2}=\emptyset$?
\end{remark}

%%% Closures of a set

\begin{definition}[Closure of a subset]\mml{pre_topc:def 7}
Let $T$ be a topological space, let $A\subset T$ be a subset of $T$.
We define the term the \define{Closure of $A$} to be the subset
$\closure{A}$ of $T$ such that
for $p\in T$ we have $p\in\closure{A}$ if and only if for each open
subset $U\subset T$ containing $p\in U$ we have $A\cap U\neq\emptyset$.
\end{definition}

\begin{theorem}[Properties of the closure of a subset]
Let $T$ be a topological space, let $A$ and $B$ be subsets of $T$.
\begin{enumerate}
\item\textsc{Projectivity:} We have $\closure{\closure{A}}=\closure{A}$.
\item We have $\closure{A}$ be a closed subset of $T$.
\item\mml{pre_topc:18} We have $A\subset\closure{A}$.
\item\mml{pre_topc:19} If $A\subset B$, then $\closure{A}\subset\closure{B}$.
\item\mml{pre_topc:20} We have $\closure{A\cup B}=\bigl(\closure{A}\bigr)\cup\bigl(\closure{B}\bigr)$.
\item\mml{pre_topc:21} We have $\closure{A\cap B}\subset\bigl(\closure{A}\bigr)\cap\bigl(\closure{B}\bigr)$.
\item\mml{tops_1:5} If $B$ is closed and $A\subset B$, then $\closure{A}\subset B$.
\end{enumerate}
\end{theorem}

\begin{definition}[$T_{0}$, $T_{1}$, $T_{2}$, regular, normal topological spaces]\mml{pre_topc:def 8-14}
Let $T$ be a topological space.
\begin{enumerate}
\item We say $T$ is ``$T_{0}$'' (or \define{Kolmogorov})
  if for all points $x\in T$ and $y\in T$
  such that, whenever an open subset $U\subset T$ satisfies $x\in U$
  iff $y\in U$, implies $x = y$.
\item%\mml{pre_topc:def 9}
  We say $T$ is ``$T_{1}$'' (or \define{Frechet})
  if for all distinct points $x\in T$ and $y\in T$ (s.t., $x\neq y$)
  there exists an open subset $U\subset T$ such that $x\in U$ and
  $y\in T\setminus U$.
\item%\mml{pre_topc:def 10}
  We say $T$ is \define{Hausdorff} (or ``$T_{2}$'')
  if for each pair of distinct points $x\in T$ and $y\in T$ 
  there exists disjoint open subsets $U_{1}\subset T$ and
  $U_{2}\subset T$ such that $x\in U_{1}$ and $y\in U_{1}$ (and
  $U_{1}\cap U_{2}=\emptyset$).
\item%\mml{pre_topc:def 11}
  We say $T$ is \define{Regular} if
  for each point $p\in T$ and closed subset $F\subset T$ such that
  $p\in T\setminus F$ there exists disjoint open subsets $U_{1}\subset T$ and
  $U_{2}\subset T$ such that $p\in U_{1}$ and $F\subset U_{2}$ (and
  $U_{1}\cap U_{2}=\emptyset$).
\item%\mml{pre_topc:def 12}
  We say $T$ is \define{Normal} if
  for each pair of disjoint closed subsets $F_{1}\subset T$ and
  $F_{2}\subset T$ there exists a disjoint pair of open subsets
  $U_{1}\subset T$ and $U_{2}\subset T$ such that $F_{1}\subset U_{1}$
  and $F_{2}\subset U_{2}$ (and $U_{1}\cap U_{2}=\emptyset$).
\item%\mml{pre_topc:def 13}
  We say $T$ is \define{$T_{3}$} if $T$ is $T_{1}$ and regular
\item%\mml{pre_topc:def 14}
  We say $T$ is \define{$T_{4}$} if $T$ is $T_{1}$ and normal
\end{enumerate}
\end{definition}

\begin{remark}[Terminology]
These properties $T_{n}$ for $n>2$ are always of the form ``$T_{1}$ and\dots''.

We seldom see $T_{3}$ in the literature (more often we find ``regular $T_{1}$''),
nor do we see $T_{4}$ much either (we often see ``normal $T_{1}$'').
\end{remark}

\begin{example}[Empty topological space is $T_{0}$]
The empty topological space is $T_{0}$. (This is a fact registered in \texttt{pre\_topc}.)

\begin{proof}[Proof sketch]
Let $T$ be a topological space. Assume $T$ is empty.
Let $x$, $y$ be points of $T$. Assume for $U$ being an open subset of
$T$ we have $x\in U$ iff $y\in U$.
But $x\in U$ implies a contradiction (and similarly $y\in U$ implies a
contradiction), which implies the result by the explosion principle.
\end{proof}
\end{example}

\begin{theorem}[Relation among these properties]\mml{pre_topc}
Let $T$ be a topological space. We see the following holds:
\begin{enumerate}
\item If $T$ is $T_{1}$, then $T$ is $T_{0}$.
\item If $T$ is $T_{2}$, then $T$ is $T_{1}$ (and $T$ is $T_{0}$)
\end{enumerate}
\end{theorem}

\begin{theorem}[Diagonal map of $T$ is closed iff $T$ is Hausdorff]
Let $T$ be a topological space.
Then $T$ is Hausdorff if and only if the image of the diagonal map (\S\ref{defn:set-theory:diagonal-function}) $\Delta\colon T\to T\times T$
is closed in $T\times T$.
\end{theorem}

\begin{remark}[Generalization of Hausdorff property]
This previous theorem (the image of $\Delta$ is closed in $T\times T$)
is used or abused by algebraic geometers to generalize the notion of
``Hausdorff'' for algebraic varieties. It is a fact that prevarieties
are Hausdorff iff they are singletons (see, e.g.,
Osserman~\cite[ch.5]{osserman2021concise}). 
\end{remark}