%%
%% topology.tex
%% 
%% Made by Alex Nelson <pqnelson@gmail.com>
%% Login   <alex@lisp>
%% 
%% Started on  2025-07-23T09:41:40-0700
%% Last update 2025-07-23T09:41:40-0700
%% 

\chapter{Point-Set Topology}

\begin{remark}[Motivation for topology]
Different mathematicians have different motivations for topology. For
our purposes, we need some notion of a ``space'', and topological
spaces are good enough. The definitions may seem \Latin{ad hoc}, but
they are ``reverse engineered'' from considering the situation in
$\RR^{n}$. 
\end{remark}

\begin{remark}[Topology requires classical mathematics]
In many proofs and constructions throughout topology, we will require
the axiom of choice (which forces our hand to use classical
logic). There are ways to formalize some of these notions in more
constructive settings. We are not interested in them.
\end{remark}

\section{Topological Spaces}

\begin{definition}[Topological space]\mml{pre_topc}
A \define{Topological Space} $T$ consists of a structure
$T=\structure{U,\tau}$ where
\begin{itemize}
\item the set $U$ is the \emph{carrier} of $T$, and
\item the \emph{topology} $\tau$ of $T$ is a family of subsets of $U$
\end{itemize}
where we call the elements of $\tau$ \define{Open} subsets of $T$ and
the elements of $T$ are called \define{Points} of $T$, such that
\begin{itemize}
\property{$U$ is open} $U\in\tau$
\property{$\emptyset$ is open} $\emptyset\in\tau$
\property{Topology closed under arbitrary unions}
  for any $A$ being an arbitrary family of subsets of $T$ such that
  $A\subset\tau$ (or, equivalently, every $a\in A$ implies $a\in\tau$),
  we have $\Union A\in\tau$
\property{Topology closed under finite intersections}
  for any subsets $A\subset T$ and $B\subset T$ such that $A\in\tau$
  and $B\in\tau$, we have $A\cap B\in\tau$.
\end{itemize}\vskip-\medskipamount
\end{definition}

\begin{example}[Nonempty topological space]
Consider the set $X=\{\emptyset\}$. Take $\tau=\{\emptyset,X\}$.
Then $T=\structure{U=X,\tau}$ forms a topological space.

\begin{proof}
We see that $\tau$ really is a family of subsets of $X$.
We see $X\in\tau$ and $\emptyset\in\tau$.

For $A$ being a family of subsets of $T$ such that $A\subset\tau$ we
have $\Union A\in\tau$. There are only four possibilities here since
the only family of subsets would be $A=\emptyset$ or $A=\{\emptyset\}$
or $A=\{X\}$ or $A=\{\emptyset,X\}$. In any of these four cases, we
see $\Union A\in\tau$.

Let $A\subset T$ and $B\subset T$. Assume $A\in\tau$ and $B\in\tau$. 
Then $A=\emptyset$ or $A=X$, and separately either $B=\emptyset$ or $B=X$.
Then we see $A\cap B=\emptyset$ or $A\cap B=X$ are the only possibilities. 
Hence $A\cap B\in\tau$.
\end{proof}
\end{example}

\begin{definition}[Closed subsets]
Let $T$ be a topological space, let $P\subset T$ be a subset of $T$.
We call $P$ \define{Closed} if $T\setminus P$ is an open subset of $T$.
\end{definition}

\begin{remark}[Variable $F$ used for closed subsets]
Bourbaki used the variable $F$ (and $F_{1}$, $F_{2}$, etc.) for closed
subsets of a topological space. This is because the French word for
``closed'' is \French{ferm\'{e}} which starts with the letter ``F''.
(Bourbaki uses $\mathop{\rm Fr}(A)$ for the boundary of a subset
$A\subset T$ since the french word for ``boundary'' is
\French{fronti\`{e}re}.) 
\end{remark}

\begin{example}[Trivial closed subsets]
Let $T = \structure{U,\tau}$ be a topological space.
Then both $\emptyset$ and $U$ is a closed subset of $T$.
In particular, this also means we can use ``closed'' as an adjective
for subsets of topological spaces.

\begin{proof}[Proof sketch]
We see $U$ is an open subset of $T$, therefore $T\setminus U=\emptyset$
is closed in $T$

Similarly, $\emptyset$ is an open subset of $T$, therefore
$T\setminus\emptyset=U$ is closed in $T$.
\end{proof}
\end{example}

\begin{theorem}[Union of closed subsets is closed]\mml{tops_1}
Let $T$ be a topological space. Let $F_{1}$, $F_{2}$ be closed subsets
of $T$. Then $F_{1}\cup F_{2}$ is a closed subset of $T$.
\end{theorem}

\begin{theorem}[Intersection of closed subsets is closed]\mml{tops_1}
Let $T$ be a topological space. Let $F_{1}$, $F_{2}$ be closed subsets
of $T$. Then $F_{1}\cap F_{2}$ is a closed subset of $T$.
\end{theorem}

%%% SUBSPACES

\begin{definition}[Subspaces]
Let $T = \structure{U_{T},\tau_{T}}$ be a topological space.
We define a \define{Subspace} of $T$ to be a topological space
$S=\structure{U_{S},\tau_{S}}$ such that $U_{S}\subset U_{T}$ and also
for $P\subset S$ we have $P\in\tau_{S}$ if and only if there is a
subset $Q\subset T$ such that $Q\in\tau_{T}$ and $P = Q\cap U_{S}$.

If $P\subset T$ is a nonempty subset, then we may introduce the term
$T|_{P}$ for the [strict] Subspace of $T$ whose underlying set is $P$.
\end{definition}


%%% CONTINUOUS

\begin{definition}[Continuous functions]
Let $S$, $T$ be topological spaces. Let $f\colon S\to T$ be a function.
We call $f$ \define{Continuous} if for every closed subset $P\subset T$
we have $\preimage{f}{P}$ is closed in $S$.
\end{definition}

\begin{remark}[Intuition behind continuous functions]
We should think of continuous functions using metric spaces $\structure{U,d}$.
If $X$ and $Y$ are metric spaces, then a function $f\colon X\to Y$ is
continuous at $x_{0}\in X$ if for each $\varepsilon>0$ there is a $\delta>0$ such that 
$d_{X}(x,x_{0})<\delta$ implies $d_{Y}(y,f(x_{0}))<\varepsilon$.

This can be rephrased as: for every open ball $B_{Y}$ in $Y$
containing $f(x_{0})$, there exists an open ball $B_{X}$ in $X$ such
that $\preimage{f}{B_{Y}}\subset B_{X}$. We can replace ``open ball''
with ``open sets'' to generalize the notion to functions between
topological spaces.

The definition we have formalized is logically equivalent, though
perhaps pedagogically deficient.
\end{remark}

\begin{example}[Identity function is continuous]
Let $T$ be a topological space. The identity function $\id_{T}\colon T\to T$
is a continuous function.
\end{example}

\begin{example}[Constant function is continuous]
Let $T$ and $T'$ be topological spaces, let $c\in T'$.
The constant function $f\colon T\to T'$ sending $x\in T$ to $f(x)=c$
is continuous.
\end{example}

\begin{theorem}[Continuous function restricted to subspace is continuous]\mml{pre_topc:27}
Let $T$ and $T'$ be topological spaces, let $S\subset T$ be a subspace
of $T$. If $f\colon T\to T'$, then its restriction to $S$,
$$f|_{S}\colon S\to T',$$
is a continuous function.
\end{theorem}

\begin{theorem}[Composing continuous functions yields a continuous function]
Let $T_{1}$, $T_{2}$, $T_{3}$ be topological spaces.
Let $f\colon T_{1}\to T_{2}$ and $g\colon T_{2}\to T_{3}$ be
continuous functions.
Then $g\circ f\colon T_{1}\to T_{3}$ is a continuous function.
\end{theorem}

\begin{remark}[Category of topological spaces]
We can form the large category $\Top$ of topological spaces, specifically:
\begin{itemize}
\property{Objects} The objects of $\Top$ consists of topological spaces
\property{Morphisms} The morphisms of $\Top$ are the continuous
  functions between topological spaces in $\Top$.
\end{itemize}%
We can form a small category if we are given a set $U$ of topological
spaces, which we denote by $\Top_{U}$.
\end{remark}

\begin{definition}[Homeomorphism and Homeomorphic spaces]
Let $X$ and $Y$ be topological spaces. A \define{Homeomorphism} from
$X$ to $Y$ is a continuous function $f\colon X\to Y$ such that there
exists an inverse function $f^{-1}\colon Y\to X$ which is also a
continuous function. That is to say, it is a bijection $f$ of the
underlying sets such that $f$ is continuous and its inverse $f^{-1}$
is continuous.

When there exists a homeomorphism between $X$ and $Y$, we say that $X$
and $Y$ are \define{Homeomorphic}. Importantly, this relation (``$X$
and $Y$ are homeomorphic'') forms an equivalence relation.

Caution: a continuous bijection \emph{is not necessarily a homeomorphism!}
For example, consider the function $f\colon[0,1)\to S^{1}$ defined by
$f(x)=\exp(2\pi\I x)$. This is a bijection and it is continuous, but
it is not a homeomorphism.
\end{definition}

\begin{remark}[Topological invariants preserved by homeomorphisms]
We care about ``topological invariants'' in topology. It is hard to
actually determine if two spaces are homeomorphic or not, but it is
easier to find some invariant which one space has but the other lacks.

We could define a ``topological invariant'' to be an equivalence class
of topological spaces (which share the same invariant or property).
\end{remark}

%%% Closures of a set

\begin{definition}[Closure of a subset]
Let $T$ be a topological space, let $A\subset T$ be a subset of $T$.
We define the term the \define{Closure of $A$} to be the subset
$\closure{A}$ of $T$ such that
for $p\in T$ we have $p\in\closure{A}$ if and only if for each open
subset $U\subset T$ containing $p\in U$ we have $A\cap U\neq\emptyset$.
\end{definition}

\begin{theorem}[Properties of the closure of a subset]
Let $T$ be a topological space, let $A$ and $B$ be subsets of $T$.
\begin{enumerate}
\item\textsc{Projectivity:} We have $\closure{\closure{A}}=\closure{A}$.
\item We have $\closure{A}$ be a closed subset of $T$.
\item We have $A\subset\closure{A}$.
\item If $A\subset B$, then $\closure{A}\subset\closure{B}$.
\item We have $\closure{A\cup B}=\bigl(\closure{A}\bigr)\cup\bigl(\closure{B}\bigr)$.
\item We have $\closure{A\cap B}=\bigl(\closure{A}\bigr)\cap\bigl(\closure{B}\bigr)$.
\item\mml{tops_1:5} If $B$ is closed and $A\subset B$, then $\closure{A}\subset B$.
\end{enumerate}
\end{theorem}


\begin{definition}[$T_{0}$, $T_{1}$, $T_{2}$, regular, normal topological spaces]\mml{pre_topc:def 8-14}
Let $T$ be a topological space.
\begin{enumerate}
\item We say $T$ is ``$T_{0}$'' (or \define{Kolmogorov})
  if for all points $x\in T$ and $y\in T$
  such that, whenever an open subset $U\subset T$ satisfies $x\in U$
  iff $y\in U$, implies $x = y$.
\item%\mml{pre_topc:def 9}
  We say $T$ is ``$T_{1}$'' (or \define{Frechet})
  if for all distinct points $x\in T$ and $y\in T$ (s.t., $x\neq y$)
  there exists an open subset $U\subset T$ such that $x\in U$ and
  $y\in T\setminus U$.
\item%\mml{pre_topc:def 10}
  We say $T$ is \define{Hausdorff} (or ``$T_{2}$'')
  if for each pair of distinct points $x\in T$ and $y\in T$ 
  there exists disjoint open subsets $U_{1}\subset T$ and
  $U_{2}\subset T$ such that $x\in U_{1}$ and $y\in U_{1}$ (and
  $U_{1}\cap U_{2}=\emptyset$).
\item%\mml{pre_topc:def 11}
  We say $T$ is \define{Regular} if
  for each point $p\in T$ and closed subset $F\subset T$ such that
  $p\in T\setminus F$ there exists disjoint open subsets $U_{1}\subset T$ and
  $U_{2}\subset T$ such that $p\in U_{1}$ and $F\subset U_{2}$ (and
  $U_{1}\cap U_{2}=\emptyset$).
\item%\mml{pre_topc:def 12}
  We say $T$ is \define{Normal} if
  for each pair of disjoint closed subsets $F_{1}\subset T$ and
  $F_{2}\subset T$ there exists a disjoint pair of open subsets
  $U_{1}\subset T$ and $U_{2}\subset T$ such that $F_{1}\subset U_{1}$
  and $F_{2}\subset U_{2}$ (and $U_{1}\cap U_{2}=\emptyset$).
\item%\mml{pre_topc:def 13}
  We say $T$ is \define{$T_{3}$} if $T$ is $T_{1}$ and regular
\item%\mml{pre_topc:def 14}
  We say $T$ is \define{$T_{4}$} if $T$ is $T_{1}$ and normal
\end{enumerate}
\end{definition}

\begin{remark}[Terminology]
These properties $T_{n}$ for $n>2$ are always of the form ``$T_{1}$ and\dots''.

We seldom see $T_{3}$ in the literature (more often we find ``regular $T_{1}$''),
nor do we see $T_{4}$ much either (we often see ``normal $T_{1}$'').
\end{remark}

\begin{example}[Empty topological space is $T_{0}$]
The empty topological space is $T_{0}$.

\begin{proof}[Proof sketch]
Let $T$ be a topological space. Assume $T$ is empty.
Let $x$, $y$ be points of $T$. Assume for $U$ being an open subset of
$T$ we have $x\in U$ iff $y\in U$.
But $x\in U$ implies a contradiction (and similarly $y\in U$ implies a
contradiction), which implies the result by the explosion principle.
\end{proof}
\end{example}

\begin{theorem}[Relation among these properties]
Let $T$ be a topological space. We see the following holds:
\begin{enumerate}
\item If $T$ is $T_{1}$, then $T$ is $T_{0}$.
\item If $T$ is $T_{2}$, then $T$ is $T_{1}$ (and $T$ is $T_{0}$)
\end{enumerate}
\end{theorem}

\begin{theorem}[Diagonal map of $T$ is closed iff $T$ is Hausdorff]
Let $T$ be a topological space.
Then $T$ is Hausdorff if and only if the image of the diagonal map (\S\ref{defn:set-theory:diagonal-function}) $\Delta\colon T\to T\times T$
is closed in $T\times T$.
\end{theorem}

\begin{remark}[Generalization of Hausdorff property]
This previous theorem (the image of $\Delta$ is closed in $T\times T$)
is used or abused by algebraic geometers to generalize the notion of
``Hausdorff'' for algebraic varieties. It is a fact that prevarieties
are Hausdorff iff they are singletons (see, e.g.,
Osserman~\cite[ch.5]{osserman2021concise}). 
\end{remark}