%%
%% structures.tex
%% 
%% Made by Alex Nelson <pqnelson@gmail.com>
%% Login   <alex@lisp>
%% 
%% Started on  2025-07-29T10:17:12-0700
%% Last update 2025-07-29T10:17:12-0700
%% 

\chapter{Structures}

\begin{remark}[Metamathematical concerns about ``structures'']
We often run into definitions like ``A \emph{monoid} is a triple
$(M,\star,e)$ where $M$ is a set, $\star\colon M\times M\to M$ is a
binary operator on $M$, and $e\in M$ is an identity element, which
satisfies associativity for the operator
$x\star(y\star z)=(x\star y)\star z$ for all $x,y,z\in M$
and $e$ is an identity element, i.e., $e\star x=x\star e=x$ for all
$x\in M$''. In this example, where does the triple $(M,\star,e)$ live?


\parshape 7 %13
%% 0pt \hsize 0pt \hsize 0pt \hsize 0pt \hsize
%% 0pt \hsize 0pt \hsize 0pt \hsize 0pt \hsize
%% .3\hsize .7\hsize .3\hsize .7\hsize .3\hsize .7\hsize .3\hsize .7\hsize
%% .3\hsize .7\hsize .3\hsize .7\hsize .3\hsize .7\hsize .3\hsize .4\hsize
%% .3\hsize .4\hsize .3\hsize .4\hsize 0pt .7\hsize 0pt .7\hsize
0pt .7\hsize 0pt .7\hsize 0pt .7\hsize 0pt .7\hsize
0pt .7\hsize 0pt .7\hsize 0pt .7\hsize
%% .7\hsize 0pt .7\hsize 
%% 0pt .7\hsize 0pt .7\hsize 0pt .7\hsize 0pt .7\hsize 
%% 0pt \hsize
\noindent
\hbox to 0pt{\hskip.725\hsize
\vbox to 0pt{%
\vskip0\baselineskip%\vskip14\baselineskip
\hskip-2pc\includegraphics{img/img.0}

\noindent{\footnotesize {\bf Fig.~1.} Mathematical Platonism \hfill\break as a sky scraper.}
\vss}\hss}%
\indent Metamathematically, we have a tower of logics and
set theories (see Figure 1). On floor $2N$ (usually $2N=2$), we have a set theory which
we assert \emph{is} ``mathematical reality''. All ``real''
mathematical objects live in the set theory on floor $2N$. We then use
this to build a first-order logic on floor $2N+1$ which is the
\emph{ambient logic} used to write down proofs in Working
Mathematics. Not only that, we write down an axiomatic set theory
using the ambient logic, and this produces floor $2N+2$. This is the
\emph{ambient set theory}.

\parshape 5
0pt .7\hsize 0pt .7\hsize 0pt .7\hsize 0pt \hsize
0pt \hsize 
\indent
The triple $(M,\star,e)$ lives on floor $2N$, but is not distinguished
from triples from floor $2N+2$. At least, not in Working Mathematics,
not historically. Bourbaki's disinterest in the foundations of
mathematics should be blamed for this. \textbf{We will break from tradition}
and use angled brackets writing $\structure{M,\star,e}$ to indicate
that we have a structure and not some tuple.

In Figure~1, the Working Mathematician develops mathematical theories
in Logic \#3. Logic \#3 is purely syntactic for syntactic proofs, and
Set Theory \#4 is purely syntactic constructed as axioms within Logic \#3.
When we develop a theory of monoids, we state it using these syntactic
levels. But when we consider \emph{models} (i.e., \emph{examples}) of
monoids, we look for them in Set Theory \#2 (``mathematical reality'').

However, the \emph{subject} of model theory requires building two more
floors to the skyscraper, so we use logic \#3 to state and prove
results concerning models of a theory stated in set theory \#6 using
the sets from set theory \#4 as the models. However, model theory
textbooks neglect this important point, so set theories \#4 and \#6
are neither clearly delineated nor distinguished.

(This is seldom discussed in the literature; see, e.g.,
Hurkyl~\cite{hurkyl2012avoid}.) 
\end{remark}

\begin{remark}[Finitary metatheory is informal]
The ``finitary metatheory'' foundations in Figure~1 underlying the
``Platonic skyscraper'' is \emph{always} informal. This is because
formal Logic (and formal Mathematics) is a level in the skyscraper.
Roughly this ``finitary metatheory'' corresponds to the level of
Mathematics taught to ten-year olds.

Hilbert's programme first proposed using a finitary metatheory to
build a formal logic (with which we formalize Mathematics), then use
the finitary methods to prove consistency of Mathematics. Of course,
G\"{o}del killed Hilbert's programme. But the general framework proved
useful for discussing metamathematics (coincidentally, Hilbert coined
the term ``Metamathematics'' while working on his research programme).
\end{remark}

\begin{remark}[Notation about structures]
We will abuse notation and write for structures $S=\structure{U,\dots}$
elements of the underlying set as $x\in S$ instead of $x\in U$.
Similarly, we abuse language referring to $S$ as empty (or nonempty)
if its underlying set $U$ is empty (resp., nonempty), and so on.

Furthermore, we will freely write an instance of a structure using the
same notation. For example, a 1-sorted structure looks like
$S=\structure{U}$ consisting of an underlying set $U$ called its
\emph{carrier}. An example of this would be $\structure{\RR}$ where
the underlying set is the real numbers $\RR$. We could also write this
as $\structure{U = \RR}$ to indicate the component $U$ is equal to
$\RR$ for this structure.

Note that Alan U.\ Kennington's book on differential geometry uses the
notation $S\smt\structure{S,\dots}$ to refer to $S$ as the underlying
set of a structure which we refer to by the same identifier $S$.
\end{remark}

%\DeclareMathSymbol{\QED}{\mathrel}{operators}{"D0}
\begin{remark}[Inheritance and structures]
We will avoid using tuples to describe structures, instead we will use
finite maps in ``Mathematical Reality'' where the keys are the names
of the fields. This allows us to describe \emph{inheritance} of
structures, so that a monoid $\structure{U,\cdot,1}$ \emph{really is}
a magma $\structure{U,\cdot}$.
\end{remark}

\section{Basic Structures}

\begin{definition}[One-sorted and Pointed structures]\mml{struct_0}
We define a \define{One-Sorted} structure to be a structure
$S=\structure{U}$ with an underlying set $U$ called its \define{Carrier}.

We have pointed one-sorted structures. For example, a \define{One Structure}
to be a one-sorted structure equipped with an extra element
$O=\structure{U,1_{U}}$ where the \define{One} is the element $1_{U}\in U$.
We include the identifier/variable used for the underlying set as a
subscript of the one element just for clarity.

A \define{Zero Structure} is a one-sorted structure
$Z=\structure{U,0_{U}}$ equipped with a \define{Zero} element $0_{U}\in U$.
We call an element $z$ of $Z$ \define{Zero} if $z=0_{U}$.

We also have a \define{Zero-One Structure} $W=\structure{U,0_{U},1_{U}}$ which
is both a one structure and a zero structure.
\end{definition}

\begin{remark}[Abusing language]
We will abuse language (and/or notation) as follows:

Let $S$ be a 1-sorted structured. We call it \define{Empty} if the
carrier of $S$ is empty. Furthermore, we will refer to an
\define{Element of $S$} as shorthand for an Element of the carrier of $S$,
a \define{Subset of $S$} is a Subset of the carrier of $S$, and a
\define{Subset-Family of $S$} is a Subset-Family of the carrier of $S$.

Let $X$ be a set. We will speak of a \define{Function} of $S$ to $X$
as a Function of the carrier of $S$ to $X$, and similarly a
\define{Function} of $X$ to $S$ is a Function of $X$ to the carrier of $S$.
Similarly, we may speak of \emph{Partial Functions} from $S$ to $X$
(or from $X$ to $S$).

Now, let $T$ be a 1-sorted structure. We define a \define{Function}
from $S$ to $T$ to be a Function from the carrier of $S$ to the
carrier of $T$. Similarly, we may speak of \emph{Partial Functions}
from $S$ to $T$. So we may speak of binary operators of $S$ and unary
operators of $S$ (and $n$-ary operators of $S$) as analogous operators
on the underlying set of $S$.

A \define{Finite Sequence} of $S$ is a Finite Sequence of the carrier
of $S$. A \define{Sequence} of $S$ is a sequence of the carrier of $S$.
A \define{Many-Sorted Set} of $S$ is a Many-Sorted Set of the
carrier of $S$.

We also write $\id_{S}$ for the Function from $S$ to $S$ equal to
$\id$ on the carrier of $S$.

Let $x$ be an object. We will write ``$x\in S$'' as an abuse of
notation for ``$x\in U$'' where $U$ is the carrier of $S$.

We will call $S$ \define{Finite} if the carrier underlying $S$ is
finite. Similarly, $S$ is \define{Infinite} if the carrier underlying
$S$ is infinite.

We also speak of a \define{Cover} of $S$, when we mean a cover of the
underlying set of $S$.

We will also speak of the \define{Cardinality} of $S$ to be the
cardinality of the underlying set of $S$.

Let $x$ be an object. We will write ``$x(\in S)$'' to stress that $x$ is
an element of $S$.
\end{remark}

\begin{definition}[Inclusion function of 1-sorted structures]\mml{yellow_9:def 1}\label{yellow_9:def 1}
Let $R$ and $S$ be 1-sorted structures. Assume the carriers satisfy
$U_{R}\subset U_{S}$. We define the \define{Inclusion} of $R$ into $S$
to be the function $\incl\colon R\to S$ equal to $\iota:=\id_{U_{R}}$. 
\end{definition}

\begin{definition}[Degenerate zero-one structures]\mml{struct_0:def 8}
Let $D$ be a zero-one structure. We define an attribute, calling
$D$ \define{Degenerate} if $0_{D}=1_{D}$ its one is equal to
zero. Otherwise we call $D$ \define{Nondegenerate}.
\end{definition}

\begin{definition}[2-sorted structures]
We define a \define{2-sorted} structure to be a structure
$T=\structure{U,U'}$ where $U$ is the carrier of $T$ (i.e., $T$ is a
1-sorted structure) and $U'$ is another set (sometimes called the
``carrier'\thinspace'' of $T$).

These structures appear in category theory, where a category structure consists
of a set of objects and a set of morphisms equipped with some extra
structure. 
\end{definition}

\section{Basic Algebraic Structures}

\begin{definition}[Magmas]\mml{algstr_0}
A (multiplicative) \define{Magma} is a one-sorted structure
$\structure{U, \cdot}$ where $U$ is a set called its \define{Carrier}
or \emph{Underlying Set}, and $\cdot\colon U\times U\to U$ is its
\emph{binary operator} called its \define{Multiplication} operator
(sometimes we will use $\star$ or $*$ or $\times$ for the binary
operator). We will often drop the $\cdot$, writing $xy$ instead of
$x\cdot y$.

We call a magma $M$ \define{Commutative} if for all $x,y\in M$ we have
$x\cdot y=y\cdot x$. In this case, it is common to use $+$ instead of
$\cdot$ for the binary operator. In this case, we often refer to it as
an \define{Additive Magma}.
\end{definition}

\begin{definition}[Cancelable elements of an additive magma]
Let $A$ be an additive magma.
\begin{enumerate}
\item\mml{algstr_0:def 3} Let $x\in A$. We call $x$ \define{left additively-cancelable}
  if for all $y$, $z\in A$ we have $x+y=x+z$ implies $y=z$.
\item\mml{algstr_0:def 4} Let $x\in A$. We call $x$ \define{right additively-cancelable}
  if for all $y$, $z\in A$ we have $y+x=z+x$ implies $y=z$.
\item\mml{algstr_0:def 5} Let $x\in A$. We call $x$ \define{additively-cancelable}
  when it is both left and right add-cancelable.
\end{enumerate}
\end{definition}

\begin{definition}[Cancelable additive magma]\label{defn:add-magma:add-cancelable}
Let $A$ be an additive magma.
\begin{enumerate}
\item\mml{algstr_0:def 6} We call $A$ \define{left additively-cancelable}
  if every element of $A$ is left add-cancelable.
\item\mml{algstr_0:def 7} We call $A$ \define{right additively-cancelable}
  if every element of $A$ is right add-cancelable.
\item\mml{algstr_0:def 8} We call $A$ \define{additively-cancelable}
  when it is both left and right add-cancelable.
\end{enumerate}
\end{definition}

\begin{definition}[Monoids]\label{defn:monoid}
A (multiplicative) \define{Monoid} is a multiplicative Magma $M$ such that
\begin{itemize}
\property{Associativity}\mml{group_1:def 3} for all $x,y,z\in U$ we have $x(yz)=(xy)z$
\property{Unital}\mml{group_1:def 1} there exists an element $e\in U$ such that for all
  $x\in U$ we have $xe=x$ and $ex=x$
\end{itemize}
Furthermore, the unital element $e$ is unique, so we will write
$1_{M}$ for the unital element of a monoid $M$ when the binary
operator is multiplication (and $0_{M}$ when the binary operator is
addition). 

It is often more natural to work with monoids than magmas for a
variety of reasons. (Later, we will see that a monoid is really the
same thing as a category with a single object.)
\end{definition}

\begin{remark}[Ambiguity in structure of a monoid]
There is some ambiguity regarding the exact structure of a
monoid. Specifically, is it a double $M=\structure{U,\cdot}$ with this
unital axiom? Then $1_{-}$ is a ``term constructor'' in the ambient
logic (it eats in a magma $M$ and spits out an element $1_{M}\in M$).

On the other hand, we could define a monoid to be a triple
$M=\structure{U,\cdot,1_{U}}$ and have the extra element $1_{U}\in U$ obey the
unital condition. What could go wrong here? Well, an ordered pair is
not an ordered triple, so a monoid could not also be a magma (they are
different objects in ``mathematical reality'').

Thanks to the invention of proof assistants, we have much experience
exploring which formalization reflects mathematical practice, and it
appears the former should be preferred: using ``the smallest structure possible''
has its benefits. Most relevant for us, every monoid \emph{really is}
a magma.
\end{remark}

\begin{definition}[Groups and Inverse Operator]\mml{group_1}
A \define{Group} is a Monoid $G$ such that
\begin{itemize}
\property{Invertibility}\mml{group_1:def 2} For each $x\in G$, there
  exists a $y\in G$ such that $xy=1_{G}$ and $yx=1_{G}$.
\end{itemize}
Associativity makes the element $y$ unique, so we will write this as
$x^{-1}$. The mapping on $G$ sending $x\mapsto x^{-1}$ is called the
\define{Inverse Operator} of $G$.
\end{definition}

\begin{definition}[Loops Structure]\mml{algstr_0}
Following the tradition of Polish logicians, we will define a
\define{Multiplicative Loop Structure} is a multiplicative magma which
is also a one structure $\structure{U,\cdot,1_{U}}$.

An \define{Additive Loop Structure} is an additive magma which is also a zero
structure $\structure{U,+,0_{U}}$.

A (multiplicative) Loop is a (multiplicative) Loop Structure which is
unital and invertible. The One field of the Loop Structure is the
multiplicative identity element.
\end{definition}

\begin{remark}[Multiplicative loops should be deprecated]
Multiplicative loops are not used by the Working Mathematician.
Instead they work with unital multiplicative magmas (\S\ref{defn:monoid}).
This is felt especially when working with rings and algebras, because
not all algebras are unital (e.g., the Banach algebra of complex-valued
continuous functions on a noncompact space).
\end{remark}

\begin{definition}[Complementable elements of additive loops]
Let $A$ be an additive loop.
\begin{enumerate}
\item\mml{algstr_0:def 10} Let $y$ be an element of $A$. We call $y$
  \define{Left complementable} whenever there exists a $y\in A$ such
  that $x+y=0$;
\item\mml{algstr_0:def 11} Let $x$ be an element of $A$. We call $x$
  \define{Right complementable} whenever there exists a $y\in A$ such
  that $x+y=0$;
\item\mml{algstr_0:def 12} Let $x$ be an element of $A$. We call $x$
  \define{Complementable} if $x$ is both left and right complementable.
\end{enumerate}
\end{definition}

\begin{definition}[Complementable additive loops]
Let $A$ be an additive loop.
\begin{enumerate}
\item\mml{algstr_0:def 15} We call $A$ \define{Left complementable} whenever every element of $A$ is left complementable.
\item\mml{algstr_0:def 16} We call $A$ \define{Right complementable} whenever every element of $A$ is right complementable.
\item\mml{algstr_0:def 17} We call $A$ \define{Complementable}
  whenever $A$ is both left and right complementable.
\end{enumerate}
\end{definition}

\begin{definition}\label{defn:loop:unital}
Let $M$ be a nonempty multiplicative loop structure.
\begin{enumerate}
\item\mml{vectsp_1:def 4} We call $M$ \define{Right unital} if
  for all $a\in M$ we have $a\star1=a$;
\item\mml{vectsp_1:def 8} We call $M$ \define{Left unital} if
  for all $a\in M$ we have $1\star a=a$;
\item\mml{vectsp_1:def 6} We call $M$ \define{Well-unital} if for all
  $x\in M$ we have $x\star 1=x$ and $1\star x=x$.
\end{enumerate}
\end{definition}

\begin{definition}[Ringoid, double magma, and double loop structures]\mml{algstr_0}
A \define{Double Magma Structure} is a structure $D=\structure{U,+,\cdot}$
which is both a multiplicative magma and an additive magma.

A \define{Ringoid Structure} 
is a double magma structure and a zero structure $R=\structure{U,\cdot,+,0_{U}}$ which is
a magma in $U$ and $\cdot$, adjoined with a binary operator $+\colon U\times U\to U$
called \emph{Addition},
and a ``zero element'' $0_{U}\in U$.

A \define{Double Loop Structure} is a Ringoid which is also a multiplicative
loop, i.e., it is a structure $\structure{U,\cdot,+,0_{U},1_{U}}$. 
\end{definition}

\begin{definition}\label{defn:double-loop:distributive}
Let $R$ be a nonempty double \strikethrough{loop} magma structure.
\begin{enumerate}
\item\mml{vectsp_1:def 2} We call $R$ \define{Right-distributive} if
  for all $a$, $b$, $c\in R$ we have $a(b+c)=(ab)+(ac)$;
\item\mml{vectsp_1:def 3} We call $R$ \define{Left-distributive} if
  for all $a$, $b$, $c\in R$ we have $(b+c)a=(ba)+(ca)$;
\item\mml{vectsp_1:def 7} We call $R$ \define{Distributive} if for all
  $a$, $b$, $c\in R$ we have both $a(b+c)=(ab)+(ac)$ and $(b+c)a=(ba)+(ca)$;
\end{enumerate}
\end{definition}

\begin{definition}[Pointed and Absorption magmas and monoids]\mml{algstr_0}
We define an \define{Magma with Zero} to consist of the structure
$M=\structure{U,\cdot,0_{U}}$. We further call it an \define{Absorption Magma}
if it is such that
\begin{itemize}
\property{Absorption property} for all $x\in M$, we have
  $x\cdot0_{U}=0_{U}$ and $0_{U}\cdot x=0_{U}$.
\end{itemize}
We define a \define{Monoid with Zero} to be an associative unital
magma with zero. Furthermore, an \define{Absorption Monoid} to be an
absorption magma which is unital and associative.
\end{definition}

\begin{definition}[Left-zeroed and right-zeroed]\label{defn:add-loop:zeroed}
Let $A$ be a nonempty additive loop.
\begin{enumerate}
\item\mml{algstr_1:def 2} We call $A$ \define{Left-zeroed} whenever every $a\in A$ satisfies $0+a=a$.
\item\mml{rlvect_1:def 4} We call $A$ \define{Right-zeroed} whenever every $a\in A$ satisfies $a+0=a$.
\end{enumerate}
\end{definition}

\begin{theorem}\mml{binom:1}
Let $R$ be a nonempty left-distributive right-zeroed double loop structure, and
supposed it is left additively-cancelable {\rm(\S\ref{defn:add-magma:add-cancelable})}.
Then for any $a\in R$, we have $0\star a=0$.
\end{theorem}

\begin{theorem}\mml{binom:2}
Let $R$ be a nonempty right-distributive left-zeroed double loop structure, and
supposed it is right additively-cancelable {\rm(\S\ref{defn:add-magma:add-cancelable})}.
Then for any $a\in R$, we have $a\star 0=0$.
\end{theorem}

\begin{example}[Real numbers]
The real numbers forms a one-sorted structure $\structure{U=\RR}$,
a zero structure $\structure{U=\RR,0_{U}=0}$, a one structure
$\structure{U=\RR,1_{U}=1}$, a zero-one structure, a multiplicative magma
structure under multiplication, an additive magma structure under
addition. Hence it also gives us a double magma, a double loop, a ringoid,
a pointed magma, and every other structure we have defined so far.
\end{example}

