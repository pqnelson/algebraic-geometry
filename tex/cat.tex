%%
%% cat.tex
%% 
%% Made by Alex Nelson <pqnelson@gmail.com>
%% Login   <alex@lisp>
%% 
%% Started on  2025-08-05T09:47:03-0700
%% Last update 2025-08-05T09:47:03-0700
%% 

\chapter{Category Theory}
\needs{graph.tex}

\section{Basic Definitions}

\begin{remark}[Category is a cryptomorphic notion]
There are at least three different ways to axiomatize the notion of a
``category'', each with a different structure. But they are equivalent
in the sense that we can translate one axiomatization faithfully into
another. This quality (having multiple distinct axiomatizations which
are not obviously equivalent) is called \define{Cryptomorphism}.
\end{remark}

\begin{definition}[Category structure with one collection of morphisms]\mml{cat_1}
A \define{Category Structure} extends a multigraph structure $C=\structure{U,U_{1},s,t,\cdot,1}$ where
\begin{itemize}
\item the underlying set $U$ is considered to be the set of objects,
\item $U_{1}$ is the set of morphisms,
\item $s\colon U_{1}\to U$ is the \emph{source map} 
or \emph{dom-map},
\item $t\colon U_{1}\to U$ is the \emph{target map} or
\emph{cod-map} giving us the codomain of a morphism,
\item $\cdot\colon U_{1}\times U_{1}\pto U_{1}$ is a partial function
  of morphisms called \emph{composition}, and
\item $1\colon U\to U_{1}$ is the identity morphism.
\end{itemize}
Usually we write $\Ob(C)$ for the set of objects (as a synonym for $U$
of $C$), and either $\Mor(C)$ or $\Hom(C)$ for the set of morphisms.

When we have $f\in U_{1}$ such that $s(f)=a\in U$ and $t(f)=b\in U$,
we will indicate this by saying ``$f\colon a\to b$ is a morphism of $C$''.
The set of all such morphisms is written $\hom(a,b)\subset U_{1}$. Specifically,
$\hom(a,b)=\{f\in U_{1}\mid s(f)=a\land t(f)=b\}$.

In this case, we define a \define{Category} to consist of a category
structure $\cat{C}$ such that
\begin{itemize}
\property{Category-like} for any morphism $f\in U_{1}$ and $g\in U_{1}$,
  we have $(g,f)\in\cdot$ iff $s(g)=t(f)$
\property{transitive} for any morphism $f\in U_{1}$ and $g\in U_{1}$
  such that $s(g)=t(f)$, we have $s(g\cdot f)=s(f)$ and $t(g\cdot f)=t(g)$
\property{Composition is associative} for any morphism $f\in U_{1}$ and $g\in U_{1}$
  and $h\in U_{1}$
  such that $s(h) = t(g)$ and $s(g)=t(f)$,
  we have $h\cdot(g\cdot f)=(h\cdot g)\cdot f$.
\property{reflexive} for any object $x\in U$, there exists a morphism
  from $x$ to $x$.
\property{with identities} for any object $a\in U$ there exists a
  morphism $i\in U_{1}$ such that for any object $b\in U$ we have
  (i) if there is a morphism from $a$ to $b$, then every morphism
  $g\colon a\to b$ satisfies $g\cdot i=g$, and\hfill\break
  (ii) if there is a morphism from $b$ to $a$, then every morphism
  $f\colon b\to a$ satisfies $i\cdot f=f$.
\end{itemize}
\end{definition}

\begin{remark}
The preceding definition is in the same spirit as Mac~Lane~\cite{maclane1998categories}. It may be traced back
to Grothendieck~\cite[\S4]{grothendieck1961techniques}. This has the
advantage that the definition generalizes to the notion of an
\emph{internal category} object. In fact, when we have a notion of the
category of sets, an internal category object in the category of sets
is precisely the same as the previous definition.
\end{remark}

\begin{remark}[Collections or sets]
We are lying a bit here, because we should define a category to
consist of a ``collection'' of objects and a ``collection'' of
morphisms. These colections may each separately be proper classes or sets.
But it depends on the axiomatic set theory we are working with to
describe mathematics.

This comment is also applicable to the next formalization of a category.
\end{remark}

\begin{definition}[Category structure with a family of collection of morphisms]\mml{altcat_1}
We can define a \define{Category Structure} to extend a graph
structure (\S\ref{def:graphs:graph-structure}) $C=\structure{U,A,\cdot}$
where $U$ is the underlying set (called the \emph{carrier} of $C$),
$A$ is a many-sorted set (\S\ref{def:pboole:many-sorted-set}) called
the \emph{arrows} of $C$, and
the \emph{composition} $\cdot$ is a many-sorted function (\S\ref{def:pboole:many-sorted-function}) from
$A\times A$ to $A$.

When $a\in U$ and $b\in U$, we write $\Hom(a,b)$ as another way of
writing $A_{a,b}$.

We will write $\Ob(C)$ as a synonym for the carrier $U$ of $C$.

We define a \define{Category} to consist of a category structure
$\cat{C}$ such that
\begin{itemize}
\property{transitive} for any objects $a\in U$, $b\in U$, and $c\in U$,
  if $\hom(a,b)\neq\emptyset$ and $\hom(b,c)\neq\emptyset$, then
  necessarily $\hom(a,c)\neq\emptyset$ (namely, the composition of
  $f\colon a\to b$ with $g\colon b\to c$ gives us a morphism
  $g\cdot f\colon a\to c$)
\property{associative} if the composition of $\cat{C}$ is associative,
i.e, for any objects $a\in U$ and $b\in U$ and $c\in U$ and $d\in U$, for any
morphisms $f\colon a\to b$ and $g\colon b\to c$ and $h\colon c\to d$,
we have $(h\cdot g)\cdot f=h\cdot(g\cdot f)$
\property{with units} for any objects $a\in U$ there exists a morphism
$i\in\hom(a,a)$ such that for any object $b\in U$,
 (i) for any morphism $g\colon a\to b$, we have $g\cdot i=g$, and\hfill\break
 (ii) for any morphism $f\colon b\to a$, we have $i\cdot f=f$
\end{itemize}
\end{definition}

\begin{theorem}[The two formalizations are equivalent]
For each category $\cat{C}$ with one collection of morphisms, there
exists a unique category $\cat{C}_{2}$ with a family of collections of
morphisms such that $\Ob(\cat{C})=\Ob(\cat{C}_{2})$ and
$\hom_{\cat{C}}(a,b)=\hom_{\cat{C}_{2}}(a,b)$ for all objects
$a\in\cat{C}$ and $b\in\cat{C}$ $b\in C$,  and the composition of morphisms agrees.
\end{theorem}

This is actually a \emph{metatheorem} since there is no set theoretic
function within our axiomatic set theory describing this one-to-one
correspondence.

\begin{remark}
Without loss of generality, we will simply refer to a category without
worrying about which definition to use.
\end{remark}

\begin{definition}[Common notation]
Let $\cat{C}$ be a category.

Let $x$ be any object.
We will write $x\in\cat{C}$ as an abbreviation for $x\in\Ob(\cat{C})$.

When $x\in\cat{C}$, we will write $\id_{x}\colon x\to x$ for the
identity morphism.

We will also write $\circ$ for the composition of morphisms of $\cat{C}$.

We may write $\hom_{\cat{C}}(a,b)$ or $\cat{C}(a,b)$ for the
collection of morphisms in $\cat{C}$ from $a\in\cat{C}$ to $b\in\cat{C}$.
If $\cat{C}$ is clear from context, we write $\hom(a,b)$ without fear
of ambiguity.
\end{definition}

\begin{definition}[Domain and codomain notation]
Let $\cat{C}$ be a category, let $f\colon a\to b$ be a morphism in
$\cat{C}$. We abuse notation and write $\dom(f)=a=s(f)$ for the source of
$f$, and similarly $\cod(f)=b=t(f)$ for the codomain of $f$.
\end{definition}

\begin{definition}[Large, small, and locally-small categories]
  Let $\cat{C}$ be a category.

  (1) When both the collection of objects of $\cat{C}$ forms a proper
  class and the collection of all morphisms of $\cat{C}$ forms a
  proper class, then $\cat{C}$ is called \define{Large}. If both
  collections are sets, then we call it \define{Small}.

  (2) When the collection of morphisms $\hom(a,b)$ is a set for all
  objects $a\in\cat{C}$ and $b\in\cat{C}$, then we call the category
  $\cat{C}$ \define{Locally-Small}. (It's not hard to see that all
  small categories are also locally-small.)

  (3) If $\mathcal{U}$ is a Grothendieck universe, we call $\cat{C}$
  \define{$\mathcal{U}$-Small} if $\Ob(\cat{C})\in\mathcal{U}$.

These notions depend on the particular foundations chosen for
mathematics, see Shulman~\cite{shulman2008set} for some details.
\end{definition}

\begin{example}[Some finite categories]
We may define the category $\cat{0}$ to consist of the empty
collection of objects and the empty collection of morphisms. This is
usually called the \define{Empty Category}.

We may define the category $\cat{1}$ to consist of a singleton set
denoted $\{*\}$ as its collection of objects, and only the identity
morphism $\id_{*}\colon *\to *$ as its only morphism.
\end{example}

\begin{example}[Delooping a group]
\future{Let $G$ be a group. Then the delooping of $G$ is a category
  $\delooping{G}$ consisting of a single arbitrary object $*$, and
  each $g\in G$ becomes a morphism $g\colon *\to *$. The identity
  element $e\in G$ becomes the identity morphism $\id_{*}$. The
  multiplication operator of $G$ becomes the composition operator of
  $\delooping{G}$.}
\end{example}

\begin{example}[Category of sets]
The category of all sets $\Set$ has its collection of objects be the
proper class of all sets $\Ob(\Set)=\VonNeumannUniverse$, and the
morphisms are functions between sets. This is a locally-small
category, since $\hom(a,b)$ is a set in $\Set$. Note that this also
means that $\hom(a,b)\in\Ob(\Set)$.

This definition is a little ambiguous, because it depends on the exact
axiomatic set theory used. Zermelo set theory \Z/ has
$\Ob(\Set)=\VonNeumannUniverse_{2\omega}$ for example. It also depends
on whether classical logic or intuitionistic logic is used in the
axiomatization, whether the axiom of choice holds or not, whether we
demand predicative mathematics or impredicative mathematics, if the
logic is first-order or higher-order, and so on.
\end{example}

\begin{example}[Category of Ensembles]
Let $U$ be a nonempty set. We define the category $\Ens_{U}$ whose
objects are the elements of $U$, i.e., $\Ob(\Ens_{U})=U$. Let
$a\in\Ens_{U}$ and $b\in\Ens_{U}$, then the set of morphisms from $a$
to $b$ coincides with the set of all functions $\Ens_{U}(a,b)=\Set(a,b)$.

As discussed in Section~\ref{sec:set-theory:universes}, we could
interpret $\SETS$ as all \ZF/ sets in Tarski--Grothendieck set theory,
so we could interpret $\Ens_{\SETS}$ ``as if'' it were really $\Set$.
\end{example}

\begin{example}[Category of small categories]
\future{We may form a category $\Cat$ whose objects are all small categories
and whose morphisms are all functors between categories found in $\Cat$.}

\future{We could form a ``very large'' category $\CAT$ whose objects are all
(small or large) categories and whose morphisms are all functors
between categories found in $\CAT$. We have $\CAT\notin\CAT$ to avoid
Russell-like paradoxes.}
\end{example}

\begin{definition}[Opposite category]
Let $\cat{C}$ be a category. We define the \define{Opposite} of
$\cat{C}$ to be the category $\cat{C}^{\text{op}}$ such that
\begin{enumerate}
\item $\Ob(\cat{C})=\Ob(\op{\cat{C}})$ it has the same objects as $\cat{C}$
\item The composition operator is the same in both categories $\circ_{\op{\cat{C}}}=\circ_{\cat{C}}$
\item The identity-assignment mapping is the same in both categories.
\item The source and target maps are switched, i.e.,
  $\dom_{\op{\cat{C}}}=\cod_{\cat{C}}$ and $\cod_{\op{\cat{C}}}=\dom_{\cat{C}}$.
\end{enumerate}
In particular, this means if $f\colon a\to b$ is a morphism in
$\cat{C}$, then $f\in\op{\cat{C}}(b,a)$.
\end{definition}

\subsection{Functors}

\begin{definition}[Covariant functor]
Let $\cat{C}$ and $\cat{D}$ be categories.
We define a \define{(Covariant) Functor} $F\colon\cat{C}\to\cat{D}$
consists of (i) a function of objects $F_{0}\colon\Ob(\cat{C})\to\Ob(\cat{D})$ 
and (ii) a functor of morphisms $F_{1}\colon\Hom(a,b)\to\Hom(F_{0}(a),F_{0}(b))$
such that
\begin{itemize}
\property{Preserves composition} $F_{1}(g\circ f)=F_{1}(g)\circ F_{1}(f)$
when $g\colon b\to c$ and $f\colon a\to b$ are morphisms in $\cat{C}$,
\property{Preserves identity morphisms} For each object $a\in\cat{C}$,
we have $F_{1}(\id_{a})=\id_{F_{0}(a)}$.
\end{itemize}
Note that it is common to drop subscripts on functors, and simply
write $F(a)$ instead of $F_{0}(a)$, and $F(f\colon a\to b) = F_{1}(f)$.

The two properties defining a covariant functor means that commutative
diagrams are preserved by covariant functors.

When $\cat{C}=\cat{D}$, we call $F\colon\cat{C}\to\cat{C}$ an
\define{Endofunctor} of $\cat{C}$.
\end{definition}

\begin{example}[Identity functor]
Let $\cat{C}$ be a category. We have the identity functor, denoted
$\id_{\cat{C}}\colon\cat{C}\to\cat{C}$, defined on objects
$a\in\cat{C}$ by $\id_{\cat{C}}(a)=a$ and on morphisms $f\colon a\to b$
by $\id_{\cat{C}}(f)=f$.

\begin{proof}[Proof sketch]
This is a covariant functor. We will show it satisfies both properties
of a covariant functor.

Let $f\colon a\to b$ and $g\colon b\to c$ be morphisms in
$\cat{C}$. Then $\id_{\cat{C}}(g\circ f)=g\circ f=\id_{\cat{C}}(g)\circ\id_{\cat{C}}(f)$.
Hence composition is preserved.

Let $a\in\cat{C}$. Then $\id_{\cat{C}}(\id_{a})=\id_{\id_{\cat{C}}(a)}=\id_{a}$.
Hence identity morphisms are preserved.

Hence the result.
\end{proof}
\end{example}

\begin{definition}[Contravariant functor]
Let $\cat{C}$ and $\cat{D}$ be categories.
We define a \define{Contravariant Functor} $F\colon\cat{C}\to\cat{D}$
to be a covariant functor
$\cat{C}^{\text{op}}\to\cat{D}$. Specifically, this means for any
morphisms $f\colon a\to b$ and $g\colon b\to c$ in $\cat{C}$, we have
$F(g\circ f)=F(f)\circ F(g)$.
\end{definition}

\begin{example}[Hom-functors]
Let $\cat{C}$ be a category. Let $a\in\cat{C}$ be a fixed object.
Then $\hom(a,-)$ is a covariant functor with $\hom(a,f\colon b\to c)$
sending $g\colon a\to b$ to $f\circ g$.

We also have a contravariant functor $\hom(-,a)$ which sends
$b\in\cat{C}$ to $\hom(b,a)$, and sends morphisms $h\colon b\to c$ to
$\hom(h,a)\colon\hom(c,a)\to\hom(b,a)$ which are defined by sending
$g\colon c\to a$ to $\hom(h,a)(g)=g\circ h$. Observe $g\circ h\colon b\to a$.

In particular, when $\cat{C}=\Set$ we see that $\hom(-,\{0,1\})$ is
the contravariant powerset functor. How it acts on functions describes
continuous functions in topology (and measurable functions in measure
theory). This is no coincidence.
\end{example}

\begin{definition}[Faithful and Full functors]
Let $F\colon\cat{C}\to\cat{D}$ be a functor.

Then $F$ is called \define{Full} if for
all objects $x\in\cat{C}$ an $y\in\cat{C}$, we have $F\colon\cat{C}(x,y)\onto\cat{D}(F(x),F(y))$
is a surjective function.

We call $F$ \define{Faithful} if for
all objects $x\in\cat{C}$ an $y\in\cat{C}$, we have $F\colon\cat{C}(x,y)\into\cat{D}(F(x),F(y))$
is a injective function.
\end{definition}

\begin{definition}[Embedding of categories]
Let $\cat{C}$ and $\cat{D}$ be categories.
Then an \define{Embedding} of $\cat{C}$ into $\cat{D}$ is a functor
$F\colon\cat{C}\to\cat{D}$ which is both injective on objects (i.e.,
$F\colon\Ob(\cat{C})\into\Ob(\cat{D})$ is an injective function) and faithful.
\end{definition}

\begin{definition}[Composing functors]
Let $F\colon\cat{C}\to\cat{D}$ and $G\colon\cat{D}\to\cat{E}$ be functors.
We define the \define{Composition} operator of functors to give us the
composition of $F$ followed by $G$ to be the functor $G\circ F\colon\cat{C}\to\cat{E}$
such that for each object $x\in\cat{C}$ we have $(G\circ F)(x)=G(F(x))$,
and for each morphism $f\colon x\to y$ in $\cat{C}$ we have
$(G\circ F)(f\colon x\to y)=G(F(f))\colon G(F(x))\to G(F(y))$.
\end{definition}

\begin{definition}[Isomorphism and Autofunctor]
Let $\cat{C}$ and $\cat{D}$ be categories.
An \define{Isomorphism} of $\cat{C}$ into $\cat{D}$ is a functor
$F\colon\cat{C}\to\cat{D}$ such that there exists a functor
$G\colon\cat{D}\to\cat{C}$ such that $G\circ F=\id_{\cat{C}}$ and
$F\circ G=\id_{\cat{D}}$. In this case, we write $G=F^{-1}$ and refer
to $G$ as the \define{Inverse Functor} of $F$.

We call an isomorphism $F\colon\cat{C}\to\cat{C}$ an
\define{Autofunctor} or an \emph{Automorphic Functor}.
\end{definition}

\subsection{Natural Transformation}

\begin{definition}[Natural transformation]
Let $F$, $G\colon\cat{C}\to\cat{D}$ be functors.
We define a \define{Natural Transformation} $\alpha\colon F\To G$
assigns to each object $x\in\cat{C}$ a morphism $\alpha_{x}\colon F(x)\to G(x)$
in $\cat{D}$ (called the \emph{component} or \emph{coordinates} of $\alpha$ at $x$) such
that for any morphism $f\colon x\to y$ in $\cat{C}$, the following
diagram commutes in $\cat{D}$
\begin{equation}
\vcenter{\xymatrix{
F(x) \ar[r]^{F(f)}\ar[d]_{\alpha_{x}} & F(y)\ar[d]^{\alpha_{y}}\\ 
G(x) \ar[r]_{G(f)} & G(y)
}}
\end{equation}
A natural transformation is a morphism of functors.
\end{definition}

\begin{example}[Identity natural transformation]
Let $F\colon\cat{C}\to\cat{D}$ be a functor. The identity natural
transformation $\id_{F}\colon F\To F$ assigns to each object
$x\in\cat{C}$ the identity morphism $\id_{F(x)}$ in $\cat{D}$.
\end{example}

\begin{definition}[Whiskering]
Let $F,G\colon\cat{C}\to\cat{D}$ be functors, let $\alpha\colon F\To G$
be a natural transformation.
\begin{enumerate}
\item Let $H\colon\cat{D}\to\cat{E}$ be a functor. Then
\define{Whiskering} $H$ and $\alpha$ yields the natural transformation
$H\alpha\colon(H\circ F)\to(H\circ G)$ (or $H\circ\alpha$) whose
coordinates at $A$ is $H(\alpha_{A})$.
\item Let $E\colon\cat{B}\to\cat{C}$ be a functor. Then
\define{Whiskering} $\alpha$ and $E$ yields the natural transformation
$\alpha E\colon(F\circ E)\to(G\circ E)$ (or $\alpha\circ E$) whose
coordinate at $A$ is $\alpha_{F(A)}$. 
\end{enumerate}
\end{definition}