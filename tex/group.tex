%%
%% group.tex
%% 
%% Made by Alex Nelson <pqnelson@gmail.com>
%% Login   <alex@lisp>
%% 
%% Started on  2025-08-07T08:16:35-0700
%% Last update 2025-08-07T08:16:35-0700
%% 

\chapter{Group Theory}

\section{Groups}

\begin{definition}[Multiplicative magma]
A \define{Multiplicative Magma} (or just \emph{Magma}) is a structure
$M=\structure{U,\star}$ extending a 1-sorted structure with a
\define{multiplication} operator $\star\colon U\times U\to U$.

We will often write $ab$ instead of $a\star b$, or freely use variants
like $a\cdot b$ or $a\bullet b$. We may also write the magma as a
subscript to the binary operator (i.e., write $a\star_{M}b$ to stress
$\star$ is the binary operator of $M$).
\end{definition}

\begin{definition}[Unital magmas]\mml{group_1:def 1}
Let $M$ be a multiplicative magma. We call $M$ \define{Unital}
when there exists an element $e\in M$ such that for all elements $h\in M$
we have $h\star e=h$ and also $e\star h=h$.
\end{definition}

\begin{definition}[Group-like magmas]\mml{group_1:def 2}
Let $M$ be a multiplicative magma. We call $M$ \define{Group-like}
when there exists an element $e\in M$ such that for all elements $h\in M$
we have
\begin{enumerate}
\item $h\star e=h$, and
\item $e\star h=h$, and
\item there exists an element $g\in M$ such that $h\star g=e$ and also
  $g\star h=e$
\end{enumerate}
\end{definition}

\begin{definition}[Associative magmas]\mml{group_1:def 3}
Let $M$ be a multiplicative magma. We call $M$ \define{Unital}
when for all $x,y,z\in M$ we have $(x\star y)\star z=x\star(y\star z)$.
\end{definition}

\begin{definition}[Group]\mml{group_1}
We see that a \define{Group} is a Group-like associative non-empty
multiplicative magma.
\end{definition}

\begin{theorem}\mml{group_1:1}
Let $M$ be a nonempty magma. Let $r$, $s$, $t$ be elements of $S$.
If $(r\star s)\star t=r\star(s\star t)$ and there exists an element
$e\in M$ such that for all $s'\in M$ we have $s'\star e=s'$ and
$e\star s'=s'$ and there exists an $s''\in M$ such that $s'\star s''=e$
and $s''\star s'=e$, then $M$ is a group.
\end{theorem}

\begin{theorem}\mml{group_1:3}
The magma $\structure{U=\RR,\star=+_{\RR}}$ is associative Group-like.
\end{theorem}

\begin{definition}\mml{group_1:def 4}
Let $G$ be a unital magma. We define the term $1_{G}$ to be the
element of $G$ such that for all $h\in G$ we have $h\star1_{G}=h$ and
$1_{G}\star h=h$.
\end{definition}

\begin{definition}[Notation for inverse of element in group]\mml{group_1:def 5}
Let $G$ be a group, let $h\in G$. We will write $h^{-1}$ for the
element of $G$ such that $h\star h^{-1}=1_{G}$ and $h^{-1}\star h=1_{G}$.
\end{definition}

\begin{theorem}\mml{group_1:5-10}
Let $G$ be a group, let $f$, $g$, $h\in G$ be elements of $G$. Then
all of the following hold:
\begin{enumerate}
\item If $h\star g=1_{G}$ and $g\star h=1_{G}$, then $g=h^{-1}$
\item If $h\star g=h\star f$ or $g\star h=f\star h$, then $g=f$
\item If $h\star g=h$ or $g\star h=h$, then $g=1_{G}$.
\item We have $1_{G}^{-1}=1_{G}$.
\item If $h^{-1}=g^{-1}$, then $h = g$
\item If $h^{-1}=1_{G}$, then $h=1_{G}$.
\end{enumerate}
\end{theorem}

\vfill\break
\begin{theorem}\mml{group_1:12-20}
Let $G$ be a group, let $f$, $g$, $h\in G$ be elements of $G$. Then
all of the following hold:
\begin{enumerate}
\item If $h\star g=1_{G}$, then $h=g^{-1}$ and $g=h^{-1}$
\item We have $h\star f=g$ iff $f=h^{-1}\star g$
\item $f\star h=g$ iff $f=g\star h^{-1}$
\item For any elements $x$, $y\in G$ there exists a $z\in G$ such that
  $x\star z=y$.
\item For any elements $x$, $y\in G$ there exists a $z\in G$ such that
  $z\star x=y$.
\item $(h\star g)^{-1}=g^{-1}\star h^{-1}$
\item $g\star h=h\star g$ iff $(g\star h)^{-1}=g^{-1}\star h^{-1}$
\item $g\star h=h\star g$ iff $g^{-1}\star h^{-1}=h^{-1}\star g^{-1}$
\item $g\star h=h\star g$ iff $g\star h^{-1}=h^{-1}\star g$
\end{enumerate}
\end{theorem}

\begin{definition}\mml{group_1:def 6}
Let $G$ be a group. We define the \define{Inverse Operator} of $G$ to
be the unary operator of $G$ denoted $\cdot_{G}^{-1}$ such that for
all elements $h\in G$ we have $\cdot_{G}^{-1}(h)=h^{-1}$.
\end{definition}

\begin{definition}[Power notation for elements of groups]\mml{group_1:def 8}
Let $G$ be a group. Let $h$ be an element of $G$.

Let $n$ be a natural number. Then we define $h^{n}$ inductively as
$h^{0}=1_{G}$ and $h^{n+1}=h\cdot h^{n}$.

Let $m$ be an integer.
We define the term $h^{m}$ to be
\begin{equation}
h^{m} = \cases{
h^{|m|} & if $m\geq0$\cr
(h^{-1})^{|m|} & if $m < 0$.\cr
}
\end{equation}
\end{definition}

\begin{theorem}[Properties of power notation]\mml{group_1:25-40}
Let $G$ be a group. Let $g$, $h$ be elements of $G$. We have the
following properties of power notation:
\begin{enumerate}
\item $g^{0}=1_{G}$;
\item $g^{1}=g$;
\item $g^{2}=g\cdot g$;
\item $g^{3}=g\cdot g\cdot g$;
\item $g^{2}=1$ iff $g^{-1}=g$;
\item for any $m\in\ZZ$, if $m\leq0$, we have $g^{m}=(g^{|m|})^{-1}$;
\item for any $m\in\ZZ$, $1_{G}^{m}=1_{G}$;
\item for any $m\in\ZZ$, if $m=-1$, then $g^{m}=g^{-1}$ (i.e., the
  power notation is compatible with the notation for the inverse operator);
\item for any $m,n\in\ZZ$, $g^{m+n}=(g^{m})\cdot(g^{n})$;
\item for any $m\in\ZZ$, $g^{m+1}=g^{m}\cdot g$ and $g^{m+1}=g\cdot g^{m}$;
\item for any $m,n\in\ZZ$, $g^{mn}=(g^{m})^{n}$;
\item for any $m\in\ZZ$, $g^{(-m)}=(g^{m})^{-1}$;
\item for any $m\in\ZZ$, $g^{(-m)}=(g^{-1})^{m}$;
\item for any $m\in\ZZ$, if $gh=hg$, then $(gh)^{m}=g^{m}h^{m}$;
\item for any $m\in\ZZ$, if $gh=hg$, then $g(h^{m})=h^{m}g$;
\end{enumerate}
\end{theorem}

\begin{definition}[Order of a group element]
Let $G$ be a group. Let $g$ be an element of $G$.


We\mml{group_1:def 10} say that $g$ is \define{Order Zero} if for all
$n\in\NN_{0}$, if $g^{n}=1_{G}$, then $n=0$.


We define the \define{Order}\mml{group_1:def 11} of $g$ to be the
natural number $\ord(g)$ such that $g^{\ord(g)}=1_{G}$ and for all
$m\in\NN$, $m\neq0$, such that $g^{m}=1_{G}$ we have $\ord(g)\leq m$.
In particular, if $g$ is of order zero, then $\ord(g)=0$.
\end{definition}

\begin{definition}[Cardinality of a finite group]\mml{group_1}
Let $G$ be a finite 1-sorted structure. We define $|G|$ to be the
natural number equal to the cardinality of its underlying set.

In particular,\mml{group_1:45} nonempty finite 1-sorted structures $G$ have $|G|\geq1$.
\end{definition}

\begin{definition}[Commutative multiplicative magmas]\mml{group_1:def 12}
Let $M$ be a multiplicative magma. We say $M$ is \define{Commutative}
if for all $x$, $y\in M$, we have $xy=yx$.
\end{definition}

\begin{remark}[Abelian groups reserved for additive notation]
We will follow \Mizar/'s conventions and reserve the phrase ``Abelian Group''
for the additive loop $\structure{U,+,0}$ which forms a commutative group.
This notion (of an Abelian group) should be understood as generalizing
a vector space by forgetting the scalar multiplication operation.

When we want to discuss groups which are commutative, we will speak of
``commutative groups''. If we wanted to discuss the classification of
finite simple groups, for example, then we would discuss ``finite
commutative groups''.
\end{remark}

\begin{definition}[Unity-preserving maps]\mml{group_1:def 13}
Let $G$ and $H$ be [unital] multiplicative magmas.
Let $f\colon G\to H$ be a function.
We say $f$ is \define{Unity-Preserving} if $f(1_{G})=1_{H}$.
\end{definition}

\section{Subgroups}

\begin{definition}[Inverting a subset of a group]\mml{group_2:def 1}
Let $G$ be a group. Let $A\subset G$ be a subset.
We define the \define{Inverse} of $A$ to be the subset $A^{-1}$ of $G$
equal to $A^{-1}:=\{g^{-1}\mid g\in A\}$.

We see that $(A^{-1})^{-1}=A$.
\end{definition}

\begin{theorem}\mml{group_2:2}
Let $G$ be a group, let $A$ be a subset of $G$, let $x\in G$.
Then $x\in A^{-1}$ if and only if there exists a $g\in A$ such that $x=g^{-1}$.
\end{theorem}

\begin{theorem}[Small examples of inverted subsets of groups]\mml{group_2:4-7}
Let $G$ be a group. Let $g$, $h$ be elements of $G$. Then:
\begin{enumerate}
\item $\{g\}^{-1}=\{g^{-1}\}$
\item $\{g,h\}^{-1}=\{g^{-1},h^{-1}\}$
\item $\{\}^{-1}=\{\}$
\item $G^{-1}=G$
\item for any subset $A\subset G$, we have $A\neq\emptyset$ iff $A^{-1}\neq\emptyset$
\end{enumerate}
\end{theorem}

\begin{definition}[Multiplying subsets of a multiplicative magma]\mml{group_2:def 2}
Let $M$ be a nonempty multiplicative magma. Let $A$ and $B$ be subsets of $M$.
We define the multiplication of $AB$ to be the subset of $M$ equal to
$\{ab\mid a\in A,b\in B\}$.

Further, when $M$ is commutative, we see $AB=BA$.

If necessary, we may sometimes write $A\star B$ or $A\cdot B$ for the
multiplication of the sets.
\end{definition}

\begin{theorem}[Properties of multiplying subsets of magmas]\mml{group_2:9-18}
Let $M$ be a nonempty multiplicative magma.
Let $A$, $B$, $C$ be subsets of $M$. Then the following are all true:
\begin{enumerate}
\item $A\neq\emptyset$ and $B\neq\emptyset$ if and only if $AB\neq\emptyset$;
\item if $M$ is associative, then $(AB)C=A(BC)$;
\item if $M$ is a group, then $(AB)^{-1}=B^{-1}A^{-1}$;
\item $A(B\cup C)=(AB)\cup(AC)$;
\item $(A\cup B)C=(AC)\cup(BC)$;
\item $A(B\cap C)\subset(AB)\cap(AC)$;
\item $(A\cap B)C\subset(AC)\cap(BC)$;
\item $\emptyset\star A=\emptyset$ and $A\star\emptyset=\emptyset$;
\item if $M$ is a group and $A\neq\emptyset$, then $M\star A=M$ and
  $A\star M=M$.
\end{enumerate}
\end{theorem}

\begin{definition}[Multiplying a subset of a magma by an element]\mml{group_2:def 3,4}
Let $M$ be a nonempty multiplicative magma.
Let $g$ be an element of $M$.
Let $A$ be a subset of $M$.
Then we define the terms:
\begin{enumerate}
\item $gA$ to be the subset of $M$ equal to $gA := \{g\}\star A$;
\item $Ag$ to be the subset of $M$ equal to $Ag := A\star\{g\}$;
\end{enumerate}
\end{definition}

\begin{theorem}[Characterizing multiplying a subset of a magma by an element]\mml{group_2:27}
Let $M$ be a nonempty multiplicative magma.
Let $A$ be a subset of $M$.
Let $g$ be an element of $M$.
For any object $x$, we have $x\in gA$ if and only if there exists an
element $h$ of $M$ such that $x=gh$ and $h\in A$.
\end{theorem}

\begin{theorem}[Characterizing multiplying a subset of a magma by an element]\mml{group_2:28}
Let $M$ be a nonempty multiplicative magma.
Let $A$ be a subset of $M$.
Let $g$ be an element of $M$.
For any object $x$, we have $x\in Ag$ if and only if there exists an
element $h$ of $M$ such that $x=hg$ and $h\in A$.
\end{theorem}

\begin{theorem}[Associativity of multiplying subsets and elements of magmas]\mml{group_2:29-37}
Let $M$ be an associative nonempty multiplicative magma.
Let $A$, $B$ be subsets of $M$.
Let $g$, $h$ be elements of $M$. Then the following all hold:
\begin{enumerate}
\item $(gA)B=g(AB)$
\item $(Ag)B=A(gB)$
\item $(AB)g=A(Bg)$
\item $(gh)A=g(hA)$
\item $(gA)h=g(Ah)$
\item $(Ag)h=A(gh)$
\item $\emptyset\star g=\emptyset$ and $g\star\emptyset=\emptyset$
\item $Mg=M$ and $gM=M$
\item If $M$ is unital, $1_{M}A=A$ and $A=A1_{M}$.
\end{enumerate}
\end{theorem}

\begin{definition}[Subgroup]\mml{group_2:def 5}
Let $G$ be a group. We define [the soft type of] a \define{Subgroup}
of $G$ to be a group $H$ such that
\begin{enumerate}
\item the carrier of $H$ is a subset of $G$: $U_{H}\subset U_{G}$;
\item the binary operator is inherited from $G$:
  $\star_{H}=\star_{G}|_{H\times H}$.
\end{enumerate}
We will use the notation $H\subgroup G$ to indicate $H$ is a subgroup
of $G$.

This is well-defined because $G$ is a subgroup of itself.
\end{definition}

\begin{theorem}[Subgroups of finite groups are finite]\mml{group_2:39}
Let $G$ be a finite group. Let $H$ be a subgroup of $G$.
Then $H$ is finite.
\end{theorem}

\begin{theorem}[Elements of subgroup belong to parent group]\mml{group_2:40-42}
Let $G$ be a group. Let $H$ be a subgroup of $G$.
If $x$ is an element of $H$, then $x$ is an element of $G$.
This means $x\in H$ implies $x\in G$.
\end{theorem}

\begin{theorem}[Compatibility of multiplying elements of subgroup]\mml{group_2:43}
Let $G$ be a group. Let $H\subgroup G$ be a subgroup of $G$.
Let $g_{1}$, $g_{2}$ be elements of $G$.
Let $h_{1}$, $h_{2}$ be elements of $H$.
If $g_{1}=h_{1}$ and $g_{2}=h_{2}$, then $g_{1}g_{2}=h_{1}h_{2}$.
\end{theorem}

\begin{theorem}[Identity element of subgroups]\mml{group_2:44-47}
Let $G$ be a group.
Let $H\subgroup G$, $H_{1}\subgroup G$, and $H_{2}\subgroup G$ be subgroups of $G$.
Then the following all hold:
\begin{enumerate}
\item $1_{H}=1_{G}$
\item $1_{H_{1}}=1_{H_{2}}$
\item $1_{G}\in H$
\item $1_{H_{1}}\in H_{2}$
\end{enumerate}
\end{theorem}

\begin{theorem}[Inverting elements of subgroups]\mml{group_2:48}
Let $G$ be a group.
Let $H\subgroup G$ be a subgroup of $G$.
Let $h$ be an element of $H$, let $g$ be an element of $G$.
If $g=h$, then $g^{-1}=h^{-1}$.
\end{theorem}

\begin{theorem}[Multiplying elements of subgroup]\mml{group_2:50}
Let $G$ be a group. Let $g_{1}$ and $g_{2}$ be elements of $G$.
Let $H\subgroup G$ be a subgroup.
If $g_{1}\in H$ and $g_{2}\in H$, then $g_{1}g_{2}\in H$.
\end{theorem}

\begin{theorem}[Group inversion operator closed on subgroups]\mml{group_2:51}
Let $G$ be a group. Let $H\subgroup G$. Let $g$ be an element of $G$.
If $g\in H$, then $g^{-1}\in H$.
\end{theorem}

\begin{theorem}[Subgroups of commutative groups are commutative]\mml{group_2:53}
Let $G$ be a commutative group. Let $H\subgroup G$ be a subgroup.
Then $H$ is commutative.
\end{theorem}
(This is a registered fact in \texttt{GROUP\_2}.)

\begin{theorem}[Every group is a subgroup of itself]\mml{group_2:54}
Let $G$ be a group. Then $G$ is a subgroup of $G$.
\end{theorem}

\begin{theorem}[Transitivity of subgroups]\mml{group_2:56}
Let $G_{1}$, $G_{2}$, $G_{3}$ be groups.
If $G_{1}$ is a subgroup of $G_{2}$, and $G_{2}$ is a subgroup of $G_{3}$,
then $G_{1}$ is a subgroup of $G_{3}$.
\end{theorem}