%%
%% graphs.tex
%% 
%% Made by Alex Nelson <pqnelson@gmail.com>
%% Login   <alex@lisp>
%% 
%% Started on  2025-08-05T10:04:20-0700
%% Last update 2025-08-05T10:04:20-0700
%% 

\chapter{Graph Theory}
\needs{pboole.tex}

\begin{definition}[Graph structure]\label{def:graphs:graph-structure}\mml{altcat_1}
We define a \define{Graph Structure} to extend a 1-sorted structure
$G=\structure{U,A}$ consisting of an underlying set $U$ (as we do with
a 1-sorted structure) and a many-sorted set of \emph{edges} or
\emph{arrows} $A$ indexed by $U\times U$
(\S\ref{def:pboole:many-sorted-set}).
\end{definition}

\begin{remark}[First-class structure]
Lee and Rudnicki~\cite{lee2007alternative} have observed formalizing
graph theory in \Mizar/ is more natural when working with ``first
class structures''. That is to say, instead of working with structures
as model theorists describe them (finite maps in the metatheory), it
is more natural for graphs to be described using finite maps
\emph{in the object theory!}
\end{remark}

\begin{definition}[Multigraph structure]\mml{graph_1}
A \define{Multigraph Structure} extends a 2-sorted structure
$M=\structure{U,U_{1},s,t}$ where $U$ is the underlying set of $M$
called its carrier (inherited from 1-sorted structures) and $U_{1}$ is
the set inherited from 2-sorted structures, $s\colon U_{1}\to U$ is a
function called the \emph{Source} mapping, and $t\colon U_{1}\to U$
is a function called the \emph{Target} mapping.
\end{definition}